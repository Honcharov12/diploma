%\anonsection{Вступ}
\anonsection{ВСТУП}
\addcontentsline{toc}{section}{Вступ}

%\vspace{1.5em}
\hspace*{26pt} З кожним роком все більшу роль відіграють статистичні методи та алгоритми для аналізу різноманітних даних.
Метою даної дипломної роботи є дослідження часового ряду тестувальної системи DOTS для подальшого прогнозу.

Тестувальна система DOTS (Docker-oriented testing system) застосовується для автоматизації тестування алгоритмічних задач і є найбільшою автоматизованою системою перевірки задач з інформатики у харківській області, та однією з найбільших в Україні. Щорічно на базі цієї системи проходят олімпіади, турніри та різноманітні курси. І, з кожним роком, кількість користувачів зростає, а з ними і кількість серверів необхідних для нормального функціонування системи. І було б дуже доречно вміти передбачати кількість потрібних серверів на наступний день, щоб їх не було і замало, коли користувачі відчувають дискомфорт через занадто довгі черги задач, і щоб їх не було занадто багато, і не доводилось платити за сервери, які не праціють.

В даній роботі досліджуються дані, які складаються з числа щоденних відправлень завантажених рішень.
Саме в цьому мають стати в нагоді методи з аналізу часових рядів.

Для аналізу структури ряду та побудови прогнозу в дипломній роботі були використані методи ARIMA (autoregressive integrated moving average) та SSA (singular spectrum analysis).

За умови їх використання були вирішені наступні задачі:

\begin{enumerate}
	\item підготовка вихідних даних;
	\item аналіз структури часового ряду;
	\item аналіз точності побудованих моделей (ARIMA);
	\item прогнозування;
	\item порівняння роботи методів.
\end{enumerate}

