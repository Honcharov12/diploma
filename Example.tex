\documentclass[a4paper,12pt]{scrartcl}
\usepackage{pscyr}
\usepackage[14pt]{extsizes}
\usepackage{cmap}
\normalfont\normalsize
\usepackage[T2A]{fontenc}
\usepackage[utf8]{inputenc}
\usepackage[english,russian,ukrainian]{babel}
\usepackage{amssymb,amsfonts,amsmath,amsthm,mathtext,cite,enumerate,float}
\usepackage{indentfirst}
\usepackage{misccorr}
\usepackage{ccaption}
\usepackage{graphicx}
\usepackage{enumitem}
\usepackage{chngcntr}
\usepackage{makeidx}
\usepackage{framed}
\usepackage{lastpage}
\usepackage{longtable}
\usepackage{totcount}
\usepackage{etoolbox}
\usepackage{tabularx}
\usepackage{dsfont}
\usepackage{sectsty}
\usepackage{multirow}
\usepackage{gensymb}
\sectionfont{\fontsize{14}{14}\normalfont\bfseries}
\subsectionfont{\fontsize{14}{14}\normalfont\bfseries}
\subsubsectionfont{\fontsize{14}{14}\normalfont\bfseries}
\frenchspacing
\linespread{1.3}
\usepackage{geometry}
\geometry{left=3cm}
\geometry{right=1cm}
\geometry{top=2cm}
\geometry{bottom=2cm}
\geometry{footskip=\dimexpr\headsep+\ht\strutbox\relax}
\usepackage{titletoc}
\titlecontents{chapter}[0em]{\vskip12pt\bfseries\normalfont}
{\thecontentslabel\enspace}
{\hspace{1.05em}}
{\hfill\contentspage}[\vskip 3pt]
\titlecontents{section}[0em]{\normalfont}
{\thecontentslabel\enspace}
{}
{\titlerule*[1pc]{.}\quad\contentspage}[\vskip 4pt]
\titlecontents{subsection}[1em]{\normalfont}
{\thecontentslabel\enspace}
{}
{\titlerule*[1pc]{.}\quad\contentspage}[\vskip 3pt]
\dottedcontents{chapter}[1.6em]{}{1.6em}{1pc}
\setlist{nolistsep}
\setlength\parindent{5ex}
\graphicspath{{Mal/}}
\makeatletter
\renewcommand{\@biblabel}[1]{#1.}
\renewcommand*\verbatim@font{%
\normalfont\ttfamily
\hyphenchar\font\m@ne
\let\do\do@noligs\verbatim@nolig@list} 
\makeatother
\makeatletter
\newcommand{\verbatimfont}[1]{\def\verbatim@font{#1}}
\makeatother
\newtheoremstyle{algor}{3pt}{3pt}{}{5ex}{\bfseries}{.}{0.5em}{}
\theoremstyle{algor}
\newtheorem{algor}{Листинг}[section]
\counterwithin{table}{section}
\counterwithin{figure}{section}
\counterwithin{equation}{section}
\captiondelim{ - } % після точки стоїть пробіл!
%\captionnamefont{\large}
%\captiontitlefont{\large}
\makeindex
\makeatletter
\renewcommand{\@oddfoot}{\hfil \small \arabic{page}\hfil}
\renewcommand{\@evenfoot}{\hfil \small \arabic{page}\hfil}
\makeatother
\changecaptionwidth
\captionwidth{0.9\textwidth}
\captionstyle{\centering}
\usepackage{fancyhdr}
\pagestyle{fancy}
\fancyhf{}
\fancyhf[rh]{\thepage}% нумерація сторінок праворуч зверху
\renewcommand\headrulewidth{0pt}
\makeatletter % вирівнювання нижнього індексу
\newcommand{\subalign}[1]{%
  \vcenter{%
    \Let@ \restore@math@cr \default@tag
    \baselineskip\fontdimen10 \scriptfont\tw@
    \advance\baselineskip\fontdimen12 \scriptfont\tw@
    \lineskip\thr@@\fontdimen8 \scriptfont\thr@@
    \lineskiplimit\lineskip
    \ialign{\hfil$\m@th\scriptstyle##$&$\m@th\scriptstyle{}##$\crcr
      #1\crcr
    }%
  }
}
\makeatother
\newcounter{totfigures}
\newcounter{tottables}
\newcounter{totreferences}
\newcounter{totalgor}
\pretocmd{\section}{\addtocounter{totfigures}{\value{figure}}}{}{}
\pretocmd{\section}{\addtocounter{tottables}{\value{table}}}{}{}
\pretocmd{\section}{\addtocounter{totalgor}{\value{algor}}}{}{}
\pretocmd{\bibitem}{\addtocounter{totreferences}{1}}{}{}
\makeatletter
\AtEndDocument{%
\immediate\write\@mainaux{%
  \string\gdef\string\totfig{\number\value{totfigures}}%
  \string\gdef\string\tottab{\number\value{tottables}}% 
  \string\gdef\string\totalg{\number\value{totalgor}}% 
  \string\gdef\string\totref{\number\value{totreferences}}%
}%
}

\renewcommand\section{\@startsection {section}{1}{\z@}%
                                   {-3.5ex \@plus -1ex \@minus -.2ex}%
                                   {2.3ex \@plus.2ex}%
                                   {\centering\normalfont\bfseries}}


\usepackage{secdot}
\RequirePackage{caption}
\DeclareCaptionLabelSeparator{defffis}{ -- } % Разделитель
\captionsetup[table]{justification=raggedright, labelsep=defffis, format=plain, singlelinecheck=false} % Подпись таблицы слева
\captionsetup[figure]{justification=centering, labelsep=defffis, format=plain} % Подпись рисунка по центру

% Костыль для содержания
\usepackage[compact,explicit]{titlesec}
%\usepackage{etoolbox}

\makeatletter
\patchcmd{\ttlh@hang}{\parindent\z@}{\parindent\z@\leavevmode}{}{}
\patchcmd{\ttlh@hang}{\noindent}{}{}{}
\makeatother


\titleformat{\section}{\bfseries}{}{12.5mm}{\centering{\thesection\quad\bfseries{#1}}\vspace{1.5em}}
\titleformat{\subsubsection}[block]{\vspace{1em}\bfseries\normalsize}{}{12.5mm}{\thesubsubsection\;\bfseries#1\vspace{1em}}
\titleformat{\subsection}[block]{\bfseries\vspace{1pt}}{}{12.5mm}{\thesubsection\bfseries\;#1\vspace{1em}}

\titlespacing{\subsection}{0\parindent}{0.5em}{0em}
\titlespacing{\subsubsection}{0\parindent}{-0.5em}{0em}

\makeatother

% Секции без номеров (введение, заключение...), вместо section*{}
\newcommand{\anonsection}[1]{
    %\phantomsection % Корректный переход по ссылкам в содержании
    \paragraph{\centerline{{#1}}\vspace{1.5em}}
    %\addcontentsline{toc}{section}{\uppercase{#1}}
}

\makeatletter
\def\hlinewd#1{%
  \noalign{\ifnum0=`}\fi\hrule \@height #1 \futurelet
   \reserved@a\@xhline}
\makeatother

\usepackage{tocloft}
\renewcommand{\cfttoctitlefont}{\bfseries\hspace{0.38\textwidth}} % СОДЕРЖАНИЕ
\renewcommand{\cftsecfont}{\hspace{0pt}}            % Имена секций в содержании не жирным шрифтом
\renewcommand\cftsecleader{\cftdotfill{\cftdotsep}} % Точки для секций в содержании
\renewcommand\cftsecpagefont{\mdseries}             % Номера страниц не жирные
\setcounter{tocdepth}{3}                            % Глубина оглавления, до subsubsection

\usepackage{fancyhdr}
\pagestyle{fancy}
\fancyhf{}
\fancyhead[R]{\textrm{\thepage}}
\fancyheadoffset{0mm}
\fancyfootoffset{0mm}
\setlength{\headheight}{17pt}
\renewcommand{\headrulewidth}{0pt}
\renewcommand{\footrulewidth}{0pt}
\fancypagestyle{plain}{ 
    \fancyhf{}
    \rhead{\thepage}
}

% Списки
\usepackage{enumitem}

\begin{document}
\def\figurename{Рисунок}
\def\tablename{Таблиця}
\verbatimfont{\ttfamily}
\let\normalint\int % PS
\def\int{\displaystyle\normalint} %PS
\let\normaliint\iint % PS
\def\iint{\displaystyle\normaliint} %PS
\let\normaliiint\iiint % PS
\def\iiint{\displaystyle\normaliiint} %PS
\let\normalsum\sum % PS
\def\sum{\displaystyle\normalsum} %PS
\renewcommand{\contentsname}{ЗМІСТ}

\pretolerance10000

\setlist[enumerate,itemize]{listparindent=18mm, leftmargin=18mm} % Отступы в списках
%\anonsection{Реферат}
\label{sec:Refer}


\hspace*{26pt} У роботі проводиться розрахунок площі криволінійного паркану змінної висоти. Висота задається як функція декартових координат точки паркана.

Пошук алгоритмів розв’язання задачі виконаний на сайті \cite{GoogleWebPage}. Для обчислення площі використовуються криволінійні інтеграли 1-го роду \cite{Senchuk2006}. Чисельні приклади розв’язані у MATLAB \cite{Anufriev2005}. Використовувалася також сторінка \cite{IglinWebPage}.

\vspace{0.25cm}

Іл.\,\totfig. Табл.\,\tottab. Лист.\,\totalg. Бібліогр.:\,\totref\,назв.



%\newpage
%\tableofcontents
%\newpage
%%\anonsection{Перелік позначень та скорочень}
\anonsection{ПЕРЕЛІК ПОЗНАЧЕНЬ ТА СКОРОЧЕНЬ}
\addcontentsline{toc}{section}{Перелік позначень та скорочень}


\hspace*{26pt}


%\newpage
%%\anonsection{Вступ}

\begin{center}
\textbf{ВСТУП}
\end{center}
\addcontentsline{toc}{section}{Вступ}

%\hspace*{26pt}
Ми живемо в часи, коли інформаційні технології почали займати визначальну роль у житті людства. Зараз важко уявити життя без телефонів, планшетів, комп'ютерів, розумних годинників та інших девайсів, за допомогою яких можна покращити якість сна, автоматизувати процеси на виробництві, дізнатись думку населення з певного питання та безліч інших сфер діяльності в яких застосовуються інформаційні технології. Завдяки зростанню обчислювальних потужностей та об'ємів зберігаємих даних став можливим новий стрибок у розвитку машинного навчання, завдяки якому, в свою чергу, технології почали ще краще поліпшувати наше життя. В останні роки дуже стрімко розвиваються нейронні мережі (neural networks) та глибинні нейронні мережі (deep neural networks, DNN). Існує багато різноманітних методологій та підрозділів нейронних мереж, одним з яких є самонавчання.

Термін самонавчання (self-supervised learning, SSL) використовувався у різних контекстах та сферах, таких як навчання уявлень (representation learning), нейронні мережі, робототехніка, обробка природних мов (natural language processing, NLP) та навчання з посиленням (reinforcement learning). У всіх випадках основна ідея полягає в тому, щоб автоматично генерувати якийсь контрольний сигнал для вирішення якогось завдання (як правило, вивчати подання даних або автоматично розмічати набір даних) [1].

В типовiй SSL задачi ми на входi маємо величезну нерозмiчену вибiрку, SSL-завдання полягае у тому, що необхiдно розробити алгоритм, що для кожного об’єкта сформує псевдо-мiтку (pseudo label), але нас цікавить не стільки якість рішення придуманого нами завдання (її називають pretext task), скільки представлення (representation) об'єктів, яке буде вивчено в процесі її рішення. Це представлення можна в подальшому використовувати вже при вирішенні будь-якої задачі з мітками, яку називають наступною задачею (downstream task). Одна з головних причин самонавчання $-$ невеликий обсяг розмічених даних. На відміну від навчання з частково розміченими даними в самонавчанні використовуються абсолютно довільні нерозмічену дані, що не мають відношення до розв'язуваної задачі [2].

Сучасна обробка тексту (NLP) приблизно на 80\% складається з самонавчання. Наприклад, за допомогою самонавчання можна знайти майже всі представлення слів (а також текстів). Наприклад, в класичному алгоритмі word2vec беруть нерозмічений корпус текстів, потім самі придумують завдання з мітками (по сусіднім словами передбачити центральне слово або, навпаки, по центральному передбачити сусідні із ним), навчають на цьому завданні просту нейронну мережу, в результаті виходять векторні уявлення слів, які вже використовуються в інших, ніяк не пов'язаних з попередньою, задачах. Часто такий підхід називають також трансферним навчанням (transfer learning, TL). Вцілому, трансферне навчання більш широке поняття $-$ коли модель, навчену для вирішення однієї задачі, використовують для вирішення іншої. У самонавчанні важливо, щоб розмітка в попередньої задачі (псевдо-мітки) отримувалась автоматично. 
SSL і TL мають переваги та недоліки. Важко судити, чи один кращий за інший. Існуючі емпіричні результати показують, що в певних завданнях SSL перевершує рівень TL; в інших завданнях TL працює краще, ніж SSL [3].


%Наприклад, представлення, отримані методом Cove, в якому використовується кодер для завдання машинного перекладу, будуть трансферними, але не отриманими за допомогою самонавчання.

Останнім часом, набирає популярність підрозділ самонавчання $-$ порівняльне навчання (constrastive learning).

Метою даної дипломної роботи є дослідження роботи алгоритмів порівняльного навчання Deep InfoMax та Momentum Contrast.

Були поставлені наступні задачі:

\begin{enumerate}
	\item вибір даних для аналізу роботи алгоритмів;
	\item реалізація методів Deep InfoMax та Momentum Contrast;
	\item дослідження роботи вищеназваних методів на обраних даних;
	\item порівняння роботи алгоритмів.
\end{enumerate}


%\newpage
%%\section{Огляд літературних джерел}
\section{ОГЛЯД ЛІТЕРАТУРНИХ ДЖЕРЕЛ}
\label{sec:Theory}

Самонавчання дає нам змогу безкоштовно використовувати різноманітні позначки, що постачаються з даними. Мотивація досить проста: створення набору даних із чіткими мітками є дорогим, але постійно створюються немічені дані. Щоб використовувати набагато більшу кількість немаркованих даних, одним із способів є правильне встановлення цілей навчання, щоб отримати контроль від самих даних.

До основних понять самоначання можна віднести:

\begin{enumerate}
	\item Попередня задача (pretext task) $-$ задача з штучно створеними мітками (псевдо-мітками), на якій навчається модель, щоб вивчити хороші вистави (representations) об'єктів. Наприклад, є надія, що в нейронної мережі на попередньої задачі навчаться початкові і середні шари, їх потім заморожують і навчають останні шари.
	\item Псевдо-мітки (pseudo labels) $-$ мітки, які виходять автоматично, без ручного розмітки, але навчання яким сприяє формуванню хороших представлень.
	\item Подальша задача (downstream task) $-$ задача на якій перевіряють якість отриманих уявлень. Майже у всіх експериментах в статтях по самонавчання на отриманих ознакових представленнях в наступних завданнях навчають прості моделі: логістичну регресію або метод найближчого сусіда. Таким чином, самонавчання - це напрямок в глибокому навчанні, який прагне зробити глибоке навчання процедурою попередньої обробки даних, тобто мережі потрібні для формування ознак, вони навчаються на дешевій розмітці великих наборів спочатку нерозмічену даних, а сама задача вирішується простою моделлю.
\end{enumerate}

\subsection{Порівняльне навчання}

В останні роки значно менше робіт з вибору попередніх завдань, оскільки основним напрямком в самообучении стало порівняльне навчання (Contrastive Learning). На вхід нейромережі подається пара об'єктів і вона визначає, схожі вони чи ні. Якщо об'єкт подається як аугментований патч з зображення, схожі повинні бути патчі з одного зображення, а не схожі $-$ з різних. В якості оптимізуємих функцій використовується взаємна інформація або пов'язані з нею функції, наприклад, її нижня оцінка InfoNCE :

\begin{equation}\label{eq:infonce}
-E_{x}\left[ \log{\frac{exp(f(x)^{T}f(x^{+}))}{exp(f(x)^{T}f(x^{+}))+\sum_{j=1}^{N-1}exp(f(x)^{T}f(x_{j}))}} \right],
\end{equation}

\noindent де $x$ $-$ обраний патч, званий якірним, \newline
\hspace*{15pt} $x^{+}$ $-$ схожий на нього, \newline
\hspace*{15pt} $x_{j}$ $-$ несхожий (їх $N-1$ штука), \newline
\hspace*{15pt} $f()$ $-$ кодер об'єкту (подання його у вигляді речового вектора). 

\vspace{1.5em}

Така функція помилки майже перетворюється в Triplet Loss, якщо використовувати тільки 1 негативний приклад.

\subsection{Deep InfoMax}

Окреслимо загальні параметри навчання кодеру для максимізації взаємної інформації між його входом і виходом. Нехай $X$ і $Y$ є областю і діапазоном неперервної та (майже скрізь) диференційованої параметричної функції, $E_{\psi}: X \rightarrow Y$ з параметрами $\psi$ (наприклад, нейронна мережа). Ці параметри визначають сімейство кодерів $E_{\Psi} = {E_{\phi}}_{\psi \in \Psi}$ над $\Psi$. Припустимо, що нам дано набір навчальні приклади на вхідному просторі, $X$: $\mathbf{X} := {x^{(i)} \in X}^{N}_{i = 1}$, з емпіричним розподілом ймовірностей $\mathbb{P}$. Визначимо $\mathbb{U_{\psi,P}}$ як граничний розподіл, індукований проштовхуванням зразків від $\mathbb{P}$ через $\mathbb{E_{\psi}}$. Тобто, $\mathbb{U_{\psi, P}}$ $-$ розподіл за кодуваннями $y \in Y$, отриманий шляхом вибірки спостережень $x \sim X$, а потім вибірки $y \sim E_{\psi}(x)$.

Приклад кодера даних зображень наведено на рис. \ref{fig:deepinfo1}, який буде використаний у наступних розділах, але цей підхід можна легко адаптувати для тимчасових даних. 

\vspace{1em}

\begin{figure}[h]
  \includegraphics[width=\textwidth, height=5cm, natwidth=277, natheight=144]{Mal/deepinfo1.jpg}
  \caption{Базова модель кодера в контексті даних зображень.}
  \label{fig:deepinfo1}
\end{figure}

Подібно до принципу оптимізації infomax, можна стверджувати, що кодер повинен навчатися відповідно до таких критеріїв:

\begin{enumerate}
	\item Максимізація взаємної інформації. Необхідно знайти набір параметрів $\psi$, таких, щоб взаємна інформація $L(X; E_{\psi} (X))$ була максимальною. Залежно від кінцевої цілі, це максимізація може бути здійснена за повним входом, $X$ або деякою структурованою або “локальною” підмножиною.
	\item Статистичні обмеження. Залежно від кінцевої мети для подання, граничні $\mathbb{U}_{\psi, \mathbb{P}}$ повинні відповідати попередньому розподілу, $\mathbb{V}$. Грубо кажучи, це може бути використано для заохочення вихідних даних кодера до бажаних характеристик (наприклад, незалежності).
\end{enumerate}

Формулювання цих двох цілей називається Deep InfoMax (DIM).

Основні рамки максимізації взаємної інформації представлені на рис. \ref{fig:deepinfo2}. 

\vspace{1em}

\begin{figure}[h]
  \includegraphics[width=\textwidth, height=5cm, natwidth=266, natheight=160]{Mal/deepinfo2.jpg}
  \caption{Deep InfoMax (DIM) з глобальною MI(X; Y) ціллю.}
  \label{fig:deepinfo2}
\end{figure}

Підхід слідує взаємній інформаційній нейронній оцінці (MINE), яка оцінює взаємну інформацію шляхом підготовки класифікатора для розрізнення зразків, що надходять із спільного розподілу $\mathbb{J}$ та добутку граничних значень, $\mathbb{M}$, випадкових величин $X$ та $Y$. MINE використовує нижню межу для MI на основі представлення Донскер-Варадхана (DV) KL-дивергенції,

\begin{equation}\label{eq:dv}
L(X; Y) := D_{KL}(\mathbb{J}||\mathbb{M}) \ge \hat{L}_{\omega}^{DV}(X;Y):=\mathbb{E}_{\mathbb{F}}|T_{\omega}(x,y)| - \log{\mathbb{E}_{\mathbb{M}}[e^{T_{\omega}(x,y)}}],
\end{equation}

\noindent де $T_{\omega}: X \times Y \rightarrow \mathbb{R}$ функція дискримінатора, змодельована нейронною мережею з параметрами $\omega$.

\vspace{1.5em}

На найвищому рівні оптимізується $\mathbb{E_{\psi}}$ одночасно оцінюючи та максимізуючи $L(X, E_{\psi}(X))$,

\begin{equation}\label{eq:e_psi_opt}
(\hat{\omega}, \hat{\psi})_{G} = \underset{\omega,\psi}{\arg\max}\hat{L}_{\omega}(X;E_{\psi}(X)),
\end{equation}

\vspace{1.5em}

\noindent де нижній індекс $G$ позначає \texttt{<<глобальний>>}. Однак існують деякі важливі відмінності, які відрізняють цей підхід від MINE. По-перше, оскільки кодер і оцінювач взаємної інформації оптимізують одну і ту ж мету і вимагають подібних обчислень, ми ділимо рівні між цими функціями, так що $E_{\psi} = f_{\psi} \circ C_{\psi}$ і $T_{\psi, \omega} = D \circ g \circ (C_{\psi}, E_{\psi})$, де $g$ - функція, яка поєднує вихідні дані кодера з нижчим шаром.

По-друге, враховуючи зацікавленність в максимальному збільшенні MI і не турбуючись його точним значенням, можна покластися на розбіжності, не пов’язані з KL, які можуть запропонувати вигідні компроміси. Наприклад, можна визначити оцінювач МІ Дженсена-Шеннона:

\begin{equation}\label{eq:mi}
\begin{aligned}
&\hat{L}_{\omega, \psi}^{(infoNCE)}(X;E_{\psi}(X)) := \\
&\quad \mathbb{E_{P}}\left[T_{\psi,\omega}(x, E_{\psi}(x)) - \mathbb{E_{\tilde{P}}}\left[\log{\sum_{x'}{e^{T_{\psi, \omega}(x',E_{\psi}(x))}}}\right]\right],
\end{aligned}
\end{equation}

\noindent де $x$ $-$ вхідний зразок, \newline
\hspace*{15pt} $x'$ $-$ вхід, відібраний з $\tilde{\mathbb{P}} = \mathbb{P}$, \newline
\hspace*{15pt} $sp(z) = \log{(1 + e^{z})}$ $-$ функція softplus. 

\vspace{1.5em}

Подібний оцінювач з'явився у контексті мінімізації сумарної кореляції, і це становить знайому двійкову перехресну ентропію. Це добре розуміється з точки зору оптимізації нейронних мереж, що на практиці такий підхід працює краще (наприклад, є стабільнішим), ніж ціль на основі DV. Інтуїтивно оцінювач, що базується на Йенсена-Шеннона, повинен поводитись подібно до оцінювача, що базується на DV в рівнянні. \ref{eq:mi}, оскільки обидва діють як класифікатори, цілі яких максимізують очікуване $log$-співвідношення з'єднання над добутком маржиналів.

Оцінка контрастності шуму (Noise-Contrastive Estimation) також може використовуватися з DIM шляхом максимізації:

\begin{equation}\label{eq:dim}
\hat{L}_{\omega,\psi}^{infoNCE}(X; E_{\psi}(X)) := \mathbb{E_{P}}\left[T_{\psi,\omega}(x, E_{\psi}) - \mathbb{E_{\tilde{P}}}\left[\log{\sum_{x'}{e^{T_{\psi, \omega}(x', E_{\psi}(x))}}}\right]\right].
\end{equation}

\vspace{1.5em}

Для DIM ключовою різницею між формулюваннями DV, JSD та infoNCE є те, чи з'являється очікування над $\mathbb{P / \tilde{P}}$ всередині $log$ або поза ним. Насправді ціль, що базується на JSD, відображає оригінальну формулювання NCE, яка формулює ненормалізовану оцінку щільності як двійкову класифікацію між розподілом даних та розподілом шуму. DIM встановлює розподіл шуму на добуток граничних значень на $X / Y$, а розподіл даних $-$ на справжнє з'єднання. Формулювання infoNCE слідує версії NCE на основі softmax, подібній до тих, що використовуються у спільноті моделювання мов і яка має міцні зв'язки з бінарною перехресною ентропією в контексті поріввняльного навчання. На практиці реалізації ці оцінювачи здаються досить схожими і можуть повторно використовувати більшість того самого коду. Дослідження JSD та infoNCE у експериментах дозволяє виявити, що використання infoNCE часто перевершує JSD у подальших завданнях, хоча цей ефект зменшується із більш складними даними. Однак для того, щоб infoNCE та DV вимагали великої кількості негативних зразків (зразки з $\mathbb{\tilde{P}}$), вони повинні бути конкурентоспроможними.

Завдання в рівнянні \ref{eq:e_psi_opt}  можна використовувати для максимізації MI між входом і виходом, але в кінцевому рахунку це може бути небажаним залежно від завдання. Наприклад, тривіальний шум на рівні пікселів марний для класифікації зображень, тому подання може не отримати вигоди від кодування цієї інформації (наприклад, при навчанні з нульовим знімком, навчанні передачі тощо). Для того, щоб отримати представлення, більш придатне для класифікації, можна замість цього максимізувати середній MI між представленням високого рівня та локальними плямами зображення. Оскільки одному і тому ж представництву рекомендується мати високий MI з усіма патчами, це надає перевагу кодуванню аспектів даних, які спільно використовуються між виправленнями.

Припустимо, що вектор ознак має обмежену ємність (кількість одиниць та діапазон) і припустимо, що кодер не підтримує нескінченні вихідні конфігурації. Для максимізації MI між усім входом і поданням, кодер може вибрати який тип інформації на вході передається через кодер, наприклад, шум, характерний для локальних латок або пікселів. Однак, якщо кодер передає інформацію, характерну лише для деяких частин вводу, це не збільшує МІ з будь-якими іншими патчами, які не містять згаданого шуму. Це спонукає кодер віддавати перевагу інформації, яка передається у вхідних даних.

DIM-фреймворк представлено на рис. \ref{fig:deepinfo3}. 

\vspace{1em}

\begin{figure}[h]
  \includegraphics[width=\textwidth, height=5cm, natwidth=291, natheight=188]{Mal/deepinfo3.jpg}
  \caption{Максимізація взаємної інформації між локальними та глобальними характеристиками.}
  \label{fig:deepinfo3}
\end{figure}

Спочатку потрібнозакодувати вхідні дані до карти об’єктів, $C_{\psi} = \left\{ C_{\psi}^{(i)} \right\}_{i=1}^{M \times M}$, що відображає корисну структуру в даних (наприклад, просторову місцевість), індексовану в цьому випадку $i$. Далі, можна узагальнити цю локальну карту об’єктів у загальну характеристику: $E_{\psi} = f_{\psi} \circ C_{\psi}(x)$. Потім необхідно визначити оцінювач MI на глобальних та локальних парах, максимізуючи середню оцінку MI:

\begin{equation}\label{eq:max_mi}
(\hat{\omega}, \hat{\psi})_{L} = \underset{\omega,\psi}{\arg\max}\frac{1}{M^{2}}\sum_{i=1}^{M^{2}}\hat{L}_{\omega, \psi}(C_{\psi}^{(i)}(X); E_{\psi}(X)).
\end{equation}

\vspace{1.5em}

Абсолютна величина інформації є лише однією бажаною властивістю уявлення. Залежно від програми хороші уявлення можуть бути компактними, незалежними, розплутаними або незалежно керованими. DIM накладає статистичні обмеження на вивчені уявлення, неявно навчаючи кодер так, щоб образ міри $\mathbb{U_{\psi, P}}$ збігався з попереднім $\mathbb{V}$. Це робиться шляхом навчання дискримінатора, $D_{\phi} : Y \rightarrow \mathbb{R}$, для оцінки розбіжності, $D(\mathbb{V} || \mathbb{U_{\psi, P}})$. Після чого тренуємо кодер для мінімізації оцінки:

\begin{equation}\label{eq:div}
\begin{aligned}
&(\hat{\omega}, \hat{\psi})_{P} \underset{\psi}{\arg\min}\,\underset{\phi}{\arg\max}\hat{D}_{\phi}(\mathbb{V}||\mathbb{U_{\psi,P}}) \\
&= \mathbb{E_{V}}[\log{D_{\phi}(y)}] + \mathbb{E_{P}}[\log{(1-D_{\phi}(E_{\psi}(x)))}].
\end{aligned}
\end{equation}

\vspace{1.5em}

Цей підхід подібний до того, що робиться в змагальних автокодерах, але без генератора. Він також схожий на шум як цілі, але тренує кодер, щоб він неявно відповідав шуму, а не використовував апріорні вибірки шуму як цілі.

Всі три цілі: глобальна та локальна максимізація МІ та попереднє узгодження можуть бути використані разом. І, таким чином, можна досягнути своєї повної мети для Deep InfoMax (DIM):

\begin{equation}\label{eq:deepinfomax}
\begin{aligned}
&\underset{\omega_{1},\omega_{2},\psi}{\arg\max}(\alpha \hat{L}_{\omega_{1},\psi}(X; E_{\psi}(X)) + \frac{\beta}{M^{2}}\sum_{i=1}^{M^{2}}{\hat{L}_{\omega_{2}, \psi}(X^{(i)}; E_{\psi}(X)))} \\
&+ \underset{\psi}{\arg\min}\,\underset{\phi}{\arg\max}\gamma\hat{D}_{\phi}(\mathbb{V}||\mathbb{U_{\psi,P}}),
\end{aligned}
\end{equation}

\noindent де $\omega_{1}$ та $\omega_{2}$ $-$ параметри дискримінатора відповідно до глобальних та локальних цілей, a $\alpha$, $\beta$ та $\gamma$ $-$ гіперпараметри.

\vspace{1.5em}

\subsection{Momentum Contrast}

Порівняльне навчання та його останні розробки можна розглядати як навчання кодеру для завдання пошуку словника, як описано далі.

Розглянемо закодований запит $q$ та набір закодованих зразків $\{k_{0}, k_{1}, k_{2}, \dots\}$, які є ключами словника. Припустимо, що у словнику є один ключ (позначений як $k_{+}$), який відповідає $q$. Контрастивна втрата $-$ це функція, значення якої є низьким, коли $q$ подібне до його позитивного ключа $k_{+}$ та відрізняється від усіх інших ключів (вважається негативним ключем для $q$). Можна перевизначити функцію InfoNCE:

\begin{equation}\label{eq:infonce_simple}
L_{q} = -\log{\frac{exp(q \cdot k_{+}/\tau)}{\sum_{i=0}^{K}{exp(q \cdot k_{i}/\tau)}}},
\end{equation}

\vspace{1.5em}

\noindent де $\tau$ гіперпараметр. Сума складається з однієї позитивної та $K$ негативних екземплярів. Інтуїтивно, ця втрата є часовою втратою класифікатора на основі $(K + 1)$ softmax, який намагається класифікувати $q$ як $k_{+}$. Функції контрастивних збитків можуть також базуватися на інших формах, таких як маржові збитки та варіанти збитків NSE.

Контрастивна втрата служить неконтрольованою цільовою функцією для навчання мереж кодера, що представляють запити та ключі. Загалом, подання запиту має значення $q = f_{q} (x^{q})$, де $f_{q}$ - це мережа кодера, а $x^{q}$ - зразок запиту (так само, $k = f_{k} (x^{k})$). Їх екземпляри залежать від конкретного попередньогозавдання. Вхідними даними $x^{q}$ та $x^{k}$ можуть бути зображення, патчі або контекст, що складається з набору патчів. Мережі $f_{q}$ і $f_{k}$ можуть бути ідентичними, частково спільними або різними.

З вищенаведеної точки зору, порівняльне навчання $-$ це спосіб побудови дискретного словника на безперервних вхідних даних, таких як зображення. Словник є динамічним у тому сенсі, що ключі вибираються випадково, а кодер ключів розвивається під час навчання. Гіпотеза полягає в тому, що хорошим можливостям можна навчитися за допомогою великого словника, який охоплює багатий набір негативних зразків, тоді як кодер ключів словника зберігається якомога послідовнішим, незважаючи на його розвиток. Виходячи з цього можна описати алгорим Momentum Contrast.

В основі цього підходу лежить підтримка словника як черги зразків даних. Це дозволяє нам повторно використовувати закодовані ключі з безпосередніх попередніх міні-пакетів. Введення черги відокремлює розмір словника від розміру міні-партії. Розмір словника може бути набагато більшим, ніж типовий розмір міні-партії, і може бути гнучко та незалежно встановлений як гіперпараметр.

Зразки у словнику поступово замінюються. Поточна міні-партія потрапляє до словника, а найстаріша міні-партія в черзі вилучається. Словник завжди представляє вибіркову підмножину всіх даних, тоді як додаткові обчислення ведення цього словника є керованими. Більше того, видалення найстарішої міні-партії може бути корисним, оскільки її закодовані ключі є найбільш застарілими і, отже, найменш відповідають найновішим.

Використання черги може зробити словник великим, але це також робить важким оновлення кодера ключа шляхом зворотного розповсюдження (градієнт повинен поширюватися на всі зразки в черзі). Наївним рішенням є копіювання кодера ключа $f_{k}$ з кодера запиту $f_{q}$, ігноруючи цей градієнт. Але це рішення дає погані результати в експериментах. Припустимо, що такий збій спричинений швидко мінливим кодером, який зменшує узгодженість ключових подань. Пропонується оновити імпульс для вирішення цієї проблеми.

Формально, позначаючи параметри $f_{k}$ як $\theta_{k}$, а параметри $f_{q}$ як $\theta_{q}$, ми оновлюємо $\theta_{k}$ за допомогою:

\begin{equation}\label{eq:theta_opt}
\theta_{k} \leftarrow m\theta_{k} + (1 - m)\theta_{q}.
\end{equation}

\vspace{1.5em}

Тут $m \in [0, 1)$ $-$ коефіцієнт імпульсу. Тільки параметри $\theta_{q}$ оновлюються шляхом зворотного розповсюдження. Оновлення імпульсу в рівнянні (6) змушує $\theta_{k}$ еволюціонувати більш плавно, ніж $\theta_{q}$. Як результат, хоча ключі в черзі кодуються різними кодерами (в різних міні-партіях), різниця між цими кодерами може бути невеликою. В експериментах порівняно великий імпульс (наприклад, m = 0,999, за замовчуванням) працює набагато краще, ніж менший показник (наприклад, m = 0,9), що дозволяє припустити, що повільно розвивається кодер ключа є стрижнем для використання черги.

MoCo $-$ загальний механізм використання контрастивних втрат. Ми порівнюємо це з двома існуючими загальними механізмами на рис \ref{fig:momentum1}. Вони мають різні властивості щодо розміру та послідовності словника.

\vspace{1em}

\begin{figure}[h]
  \includegraphics[width=\textwidth, height=5cm, natwidth=474, natheight=193]{Mal/momentum1.jpg}
  \caption{Концептуальне порівняння трьох контрастивних механізмів втрат.}
  \label{fig:momentum1}
\end{figure}

Оновлення end-to-end шляхом зворотного розповсюдження є природним механізмом. Він використовує зразки в поточній міні-партії як словник, тому ключі послідовно кодуються (тим самим набором параметрів кодера). Але розмір словника поєднується з міні-пакетним розміром, обмеженим обсягом пам'яті графічного процесора. Це також кидається виклик великій міні-пакетній оптимізації. Деякі останні методи засновані на завданнях підтексту, керованих місцевими позиціями, де розмір словника може бути збільшений на кілька позицій. Але для цих завдань-претекстів можуть знадобитися спеціальні мережеві конструкції, такі як патчіфікація вводу або налаштування сприйнятливого розміру поля, що може ускладнити перенесення цих мереж на подальші завдання

Іншим механізмом є підхід банк пам'яті (memory bank). Банк пам'яті складається з подань усіх зразків у наборі даних. Словник для кожної міні-партії вибірково вибирається з банку пам'яті без зворотного розповсюдження, тому він може підтримувати великий розмір словника. Однак представлення вибірки в банку пам'яті було оновлено, коли її востаннє бачили, тому вибіркові ключі, по суті, стосуються кодерів на декількох різних етапах протягом усієї минулої епохи і, отже, менш послідовні. Оновлення імпульсу приймається на банку пам'яті. Її імпульс оновлення подається на поданнях того самого зразка, а не на кодері. Це оновлення імпульсу не має значення для нашого методу, оскільки MoCo не відстежує кожен зразок. Більше того, наш метод є більш ефективним в пам’яті, і його можна навчити на мільярдних даних, що може бути важко для банку пам'яті.

Порівняльне навчання може працювати з різними попередніми задачами.

Можна розглянути запит і ключ як позитивну пару, якщо вони походять з одного зображення, а в іншому $-$ як негативну пару вибірки. Візьмемо два випадкові «види» одного і того ж зображення під час випадкового збільшення даних, щоб сформувати позитивну пару. Запити та ключі кодуються відповідними кодерами $f_{q}$ та $f_{k}$. Кодером може бути будь-яка згорткова нейронна мережа.

Алгоритм Momentum Contrast для попередньої задачі можна реалізовувати по-різному. Наприклад: адаптувати нейронну мережу ResNet як кодер, чий останній повністю підключений шар (після загального середнього об'єднання) має вихід з фіксованим розміром (128-D). Цей вихідний вектор нормується за його L2-нормою. Це представлення запиту або ключа. Після чого потрібно обрати значення для аргументу $\tau$ з \ref{eq:infonce_simple}. Далі наведено параметр збільшення даних: обрізання $224 \times 224$ пікселів береться із зображення випадкового розміру, а потім піддається випадковому коливанню кольорів, випадковому горизонтальному перевертанню та випадковому перетворенню шкал сірого.

Обидва кодери $f_{q}$ і $f_{k}$ використовують пакетну нормалізацію (Batch Normalization, BN), як у стандартному ResNet. Можна показати, що використання BN заважає моделі засвоїти хороші уявлення.

Цю проблему можливо вирішити за допомогою тасування BN. Можна тренувати модель з декількома графічними процесорами та виконуввати BN на зразках незалежно для кожного графічного процесора (як це робиться у звичайній практиці). Для кодера ключів $f_{k}$ перемішується порядок зразків у поточній міні-партії перед розподілом між графічними процесорами (і перемішується назад після кодування). Порядок вибірки міні-партії для кодера запиту $f_{q}$ не змінюється. Це гарантує, що статистична інформація про партії, яка використовується для обчислення запиту, та її позитивний ключ надходять із двох різних підмножин. Це ефективно вирішує проблему дозволяє тренуванню принести користь від BN.

Використання перемішаного BN може стати у нагоді також у end-to-end оновленні. Це не має значення для банку пам'яті, який не страждає від цієї проблеми, оскільки позитивні ключі були від різних міні-пакетів у минулому.

\subsection{Висновки за розділом}

Порівняльне навчання $-$ це техніка машинного навчання, яка використовується для вивчення загальних особливостей набору даних без міток, навчаючи моделі, які точки даних подібні чи різні.

В результаті огляду літературних джерел для аналізу були обрані алгоритми Deep InfoMax (DIM) та Momentum Contrast.

%\newpage
%%\section{Практична реалізація}
\section{ПРАКТИЧНА РЕАЛІЗАЦІЯ}

Даний розділ присвячується аналізу часового ряду тестувальної системи DOTS за допомогою мови програмування Python. 

Python $-$ це легка в освоєнні, потужна мова програмування. Він має ефективні структури даних високого рівня і простий, але ефективний підхід до об'єктно-орієнтованого програмування [7].

Python став загальноприйнятою мовою програмування для багатьох сфер застосування науки про дані. Дана мова програмування поєднує в собі міць мов програмування з простотою використання предметно орієнтованих скриптових мов типу MATLAB або R. 

В Python є бібліотеки для завантаження даних, візуалізації, статистичних обчислень, обробки природної мови, обробки зображень і багато чого іншого [8].

Також використовувався фреймфорк Anaconda. Це безкоштовний, включаючи комерційне використання, і готовий до використання в середовищі підприємства дистрибутив Python, який об'єднує всі ключові бібліотеки Python, необхідні для роботи в області науки про дані, математики та розробки, в одному зручному для користувача крос-платформенном дистрибутиві [9].

Для тренування нейронних мереж було обрано бібліотеку PyTorch. PyTorch - це бібліотека для програм Python, яка сприяє побудові проектів глибокого навчання [10].

Також використовувалась бібліотеки NumPy та matplotlib. NumPy є основним пакетом для наукових обчислень з Python [11]. Matplotlib $-$ це всебічна бібліотека для створення статичних, анімованих та інтерактивних візуалізацій у Python [12]. 

\subsection{Дані для аналізу}

Для демонстрації та аналізу роботи алгоритмів були використані дані з датасету CIFAR-10.

Набір даних CIFAR-10 складається з 60000 кольорових зображень розміром 32x32 у 10 класах, по 6000 зображень на клас. Існує 50000 навчальних зображень та 10000 тестових зображень [13].

Набір даних розділений на п’ять навчальних партій та одну тестову партію, кожна з 10000 зображень. Тестова партія містить рівно 1000 довільно обраних зображень з кожного класу. Навчальні партії містять решту зображень у довільному порядку, але деякі навчальні партії можуть містити більше зображень одного класу, ніж інші [13]. Навчальні партії містять рівно 5000 зображень від кожного класу.

Приклад зображень з датасету наведено у рис. \ref{fig:cifar10}.

\vspace{1em}

\begin{figure}[h]
  \includegraphics[width=\textwidth, height=5cm, natwidth=1227, natheight=187]{cifar10.jpg}
  \caption{Приклад зображень з датасету CIFAR-10}
  \label{fig:cifar10}
\end{figure}

\subsection{Аналіз роботи алгоритму Deep Infomax}

Робота методу Deep InfoMax буде розглядатись в залежності від значеть гіперпараметрів $\alpha, \, \beta \, \text{та} \, \gamma$, а також коефіціенту швидкості навчання. В кожному експерименті кількість епох навчання нейронної мережі складатиме 300.

Якщо взяти $\alpha = 0,1 \, \beta = 0,1 \, \text{та} \, \gamma = 0,1$, то помилка на тренувальних даних буде 12,73 \% (див. рис. \ref{fig:deepinfoerror1}) та модель тренувалась приблизно 14 годин.

\vspace{1em}

\begin{figure}[h]
  \includegraphics[width=\textwidth, height=7cm, natwidth=854, natheight=476]{deepinfoerror1.jpg}
  \caption{Залежність помилки від епохи навчання при $\alpha = 0,1; \, \beta = 0,1; \, \gamma = 0,1$}
  \label{fig:deepinfoerror1}
\end{figure}

На рис. \ref{fig:deepinfodemo1} можна побачити, що модель доволі погано узагальнює результати.

\vspace{1em}

\begin{figure}[h]
  \includegraphics[width=\textwidth, height=7cm, natwidth=854, natheight=476]{deepinfodemo1.jpg}
  \caption{Результати тестування при $\alpha = 0,1; \, \beta = 0,1; \, \gamma = 0,1$}
  \label{fig:deepinfodemo1}
\end{figure}

Якщо взяти $\alpha = 0,5 \, \beta = 0,5 \text{та} \, \gamma = 0,5$, то помилка на тренувальних даних буде 8,05 \% (див. рис. \ref{fig:deepinfoerror2}) та модель тренувалась приблизно 11 годин.

\vspace{1em}

\begin{figure}[h]
  \includegraphics[width=\textwidth, height=7cm, natwidth=854, natheight=476]{deepinfoerror2.jpg}
  \caption{Залежність помилки від епохи навчання при $\alpha = 0,5; \, \beta = 0,5; \, \gamma = 0,5$}
  \label{fig:deepinfoerror1}
\end{figure}

При $\alpha = 0,5 \, \beta = 0,9 \text{та} \, \gamma = 0,1$ помилка на тренувальних даних буде 1,28 \% (див. рис. \ref{fig:deepinfoerror2}) та модель тренувалась приблизно 6 годин.

\vspace{1em}

\begin{figure}[h]
  \includegraphics[width=\textwidth, height=7cm, natwidth=854, natheight=476]{deepinfoerror3.jpg}
  \caption{Залежність помилки від епохи навчання при $\alpha = 0,5; \, \beta = 0,9; \, \gamma = 0,1$}
  \label{fig:deepinfoerror1}
\end{figure}

Якщо взяти $\alpha = 0,5 \, \beta = 1 \text{та} \, \gamma = 0,1$, то помилка на тренувальних даних буде 1,43 \% (див. рис. \ref{fig:deepinfoerror2}) та модель тренувалась приблизно 6 годин.

\vspace{1em}

\begin{figure}[h]
  \includegraphics[width=\textwidth, height=7cm, natwidth=854, natheight=476]{deepinfoerror4.jpg}
  \caption{Залежність помилки від епохи навчання при $\alpha = 0,5; \, \beta = 1; \, \gamma = 0,1$}
  \label{fig:deepinfoerror4}
\end{figure}

Отже, найкращі результати модель отримує при $\alpha = 0,5 \, \beta = 0,9 \text{та} \, \gamma = 0,1$. Результати тестування з даними гіперпараметрами наведено на рис. \ref{fig:deepinfodemo2}.

\vspace{1em}

\begin{figure}[h]
  \includegraphics[width=\textwidth, height=7cm, natwidth=854, natheight=476]{deepinfodemo2.jpg}
  \caption{Результати тестування при $\alpha = 0,5; \, \beta = 1; \, \gamma = 0,1$}
  \label{fig:deepinfodemo2}
\end{figure}

Усі вищенаведені результати були отримані з урахуванням коефіціенту швидкості навчання 0,001. Розглянемо результати роботи алгоритму з оптимальними гіперпараметрами з урахуванням інших можливих значень цього коефіціента.

Якщо взяти коефіціент швидості навчання 0,003 то помилка на тренувальних даних буде 1,35 \% (див. рис. \ref{fig:deepinfoerror2}) та модель тренувалась приблизно 6 годин.

\vspace{1em}

\begin{figure}[h]
  \includegraphics[width=\textwidth, height=7cm, natwidth=854, natheight=476]{deepinfoerror5.jpg}
  \caption{Залежність помилки від епохи навчання при коефіціенті швидкості навчання 0,003}
  \label{fig:deepinfoerror5}
\end{figure}



\subsection{Аналіз роботи алгоритму Momentum Contrast}




\subsection{Висновки за розділом}

В даному розділі була продемонстрована робота алгоритмів Deep InfoMax та Momentum Contrast. БУли реалізовані програми мовою програмування Python за допомогою спеціальних бібліотек для аналізу даних та програмного фреймворку Anaconda. 

Як видно з результатів, модель Deep InfoMax дає якісніший прогноз, в той час як алгоритм Momentum Contrast потребує менше часових та обчислювальних ресурсів.

%\newpage
%\section{Економічне обґрунтування}
\section{ЕКОНОМІЧНЕ ОБҐРУНТУВАННЯ}

Темою дипломної роботи є «Розробка методів представлення візуальної інформації за допомогою методів самонавчання та contrastive learning». В процесі роботи була розглянута необхідна теорія, а також був створений програмний продукт. Важливою частиною дипломної роботи є економічне обгрунтування.

\subsection{Розрахунок кошторису витрат на проведення й впровадження результатів науково-дослідної роботи}
Виконання наукових дослiджень, а також впровадження результатiв НДР вимагає певних витрат, якi необхiдно розглядати як додатковi капiталовкладення. Витрати на проведення й впровадження результатiв НДР вiдносяться до виробничих витрат. 

Як правило, всi витрати документально оформляються у виглядi кошторису. Основними статтями кошторису витрат є заробiтна плата, нарахування на заробiтну плату, вартiсть електроенергiї (технологiчна й освiтлювальної), вартiсть оренди примiщення, амортизацiйнi вiдрахування на обчислювальну технiку, вартiсть впровадження й освоєння результатiв НДР i плановi накопичення.

\subsubsection{Розрахунок фонду заробітної плати виконавців}
Розрахунок фонду заробiтної плати виконавцiв проводиться виходячи зi штатного розкладу й зайнятостi виконавцiв у данiй НДР.
Виконавцями даної НДР є керівник дипломної роботи, консультанти частини економічного обґрунтування, частини охорони праці дипломної роботи й частини цивільного захисту, а також інженер-математик. Штатний розклад приведено у табл. \ref{tab:staff}.

\begin{table}
	\captionstyle{ \raggedright}
	\caption{Штатний розклад}\label{tab:staff}
	\begin{tabular}{| p{0.20\textwidth} | p{0.12\textwidth} | p{0.12\textwidth} | p{0.14\textwidth} | p{0.14\textwidth} | p{0.14\textwidth} |}
		\hline
		Посада & Кількість виконавців & Час зайнятості, міс & Коефіціент трудової участі & Оклад на місяць, грн & Заробітна плата, $\text{Зп}_{\text{оклад}}$, грн \\
		\hlinewd{2pt}
		1 Керівник роботи, старший викладач & 1 & 4 & 0,075 & 7000 & 2100\\
		\hline
		2 Консультант частини економічного обгрунтування, професор & 1 & 4 & 0,005 & 10000 & 200\\
		\hline
		3 Консультант частини охорони праці, доцент & 1 & 4 & 0,005 & 8000 & 160\\
		\hline
		4 Консультант частини цивільного захисту, старший викладач & 1 & 4 & 0,005 & 7000 & 140\\
		\hline
		5 Виконавець, інженер-математик & 1 & 4 & 1 & 6000 & 24000\\
		\hline
	\end{tabular}
\end{table}

\newpage

Заробітна плата виконавців НДР складається з основної заробітної плати й різних доплат до неї:

\begin{equation}\label{eq:zp}
\text{Зп} = \text{Зп}_{\text{осн}} + \text{Зп}_{\text{д}},
\end{equation}

\noindent де $\text{Зп}_{\text{осн}}$ $-$ основна заробітна плата; \newline 
\hspace*{15pt} $\text{Зп}_{\text{д}}$ $-$ доплати до заробітної плати.


\begin{equation}
\text{Зп}_{\text{осн}} = \text{Зп}_{\text{оклад}} + \text{Зп}_{\text{прем}},
\end{equation}

\noindent де $\text{Зп}_{\text{оклад}}$ $-$ розмір заробітної плати за штатним розкладом; \newline
\hspace*{15pt} $\text{Зп}_{\text{прем}}$ $-$ розмір премій. 

\begin{equation}
\text{Зп}_{\text{прем}} = \text{К}_{\text{прем}} \cdot \text{Зп}_{\text{оклад}},
\end{equation}

\noindent де $\text{К}_{\text{прем}}$ $-$ коефіціент преміювання, $\text{К}_{\text{прем}} = 0,1$;

\begin{equation}
\text{Зп}_{\text{д}} = \text{К}_{\text{д}} \cdot \text{Зп}_{\text{осн}},
\end{equation}

\noindent де $\text{К}_{\text{д}}$ $-$ коефіцієнт доплат заробітної плати, $\text{К}_{\text{д}} = 0,1$. 

\vspace{1.5em}

Розрахуємо заробітню плату виконавців НДР.

Керівник дипломної роботи:

\[
\text{Зп} = 2100 + 2100 \cdot 0,1 + 0,1 \cdot (2100 + 2100 \cdot 0,1) = 2541,00 \, \text{грн}.
\]

\vspace{1.5em}

Консультант частини економічного обгрунтування:

\[
\text{Зп} = 200 + 200 \cdot 0,1 + 0,1 \cdot (200 + 200 \cdot 0,1) = 242,00 \, \text{грн}.
\]

\vspace{1.5em}

Консультант частини охорони праці:

\[
\text{Зп} = 160 + 160 \cdot 0,1 + 0,1 \cdot (160 + 160 \cdot 0,1) = 193,60 \, \text{грн}.
\]

\vspace{1.5em}

Консультант частини цивільного захисту:

\[
\text{Зп} = 140 + 140 \cdot 0,1 + 0,1 \cdot (140 + 140 \cdot 0,1) = 169,40 \, \text{грн}.
\]

\vspace{1.5em}

Консультант дипломної роботи:

\[
\text{Зп} = 24000 + 24000 \cdot 0,1 + 0,1 \cdot (24000 + 24000 \cdot 0,1) = 29040,00 \, \text{грн}.
\]

\vspace{1.5em}

Фонд заробiтної плати виконавцiв складе:

\[
\text{Зп}_{\text{заг}} = 2541,00 + 242,00 + 193,60 + 169,40 + 6400,00 = 32186,00 \, \text{грн}.
\]

\vspace{1.5em}

Всього витрати на заробiтну плату склали 32186,00 грн.

\subsubsection{Відрахування на соціальне страхування}

Відрахування на соціальне страхування й інші відрахування розраховуються на підставі отриманого значення фонду заробітної плати:

\begin{equation}\label{eq:soc}
\text{Від} = \text{К}_{\text{від}} \cdot \text{Зп},
\end{equation}


\noindent де $\text{К}_{\text{від}}$ $-$ коефіцієнт нарахувань на фонд заробітної плати приймається в розмірі $0,362$.

\[
\text{Від} = 0,22 \cdot 32186,00 = 7080,92 \, \text{грн}.
\]

\vspace{1.5em}

\subsubsection{Розрахунок технологічної електроенергії}

Розрахунок технологічної електроенергії проводиться виходячи із завантаження устаткування, що використовується під час проведення НДР (ЕОМ, принтер, сканер і ін.), по формулі (\ref{eq:texenergy}):

\begin{equation}\label{eq:texenergy}
\text{Е}_{\text{тех}} = P \sum_{i=1}^{N}\text{П}_{i}T_{i}
\end{equation}

\noindent де $P$ $-$ тариф на електроенергію, $P = 1,7808 \, \text{грн}/\text{кВт}$; \newline
\hspace*{15pt} $\text{П}_{i}$ $-$ споживана потужність $i$-ої одиниці встаткування, для комп'ютера $\text{П}_{i} = 0,3 \, \text{кВт}/\text{год}$;\newline
\hspace*{15pt} $T_{i}$ $-$ час роботи $i$-ої одиничі встаткування, $T_{1} = 288 \, \text{год}$.

\[
\text{Е}_{\text{тех}} = 1,7808 \cdot 0,3 \cdot 288 = 153,86 \, \text{грн}.
\]

\vspace{1.5em}

\subsubsection{Розрахунок електроенергії, що витрачається на освітлення}

Розрахунок електроенергії, що витрачається на освітлення, виконується виходячи з норм охорони праці по освітленню робочих місць та розраховується наступним чином:

\begin{equation}\label{eq:osv}
\text{Е}_{\text{осв}} = P \cdot N_{\text{л}} \cdot \text{П}_{\text{л}} \cdot T,
\end{equation}

\noindent де $P$ $-$ тариф на електроенергію, $P = 1,7808 \, \text{грн}/\text{кВт}$; \newline
\hspace*{15pt}$N_{\text{л}}$ $-$ кількість ламп, $N_{\text{л}} = 1$;\newline
\hspace*{15pt}$\text{П}_{\text{л}}$ $-$ споживана потужність однієї лампи, $\text{П}_{\text{л}} = 0,1 \, \text{кВт}/\text{год}$;\newline
\hspace*{15pt}$T$ $-$ час роботи ламп для освітлення, $T = 122 \, \text{год}$.

\[
\text{Е}_{\text{осв}} = 1,7808 \cdot 1 \cdot 0,1 \cdot 122 = 22,73 \, \text{грн}.
\]

\vspace{1.5em}

\subsubsection{Амортизаційні відрахування на устаткування}

Амортизацiйнi витрати розраховуються виходячи з формули (\ref{eq:amort}):

\begin{equation}\label{eq:amort}
A = \frac{a_{\text{ЕОМ}}}{12}\sum_{i=1}^{N}\text{ЗВ}_{i} \cdot T_{i},
\end{equation}

\noindent де $a_{\text{ЕОМ}}$ $-$ річна норма амортизації, прийнята в розмірі $ 25 \, \%$ залишкової вартості устаткування;\newline
\hspace*{23pt}$\text{ЗВ}_{i}$ $-$ залишкова вартість $i$-ої одиниці устаткування, $\text{ЗВ}_{1} = 4634,55 \, \text{грн}$;\newline
\hspace*{23pt}$T_{i}$ $-$ час використання $i$-ої одиниці устаткування $T_{1} = 3 \, \text{міс}$.

\[
A = \frac{0,25}{12} \cdot 4634,55 \cdot 3 = 289,66 \, \text{грн}.
\]

\vspace{1.5em}

\subsubsection{Вартість оренди приміщення}

Витрати на оренду примiщення розраховуються виходячи з формули (\ref{eq:orenda}):

\begin{equation}\label{eq:orenda}
\text{Д} = \text{К}_{\text{а}} \cdot S \cdot P \cdot T_{OP},
\end{equation}

\noindent де $\text{К}_{\text{а}}$ $-$ коефіцієнт, що враховує податок на майно, $\text{К}_{\text{а}} = 1,2$;\newline
\hspace*{19pt}$S$ $-$ площа приміщення, де проводилася НДР, $S = 6 \, \text{м}^{2}$;\newline
\hspace*{19pt}$P$ $-$ вартість оренди одного квадратного метра приміщення, $P = 200 \, \text{грн}/\text{міс}$;\newline
\hspace*{19pt}$T_{OP}$ $-$ строк оренди, $T_{OP} = 4 \, \text{міс}$.

\[
\text{Д} = 1,2 \cdot 6 \cdot 200 \cdot 4 = 5760,00 \, \text{грн}.
\]

\vspace{1.5em}

\subsubsection{Інші витрати}

Інші витрати (опалення, робота кондиціонера й ін.), згідно з формулою (\ref{eq:inshi}), приймаються в розмірі $7\%$ від вартості оренди приміщення. 

\begin{equation}\label{eq:inshi}
\text{З}_{\text{ін}} = \text{Д} \cdot \text{К}_{\text{ін}} = 5760,00 \cdot 0,07 = 403,20 \, \text{грн}. 
\end{equation}

\vspace{1.5em}

\subsubsection{Вартість впровадження й освоєння результатів НДР}

Результатом НДР є система яка дозволяє більш якісно використовувати алгоритми contrastive learning. Для використання цієї системи необхідно впровадити та освоїти програми у аналітичних відділах.

При впровадженнi та освоєннi результатiв НДР необхiдно залучити хоча б одного асистента кафедри в науково дослiдному iнститутi з вiдповiдної спецiальностi. Заробiтна плата становить 5000 грн на мiсяць в середньому. На впровадження необхiдно не менше мiсяця. У пiдсумку вартiсть впровадження та освоєння результатiв НДР складе $5000$ грн.

\subsubsection{Витрати на проведення НДР}

Витрати на проведення НДР, згідно з формулою (\ref{eq:vitraty}), являють собою суму витрат по окремих статтях:

\begin{equation}\label{eq:vitraty}
\text{З} = \text{Зп} + \text{Від} + \text{Е}_{\text{тех}} + \text{Е}_{\text{осв}} + \text{А} + \text{Д} + \text{З}_{\text{ін}} + \text{В}_{\text{вп}},
\end{equation}

\noindent де $\text{З}_{\text{ін}}$ $-$ інші витрати; \newline
\hspace*{15pt} $ \text{В}_{\text{вп}}$ $-$ вартість впровадження й освоєння результатів НДР.

Таким чином, сукупнi витрати складають:

\begin{eqnarray*}
\text{З} = 32186,00 + 7080,92 + 153,86 + 22,73 + 289,66 + \\ + 5760,00 + 403,20 + 5000  = 50896,37 \, \text{грн}.
\end{eqnarray*}

\vspace{1.5em}

\subsubsection{Планові накопичення}

Планові накопичення обираються в розмірі $30\%$ від витрат на проведення НДР та наведені у формулі (\ref{eq:plan}).

\begin{equation}\label{eq:plan}
\text{ПН} = 0,3 \cdot 50896,37 = 15268,91 \, \text{грн}.
\end{equation}

\vspace{1.5em}

\subsubsection{Кошторис витрат на проведення НДР}

Кошторис витрат на проведення НДР є сумою витрат на проведення НДР і планових накопичень. Результати розрахунку кошторису витрат представлені у табл. \ref{tab:sumNDR}.

\newpage

\begin{table}[hbt]
	\captionstyle{ \raggedright}
	\caption{Кошторис витрат на проведення НДР}\label{tab:sumNDR}
	%\centering
	\begin{tabular}{|p{0.65\textwidth}|p{0.15\textwidth}|}
		\hline
		Стаття витрат & Сума, грн \\
		\hlinewd{2pt}
		1 Заробiтна плата & 32186  \\
		\hline
		2 Вiдрахування на соцiальне страхування & 7080,92 \\
		\hline
		3 Технологiчна електроенергiя & 153,86 \\
		\hline
		4 Електроенергiя на освiтлення & 22,73 \\
		\hline
		5 Амортизацiйнi вiдрахування на устаткування & 289,66 \\
		\hline
		6 Вартiсть оренди примiщення & 5760 \\
		\hline
		7 Iншi витрати & 403,20 \\
		\hline
		8 Вартiсть впровадження та освоєння НДР & 5000 \\
		\hline
		9 Разом витрат & 50896,37 \\
		\hline
		10 Плановi накопичення & 15268,91 \\
		\hline
		11 Усього кошторис витрат на НДР & 66165,28  \\
		\hline
	\end{tabular}
\end{table}

\subsection{Класифікація й кодування запропонованої інновації}

Основними критеріями класифікації інновацій повинні бути:

\begin{enumerate}
	\item комплексність набору класифікаційних ознак, що враховують, для аналізу й кодування;
	\item можливість кількісного (якісного) визначення критерію;
	\item наукова новизна й практична цінність пропонованої ознаки класифікації
\end{enumerate}

З урахуванням наявного досвіду й наведених критеріїв пропонується наступна класифікація нововведень і інновацій.

За рівнем новизни інновації:

\begin{enumerate}
	\item Радикальні (відкриття, винаходи).
	\item Ординарні (ноу-хау, раціоналізаторські пропозиції).
\end{enumerate}

За стадією життєвого циклу товару, на якій впроваджується інновація:

\begin{enumerate}
	\item Інновації, впроваджувані на стадії стратегічного маркетингу.
	\item Інновації, впроваджувані на стадії НДОКР.
	\item Інновації, впроваджувані на стадії ОТПВ.
	\item Інновації, впроваджувані на стадії виробництва.
	\item Інновації, впроваджувані на стадії сервісного обслуговування.
\end{enumerate}

За масштабом новизни інновації:

\begin{enumerate}
	\item Інновації у світовому масштабі.
	\item Інновації в країні.
	\item Інновації в галузі.
	\item Інновації для підприємства.
\end{enumerate}

За галузю народного господарства, де впроваджується інновація:

\begin{enumerate}
	\item Наука.
	\item Освіта.
	\item Соціальна сфера.
	\item Матеріальне виробництво.
	\item Роботи й послуги.
\end{enumerate}

За сферою застосування інформації:

\begin{enumerate}
	\item Інновації для внутрішнього застосування.
	\item Інновації для нагромадження в організації.
	\item Нововведення для продажу.
\end{enumerate}

За частотою застосування інновації:

\begin{enumerate}
	\item Разові.
	\item Повторювані.
\end{enumerate}

За формою нововведення:

\begin{enumerate}
	\item Відкриття, винаходи, патенти.
	\item Раціоналізаторські пропозиції.
	\item Ноу-хау.
	\item Товарні знаки, торговельні марки, емблеми.
	\item Нові документи, що описують технологічні, виробничі, управлінські процеси, конструкції, структури, методи.
\end{enumerate}

За видом ефекту, отриманого в результаті впровадження інновації:

\begin{enumerate}
	\item Науково-технічний.
	\item Соціальний.
	\item Екологічний.
	\item Економічний (комерційний).
	\item Інтегральний.
\end{enumerate}

За підсистемою системи керування, у якій впроваджується інновація:

\begin{enumerate}
	\item Підсистема наукового супроводу.
	\item Цільова підсистема.
	\item Підсистема, що забезпечує.
	\item Керована підсистема.
	\item Керуюча підсистема.
\end{enumerate}

Наведена класифікація охоплює всі аспекти інноваційної діяльності. Для спрощення управління інноваційною діяльністю на основі цієї класифікації інновації можна кодувати. Кодування може бути спрощене (з одним знаком для ознаки) і детальне (із двома й більше знаками для ознаки). У цьому випадку використовується спрощене кодування, при якому код інновації буде мати 9 цифр. Номер цифри відповідає ознаці класифікації в запропонованому вище порядку, а значення цифри відповідає виду інновації.

Відповідно до наведеної класифікації, код інновації даної НДР 2.4.3.1.1.2.5.1.1.

\subsection{Розрахунок економічного ефекту від впровадження результатів НДР}

Щоб показати доцільність застосування досліджень, проведених у дипломній роботі, необхідно виконати розрахунок економічного ефекту.

Економічний ефект розраховується виходячи із суми, отриманих від впроваджених результатів НДР доходів: 

\begin{equation}\label{eq:economy11}
	\text{Д}_{\text{Е}} = \sum_{i=1}^{n}\text{Д}_{i},
\end{equation}

\noindent де $\text{Д}_{i}$ $-$ величина додаткових доходів або економії коштів, отриманих у результаті впровадження НДР, по $i$-му фактору. 

\vspace{1.5em}

Факторами виступають:

\begin{enumerate}
	\item Економія робочого часу особи, яка приймає рішення. Програмне забезпечення, що допомагає у вирішенні складних задач експертної оцінки об’єктів, зменшує час на прийняття рішення та кількість працівників. Середня заробітна плата фахівця у цих галузях складає 7000 грн на місяць. Зменшення загальної заробітної плати у два рази за рахунок скорочення кількості працівників призведе до економії коштів, що на рік складатиме: $\text{Д}_{1} = \frac{7000,00 \cdot 12}{2} = 42000,00 \, \text{грн}$.
	\item 2	Зниження виробничої площі за рахунок зменшення одиниць устаткування (персональних комп’ютерів). Оскільки програмне забезпечення дозволяє скоротити кількість працівників, відповідно, зменшиться кількість обладнання. Чисельність комп’ютерів в одному офісі – 10 штук. В середньому вартість одного стаціонарного комп’ютера складає 12 500 грн. Оскільки кількість працівників було зменшено вдвічі, то й у два рази зменшиться кількість одиниць устаткування, отже даний проект дозволяє зменшити витрати на 50 \%: $\text{Д}_{2} = 10 \cdot 12500,00 \cdot 0,5 = 62500,00 \, \text{грн}$.
\end{enumerate}

Для розрахунку коштів, отриманих за рахунок цих факторів, необхідно проводити детальний аналіз з залученням фахівців різних областей, що виходить за рамки розрахунку економічного ефекту від даної НДР.

Отже економічний ефект даної НДР складає:

\[
	\text{Д}_{\text{Е}} = 42000,00 + 62500,00 = 104500,00 \, \text{грн}.
\]

\vspace{1.5em}

\subsection{Укрупнена оцінка прибутковості запропонованого інноваційного проекту}

Укрупнена оцінка прибутковості інноваційного проекту дипломної роботи припускає визначення наступних показників:

\begin{enumerate}
	\item Чистий дисконтований доход по роках реалізації проекту.
	\item Чиста поточна вартість проекту по роках реалізації проекту.
	\item Індекс прибутковості проекту.
	\item Внутрішня норма прибутковості.
	\item Строк окупності проекту.
\end{enumerate}

Розрахунок цих показників проводиться виходячи з наступних даних:

\begin{enumerate}
	\item Одноразові витрати в розрахунковому році (кошторис витрат на проведення й впровадження результатів НДР).
	\item Щорічні очікувані доходи від проекту.
	\item Процентна ставка в розрахунковому році.
	\item Інфляція на розглянутому ринку.
	\item Рівень ризику проекту.
\end{enumerate}

Для початку визначимо ставку дисконту проекту по формулі:

\begin{equation}\label{eq:economy12}
	d = k + i + r,
\end{equation}

\noindent де $k$ $-$ ціна капіталу (процентна ставка), частки одиниці, $k = 0,12$;
\hspace*{19pt}$i$ $-$ інфляція на ринку, частки одиниці, $i = 0.15$;
\hspace*{19pt}$r$ $-$ рівень ризику проекту, частки одиниці, $r = 0,06$.

Отже, згідно з формулою \ref{eq:economy12} $d = 0,33$.

Чистий дисконтований доход розраховуємо по формулі:

\begin{equation}\label{eq:economy13}
	\text{ЧДД}_{t} = \frac{\text{Д}_{t} - \text{К}_{t}}{(1 + d)^{t}},
\end{equation}

\noindent де $\text{Д}_{t}$ $-$ доходи $t$-го року, грн;
\hspace*{19pt}$\text{К}_{t}$ $-$ капіталовкладення (витрати) $t$-го року, грн (у цьому випадку  кошторис витрат на НДР).

\vspace{1.5em}

Чисту поточну вартість для $t$-го року реалізації проекту визначаємо по формулі:

\begin{equation}\label{eq:economy14}
	\text{ЧПВ}_{t} = -\frac{\text{К}_{t-1}}{(1 + d)^{t-1}} + \frac{\text{Д}_{t}}{(1 + d)^{t}}.
\end{equation}	

\vspace{1.5em}

Розрахунок даного показника варто здійснювати до першого позитивного значення ЧПВ. Цей рік і завершить розрахунковий період для даного інноваційного проекту.

Розрахунок даного показника варто здійснювати до першого позитивного значення ЧПВ. Цей рік і завершить розрахунковий період для даного інноваційного проекту. Приклад розрахунку чистого дисконтного доходу і чистої поточної вартості приведено в табл. \ref{tab:chdiscdokh}

\newpage

\begin{table}
	\captionstyle{ \raggedright}
	\caption{Розрахунок чистого дисконтного доходу і чистої поточної вартості }\label{tab:chdiscdokh}
	\begin{tabular}{| p{0.02\textwidth} | p{0.12\textwidth} | p{0.10\textwidth} | p{0.08\textwidth} | p{0.11\textwidth} | p{0.11\textwidth} | p{0.12\textwidth} | p{0.12\textwidth} |}
		\hline
		$t$ & \text{Д} & \text{К} & $\frac{1}{(1 + d)^{t}}$ & $\frac{\text{Д}}{(1 + d)^{t}}$ & $\frac{\text{К}}{(1 + d)^{t}}$ & $\text{ЧДД}$ & $\text{ЧПВ}$ \\
		\hlinewd{2pt}
		0 & $-$ & 66165,28 & 1 & $-$ & 66165,28 & -66165,28 & -66165,28 \\
		\hline
		1 & 104500,00 & $-$ & 0,75 & 78375,00 & $-$ & 78375,00 & 12209,72 \\ 
		\hline
		$\Sigma$ & 104500,00 & 66165,28 & $-$ &  78375,00 & 66165,28 & 12209,72 & $-$ \\
		\hline
	\end{tabular}
\end{table}

Індекс прибутковості визначимо за формулою:

\begin{equation}\label{eq:economy15}
	\text{ІП} = \frac{\sum_{t=0}^{T}\frac{\text{Д}}{(1 + d)^{t}}}{\sum_{t=0}^{T}\frac{\text{К}}{(1 + d)^{t}}} = \frac{78375,00}{66165,28} = 1,18,
\end{equation}

\noindent де $T$ $-$ кількість років у розрахунковому періоді.

\vspace{1.5em}

Внутрішня норма прибутковості являє собою ставку дисконту, при якій величина дисконтованих доходів усього розрахункового періоду дорівнює дисконтованим капіталовкладенням. Цей показник допомагає ухвалювати рішення щодо доцільності розробки й впровадження інноваційного проекту в умовах мінливих процентних ставок, ризиках, інфляції.

Внутрішню норму прибутковості можна визначити з табл. \ref{tab:vnp}

\begin{table}[h!]
	\captionstyle{ \raggedright}
	\caption{Розрахунок ВНП}\label{tab:vnp}
	\begin{tabular}{| p{0.18\textwidth} | p{0.18\textwidth} | p{0.18\textwidth} | p{0.18\textwidth} | p{0.18\textwidth} |}
		\hline
		$d$ & 0,33 & 0.5 & 0.57 & 0.58 \\
		\hlinewd{2pt}
		$\text{ЧДД} \, \text{грн}$ & 12406,14 & 3501,38 & 395,22 & -26,03 \\
		\hline
	\end{tabular}
\end{table}

Виходячи з табл. \ref{tab:vnp} ВНП складає приблизно 0,578.

Строк окупності розраховується починаючи з місяця запуску проекту до місяця в якому досягається наступна рівність:

\begin{equation}\label{eq:economy16}
	\sum_{t=1}^{T}\frac{\text{Д}}{(1 + d)^{t}} = \sum_{t=1}^{T}\frac{\text{К}}{(1 + d)^{t}}.
\end{equation}

\vspace{1.5em}

З табл. \ref{tab:chdiscdokh} можна зробити висновок, що термін окупності складає 11 місяців.

\subsection{Висновки за розділом}

У даному розділі дипломної роботи було проведено економічне обґрунтування дослідження параметрів порівняльного навчання для вирішення задач навчання без вчителя.

Проведено ознайомлення з методикою складання кошторису витрат на НДР. Були розраховані витрати на заробітну платню виконавців дипломної роботи, витрати на електроенергію, амортизаційні відрахування, відрахування на соціальне страхування, оренду приміщення, витратні матеріали і планові накопичення. Було встановлено код інновації. Також був зроблений розрахунок економічного ефекту від впровадження результатів НДР, розрахована укрупнена оцінка прибутковості запропонованого проекту та визначено його строк окупності. Даний проект є прибутковим, оскільки індекс прибутковості більше одиниці. Цей показник допомагає ухвалювати рішення щодо доцільності розробки й впровадження інноваційного проекту в умовах мінливих процентних ставок, ризиках, інфляції.

У таблиці \ref{tab:summary} можна ознайомитись з техніко-економічними показниками.

\begin{table}[h!]
	\captionstyle{ \raggedright}
	\caption{Техніко-економічні показники}\label{tab:summary}
	\begin{tabular}{| p{0.4\textwidth} | p{0.3\textwidth} |} 
		\hline
		Найменування показника & Величина \\
		\hlinewd{2pt}
		Кошторис витрат на НДР, грн & 66165б28 \\
		\hline
		Код інновації & 2.4.3.1.1.2.5.1.1 \\
		\hline
		Економічний ефект, грн & 104500,00 \\
		\hline
		Індекс прибутковості проекту & 1,18 \\
		\hline
		Внутрішня норма прибутковості & 0,578 \\
		\hline
		Строк окупності проекту, міс & 11 \\
		\hline
	\end{tabular}
\end{table}


%\newpage
%%\section{Охорона праці і навколишнього середовища}
\section{ОХОРОНА ПРАЦІ І НАВКОЛИШНЬОГО СЕРЕДОВИЩА}

Темою дипломної бакалаврської роботи є «Структурний аналіз та прогнозування роботи тестирующей системи DOTS». В процесі роботи була розглянута необхідна теорія, а також був створений програмний продукт. Важиливим аспектом роботи, що забезпечується застосуванням дисципліни охорони праці, є збереження здоров’я розробника ПЗ, підтримання ефективності та надійності його праці на належному рівні.

\subsection{Загальні питання охорони праці}

Закон України «Про охорону праці» [15] визначає основні положення щодо реалізації конституційного права працівників на охорону їх життя і здоров’я у процесі трудової діяльності, на належні, безпечні і здорові умови праці, регулює за участю відповідних органів державної влади відносини між роботодавцем і працівником з питань безпеки, гігієни праці та виробничого середовища і встановлює єдиний порядок організації охорони праці в Україні.

Професійні захворювання мають різний характер відповідно області, де працює людина. Розробка автоматизованих систем управління, створення ЕОМ полегшують і прискорюють виконання роботи. Але слід враховувати, що для запобігання отримання професійних захворювань час роботи за ЕОМ необхідно обмежувати, а саму роботу проводити на правильно організованому робочому місці.

Крім того, важливим аспектом є виконання задач Закону України «»Про охорону навколишнього природного середовища» [16], адже будь-яка виробнича діяльність впливає на навколишнє середовище.

\subsection{Структура управління охорони праці на підприємстві}

На підприємстві, де виконувалась дипломна робота впроваджена наступна структура: директор, заступник директора, технічний відділ, відділ реалізації продукції, бухгалтерія. Сукупна кількість співробітників складає 60 чоловік на 4 відділа. Є інженер з охорони праці. До переліку найважливіших і першочергових обов`язків інженера з охорони праці належать:

\begin{enumerate}
	\item атестація робочих місць;
	\item атестація робочого місця;
	\item атестація робочих.
\end{enumerate}

Система управління охороною праці (СУОП) — це сукупність органів управління підприємством, які на підставі комплексу нормативної документації проводять цілеспрямовану, планомірну діяльність щодо здійснення завдань і функцій управління з метою забезпечення здорових, безпечних і високопродуктивних умов праці.% Головна мета управління охороною праці є створення здорових, безпечних і високопродуктивних умов праці, покращення виробничого побуту, запобігання травматизму і профзахворюванням.

\subsection{Загальна характеристика приміщення та робочого місця}

Розглянемо характеристики робочого місця, на якому виконувалась дипломна робота, що наведені у табл. \ref{tab:charumov} та \ref{tab:charrobit}. Необхідно звернути увагу на можливі ризики задля забезпечення безпечних умов праці.

\begin{table}[h]
	\captionstyle{ \raggedright}
	\caption{Загальна характеристика умов праці}\label{tab:charumov}
	%\centering
	\begin{tabular}{|m{0.3\textwidth}|m{0.3\textwidth}|m{0.3\textwidth}|}
		\hline
		Шкідливі та небезпечні фактори & Джерела виникнення небезпек & Технічна характеристика робочого місця \\
		\hlinewd{2pt}
		Електрична напруга вище 127 В; шум; випромінювання – електромагнітні, радіаційні, теплові; пил; статична електрика; іонізація повітря; пожежна небезпека у приміщенні; неякісне освітлення &
		Вентиляція; 4-ПЕОМ; папір; світильники (лампи) &
		Розміри приміщення, м: довжина – 10; ширина – 5; висота – 3. Кількість працюючих – 6 \\
		\hline
	\end{tabular}
\end{table}

\newpage

\begin{table}[h]
	\captionstyle{ \raggedright}
	\caption{Загальна характеристика робіт, що виконуються}\label{tab:charrobit}
	%\centering
	\begin{tabular}{|m{0.2\textwidth}|m{0.3\textwidth}|m{0.4\textwidth}|}
		\hline
		Категорія за важкістю робіт & Показники напруженості трудового процесу & Ступінь відповідальності за результат своєї діяльності \\
		\hlinewd{2pt}
		Легка Іа & ступінь ризику для власного життя – виключений; ступінь відповідальності за безпеку інших осіб – виключений & значущість помилки –  допустимий (напруженість праці середнього ступеня); вимагає додаткових зусиль з боку керівництва; спостереження за екраном (2 - 3 годин на зміну) \\
		\hline
	\end{tabular}
\end{table}

\subsection{Метеорологічні параметри робочої зони}

Робота на персональній ЕОМ та розробка програмного продукту супроводжуються незначним фізичним напруженням, енерговитрати організму не перевищують 139 Вт, тому така робота, відповідно до ДСН 3.3.6.042-99 [17],  відноситься до категорії Iа – легка фізична робота.

Враховуючи високе нервово-емоційне напруження користувачів ЕОМ, згідно з ДСН 3.3.6.042-99 [17] у приміщеннях необхідно встановити оптимальні параметри мікроклімату. Під оптимальними параметрами мікроклімату розуміють такі параметри, які при тривалому і систематичному впливі на людину забезпечують збереження нормального функціонального і теплового стану організму без напруги реакцій терморегуляції, створюють відчуття теплового комфорту та є передумовою збереження високого рівня працездатності. Оптимальні параметри мікроклімату наведені в табл. \ref{tab:optclim}.

\begin{table}[h]
	\captionstyle{ \raggedright}
	\caption{Оптимальні параметри мікроклімату в приміщенні з ЕОМ}\label{tab:optclim}
	%\centering
	\begin{tabular}{|m{0.13\textwidth}|m{0.3\textwidth}|m{0.16\textwidth}|m{0.15\textwidth}|m{0.16\textwidth}|}
		\hline
		Період року & Категорія виконуваних робіт за енерговитратами & Температура, $^{\degree}$С & Відносна вологість, \% & Швидкість руху повітря, м/с \\
		\hlinewd{2pt}
		Холодний & Легка Іа & 22-24 & 40-60 & 0,1 \\
		\hline
		Теплий & Легка Іа & 23-25 & 40-60 & 0,1 \\
		\hline
	\end{tabular}
\end{table}

Згідно з ДБН В.2.5-67-2013 [18], у теплий період року використовуються кондиціонери, що автоматично підтримують необхідні оптимальні параметри температури, незалежно від зовнішніх умов, а також проводиться природне провітрювання приміщення; в холодний період – проводиться опалення від центральної тепломережі і природна вентиляція.

\subsection{Освітлення приміщення}

Згідно з ДБН В.2.5-28-2018 [19] для об’єктів, пов’язаних з освітленням, відповідно до розміру об’єкта та характеристики зорової роботи визначені нормативні характеристики, які відображенні у табл. \ref{tab:normosv}.

\begin{table}[h]
	\captionstyle{ \raggedright}
	\caption{Нормативні параметри освітлення для роботи в ЕОМ}\label{tab:normosv}
	%\centering
	\begin{tabular}{|m{0.11\textwidth}|m{0.1\textwidth}|m{0.1\textwidth}|m{0.1\textwidth}|m{0.1\textwidth}|m{0.11\textwidth}|m{0.1\textwidth}|m{0.1\textwidth}|}
		\hline
		\parbox[t]{0.11\textwidth}{Харак-\\теристи-\\ка зорової роботи} & \parbox[t]{0.1\textwidth}{Наймен-\\ший розмір об’єкта розпізнавання, мм} & Розряд зорової роботи & \parbox[t]{0.1\textwidth}{Підроз-\\ряд зорової роботи} & Контраст об’єкта розпізнавання & \parbox[t]{0.11\textwidth}{Харак-\\теристи-\\ка фону} & \parbox[t]{0.1\textwidth}{Освітле-\\ність при штучному освітленні, лк} & КПО, $\text{е}_{\text{н}}$, при суміщеному освітленні, \% \\
		\hlinewd{2pt}
		Дуже високої точності & Від 0,15 до 0,3 & ІІ & в & Середній & Середній & 500 & 1,5 \\
		\hline
	\end{tabular}
\end{table}

Коефіцієнт природного освітлення (КПО) – процентне відношення природної освітленості у будь-якій точці в середині приміщення до одночасно виміряної на тому ж рівні освітленості зовнішньої горизонтальної площини рівномірно розсіяним (дифузійним) усього небосхилу:

\begin{equation}\label{eq:kpo}
\text{КПО} = \text{е}_{\text{н}} = \frac{E_{\text{в}}}{E_{\text{з}}} \cdot 100 \%.
\end{equation}

Коефіцієнт природного освітлення (КПО) показує, яку частину зовнішнього дифузійного світла небосхилу у процентах становить освітлення в певній точці на робочій поверхні всередині приміщення.

Для приміщень з одностороннім бічним освітленням нормується мінімальне значення КПО у точці, розташованій на відстані 1 м від стінки, найменш віддаленої від світлових прорізів, на перерізі вертикальної площини характерного розрізу приміщення та умовної робочої поверхні.

Для приміщень із двостороннім бічним освітленням нормується мінімальне значення КПО у точці посередині приміщення на перерізі вертикальної площини характерного розрізу приміщення та умовної робочої поверхні.

При верхньому або комбінованому освітленні нормується середнє значення КПО у точках, розташованих на перерізі вертикальної площини характерного розрізу приміщення та умовної робочої поверхні. При цьому перша та остання точки приближаються на відстані 1 м від поверхні стін або перегородок.

У разі комбінованого освітлення допускається розподіл приміщення на зони з бічним (прилеглі до зовнішніх стін з вікнами) та верхнім освітленням. Нормування та розрахунок природного освітлення у кожній зоні проводиться окремо.

Під час нормування природної освітленості визначається найменший розмір об’єкта розрізнення, відповідний йому розряд зорової роботи та нормований коефіцієнт природної освітленості.

В приміщенні, що розглядається, застосовують суміщене освітлення $–$ освітлення, при якому недостатнє за нормами природне освітлення доповнюється штучним. Мінімальна освітленість при цьому складає 500 лк. Штучне освітлення реалізується шляхом встановлення визначеної кількості ламп білого світла – ЛБ 80.

\newpage

\subsection{Шум та вібрація у робочому приміщенні}

Шум є одним з найбільш розповсюджених у виробництві шкідливих факторів. 

Джерелами шуму і вібрації є вентилятори системного блоку, накопичувач, розташовані в системному блоці комп’ютера, і принтер. Це може стати джерелом стресу і дискомфорту користувача, знижувати розумову працездатність, підвищувати втомлюваність, послаблювати увагу, сприяти появі головного болю тощо. Відповідно до ДСН 3.3.6. 037-99 [20] робочі місця у приміщеннях програмістів обчислювальних машин рівень шуму не повинен перевищувати 50 дБА. Відповідно до ДСН 3.3.6. 039-99 [21] рівень загальної вібрації для категорії 3, технологічного типу «в» не повинен перевищувати 75 дБ.

Як захист від шуму, який створюється вентиляторами системних блоків, використовується наступне:

\begin{enumerate}
	\item звукоізоляційний корпус;
	\item заміна вентилятора на більш якісний;
	\item використання звукопоглинаючих та звукоізолюючих засобів;
	\item мідні радіатори як альтернативу вентилятору;
	\item при монтажі кулерів замість гвинтів встановлювати гумові пробки,\newline \hspace*{-18mm}що дозволяють ізолювати вентилятор від корпуса.
\end{enumerate}

\subsection{Електробезпека у робочому приміщенні}

Сучасні приміщення та виробництво нерозривно пов’язане з використанням електроенергії. Тому ці приміщення є приміщеннями з підвищеною небезпекою ураження людей електричним струмом. Основними причинами ураження струмом є:

\begin{enumerate}
	\item випадкове доторкання до струмоведучих частин, що перебувають\newline \hspace*{-18mm}під напругою;
	\item поява напруги на металевих конструктивних частинах\newline \hspace*{-18mm}електрообладнання;
	\item поява напруги на відімкнених струмоведучих частинах в\newline \hspace*{-18mm}результаті;
	\item виникнення напруги кроку на ділянці землі, де перебуває людина,\newline \hspace*{-18mm} в результаті: замикання фази на землю; несправності у обладнанні\newline \hspace*{-18mm}захисного заземлення тощо.
\end{enumerate}

При виконанні роботи використовувався комп’ютер, який живиться з напругою 220 В від однієї фази трьохфазної чотирьохпровідної мережі з глухозаземленою нейтраллю.

Основними заходами захисту від ураження електричним струмом згідно з ПУЕ-2017 [22] є:

\begin{enumerate}
	\item забезпечення недоступності струмопровідних частин, що\newline \hspace*{-18mm}перебувають під напругою, для випадкового дотику;
	\item організація безпечної експлуатації електроустановок;
	\item компенсація ємнісної складової струму замикання на землю;
	\item застосування спеціальних засобів – переносних приладів і\newline \hspace*{-18mm}запобіжних пристроїв;
	\item відключення електроустаткування, що ремонтується, і вживання\newline \hspace*{-18mm}заходів проти помилкового його зворотного включення або самовключення;
	\item проведення низки організаційних заходів (спеціальне навчання,\newline \hspace*{-18mm}атестація і переатестація осіб електротехнічного персоналу, інструктажі \newline \hspace*{-18mm}ощо).
\end{enumerate}

Для захисту від ураження електричним струмом для ЕОМ застосовується занулення – це навмисне електричне з’єднання з нульовим захисним провідником металевих не струмоведучих частин електроустановки, які можуть опинитися під напругою. Головне призначення захисного заземлення – знизити потенціал на корпусі електроустаткування до безпечного значення.

\subsection{Ергономічні вимоги до робочого місця}

Електромагнітне випромінювання шкідливо впливає на здоров’я людини. Згідно НПАОП 0.00-1.28-2010 [23], потужність поглиненої дози в повітрі за рахунок супутнього не використаного рентгенівського випромінювання не повинна перевищувати 100  на відстані 5 см від поверхні пристрою, під час роботи якого воно виникає. Забезпечення захисту оператора та досягнення нормованих рівнів випромінювань ЕОМ рекомендовано застосування екранних фільтрів, локальних світлофорів та інших засобів захисту, які пройшли випробування в акредитованих лабораторіях та отримали позитивний висновок державної санітарно-епідеміологічної експертизи.

Основними принципами захисту від впливу ЕОВ є:

\begin{enumerate}
	\item тривалість роботи за ЕОМ не повинна перевищувати 4 години на\newline \hspace*{-18mm}день при цьому виконувати перерви через кожні 2 години роботи;
	\item на одну ЕОМ повинно бути виділено не менше 6 $\text{м}^{2}$, відстань між\newline \hspace*{-18mm}сусідніми ЕОМ – 1,5м.
	\item внутрішнє екранування, що дозволяє суттєво знизити\newline \hspace*{-18mm}інтенсивність шкідливого опромінювання;
	\item для попередження, своєчасної діагностики та лікування здоров’я\newline \hspace*{-18mm}людини, що пов’язано з негативним впливом ЕОМ, користувачі повинні проходити попередні (під час прийому на роботу) і періодичні\newline \hspace*{-18mm}медичні огляди.
\end{enumerate}

\subsection{Пожежна безпека}

Пожежна безпека – стан об’єкта, при якому з встановленою ймовірністю виключається ймовірність виникнення і розвитку пожежі.

По категорії вибухопожежної та пожежної небезпеки, згідно з ДСТУ Б.В.1.1-36:2016 [24] приміщення, у якому виконувалась дипломна бакалаврська робота, відноситься до категорії В – пожежонебезпечне через присутність твердих спаленних матеріалів, таких як: робочі столи, ізоляція, папір та інше. Виходячи з категорії пожежонебезпеки і поверховості будинку, ступінь вогнестійкості будинку ІІ згідно з ДБН В.1.1-7-2016 [25].

Причинами, які можуть викликати пожежу, в розглянутому приміщенні є: несправність електричної проводки і приладів, коротке замикання електричного кола, перегрів апаратури, блискавка тощо.

Згідно з вимогами ДБН В.2.5-56-2015 [26] пожежна небезпека забезпечується наступними мірами:

\begin{enumerate}
	\item організаційними заходами щодо пожежної безпеки;
	\item системою протипожежного захисту;
	\item системою запобігання пожеж, яка передбачає запобігання\newline \hspace*{-18mm}утворення пального середовища і запобігання утворення в пальному\newline \hspace*{-18mm}середовищі джерел запалювання.
\end{enumerate}

При виборі засобів гасіння пожежі для забезпечення безпеки людини від можливості поразки електричним струмом у приміщенні передбачено використання вуглекислотних вогнегасників. Вогнегасник знаходиться на видному і легко доступному місці. Відстань від можливого осередку пожежі до місця розташування вогнегасника має бути не більше, ніж 30 м. При виникненні пожежі передбачена можливість повідомлення в пожежну охорону по телефону. Також необхідним заходом безпеки є евакуаційні виходи (не менше двох).

Організаційними заходами протипожежної профілактики є вступний інструктаж при надходженні на роботу, навчання виробничого персоналу протипожежним правилам, видання необхідних інструкцій і плакатів, засобів наочної агітації, наявність плану евакуації.

\subsection{Охорона навколишнього природного середовища}

Проблема охорони й оптимізації навколишнього природного середовища виникла як неминучий наслідок сучасної промислової революції.

Збільшення використання енергії призводить до порушення екологічної рівноваги природного середовища, яке складалася століттями.

Поряд з цим, підвищення технічної оснащеності підприємств, застосування нових матеріалів, конструкцій і процесів, збільшення швидкостей і потужностей виробничих машин впливають на навколишнє середовище.

Основними задачами Закону України «Про охорону навколишнього природного середовища» [16], прийнятого 25 червня 1991 року, є регулювання відносин в області охорони природи, використання і відтворення природних ресурсів, забезпечення екологічної безпеки, попередження і ліквідація наслідків негативного впливу на навколишнє середовище господарської й іншої діяльності людини, збереження природних ресурсів, генетичного фонду, ландшафтів і інших природних об’єктів.

При масовому використанні моніторів та комп’ютерів не можна не враховувати їхній вплив на навколишнє середовище на всіх стадіях – при виготовленні, експлуатації та після закінчення терміну служби.

Міжнародні екологічні стандарти, що діють на сьогоднішній день в усьому світі, визначають набір обмежень до технологій виробництва та матеріалів, які можуть використовуватися в конструкціях пристроїв. Так, за стандартом ТСО-95, вони не повинні містити фреонів (турбота про озоновий шар), полівінілхлориді, бромідів (як засобів захисту від загоряння).

У стандарті ТСО-99 закладене обмеження за кадмієм у світлочутливому шарі екрана дисплея та ртуті в батарейках; э чіткі вказівки відносно пластмас, лаків та покриттів, що використовуються. Відмовитися від свинцю в ЕЛТ поки неможливо. Поверхня кнопок не повинна містити хром, нікель та інші матеріали, які визивають алергічну реакцію. ГДК пилу дорівнює 0,15 $\text{мг}/\text{м}^{3}$, рекомендовано 0,075 $\text{мг}/\text{м}^{3}$; ГДК озону під час роботи лазерного принтеру $-$ 0,02 $\text{мг}/\text{м}^{3}$. Особливо жорсткі вимоги до повторно використовуваних матеріалів. 

Апарати, тара і документація повинні допускати нетоксичну вторинну переробку після закінчення терміну експлуатації. В ЕПТ міститься багато біоактивних речовин, що треба ураховувати під час утилізації.

Міжнародні стандарти, починаючи з ТСО-92, включають вимоги зниженого енергоспоживання та обмеження припустимих рівнів потужності, що споживаються у неактивних режимах.

Дотримання приведених нормативних параметрів небезпечних і шкідливих виробничих факторів дозволить забезпечити більш здорові і безпечні умови роботи користувача ЕОМ.
\newpage
\subsection{Висновки за розділом}

У нашій країні питання охорони навколишнього середовища, зниження негативних наслідків втручання людини у всі сфери її життєдіяльності –  це одна з найгостріших проблем. Вона потребує негайного вирішення, особливо на сучасному етапі розвитку та вдосконалення комп’ютерної, обчислювальної техніки.

Дотримання приведених нормативних параметрів небезпечних і шкідливих виробничих факторів дозволить забезпечити більш здорові і безпечні умови роботи користувача ЕОМ.


%\newpage
%%\section{Цивільний захист}
\section{ЦИВІЛЬНИЙ ЗАХИСТ}

Цивільний захист $-$ це функція держави, спрямована на захист населення, території, навколишнього природного середовища та майна від надзвичайних ситуацій шляхом запобігання таких ситуацій, ліквідації їх наслідків та надання допомоги постраждалим в мирний час та в особливий період [13].

У даному розділі дипломної роботи розглянуті питання щодо концепції оповіщення населення в умовах надзвичайної ситуації (НС).

Актуальність проблеми оповіщення населення зумовлена тенденціями зростання втрат людей і шкоди територіям у результаті небезпечних природних явищ і катастроф. Ризик надзвичайних ситуацій техногенного і природного характеру постійно зростає

Рівень національної безпеки не може бути достатнім, якщо в загальнодержавному масштабі не буде вирішено завдання захисту населення, об'єктів економіки і національного надбання від надзвичайних ситуацій техногенного і природного характеру [14].

Основними завданнями захисту населення і території від надзвичайних ситуацій техногенного і природного характеру є:

\begin{enumerate}
	\item здійснення \hfill комплексу \hfill заходів\hfill щодо\hfill запобігання\hfill і\hfill реагування \newline \hspace*{-20mm} на надзвичайні ситуації техногенного і природного характеру;
	\item забезпечення\hfill готовності\hfill і\hfill контролю\hfill за\hfill станом\hfill готовності\hfill до\newline \hspace*{-20mm} дій\hfill і\hfill взаємодії\hfill органів\hfill управління\hfill в\hfill цій\hfill сфері,\hfill сил\hfill і\hfill засобів,\hfill призначених\newline \hspace*{-20mm} для\hfill запобігання\hfill надзвичайних\hfill ситуацій\hfill техногенного\hfill і\hfill природного\newline \hspace*{-20mm} характеру і реагування на них
\end{enumerate}

\subsection{Оповіщення населення}

Серед комплексу заходів з захисту населення за надзвичайних умов важливе місце посідає організація своєчасного інформування та оповіщення, які покладаються на органи цивільної оборони і є невід’ємним елементом усієї системи заходів [15].

Центральні та місцеві органи влади зобов’язані надавати населенню через засоби масової інформації оперативну і достовірну інформацію про стан захисту населення від НС, методи та способи їх захисту, вжиття заходів щодо забезпечення безпеки [16].

Оповіщення про загрозу виникнення НС і постійне інформування населення про них забезпечуються шляхом:

\begin{enumerate}
	\item завчасного\hfill створення\hfill і\hfill підтримки\hfill у\hfill постійній\hfill готовності\newline \hspace*{-20mm} загальнодержавної\hfill і\hfill територіальних\hfill автоматизованих\hfill систем\hfill центрального\newline \hspace*{-20mm} оповіщення населення;
	\item організаційно-технічного\hfill з’єднання\hfill територіальних\newline \hspace*{-20mm} систем\hfill центрального\hfill оповіщення\hfill і\hfill систем\hfill оповіщення\hfill на\hfill об’єктах\newline \hspace*{-20mm} господарювання;
	\item завчасного\hfill створення\hfill та\hfill організації\hfill технічного\hfill з’єднання\hfill з\newline \hspace*{-20mm} системами\hfill спостереження\hfill і\hfill контролю\hfill постійно\hfill діючих\hfill локальних\newline \hspace*{-20mm} систем\hfill оповіщення\hfill та\hfill інформування\hfill населення\hfill в\hfill зонах\hfill катастрофічного\newline \hspace*{-20mm} затоплення,\hfill районах\hfill розміщення\hfill радіаційних,\hfill хімічних\hfill підприємств,\hfill інших\newline \hspace*{-20mm} об’єктів підвищеної небезпеки;
	\item центрального\hfill використання\hfill загальнодержавних\hfill і\hfill галузевих\newline \hspace*{-20mm} систем\hfill зв’язку:\hfill радіо,\hfill провідного,\hfill телевізійного\hfill оповіщення,\newline \hspace*{-20mm} радіотрансляційних мереж та інших технічних засобів передачі інформації.
\end{enumerate}

Оповіщення організовують засобами радіо та телебачення. Для того, щоб населення своєчасно увімкнуло засоби оповіщення, використовують сигнали транспортних засобів, а також переривисті гудки підприємств.

Завивання сирен, переривисті гудки підприємств та сигнали транспортних засобів означають попереджувальний сигнал "Увага всім!". Той, хто почув цей сигнал, повинен негайно увімкнути теле- чи радіоприймачі та прослухати екстрене повідомлення місцевих органів влади чи управління з НС та цивільного захисту населення. Усі подальші дії визначаються їхніми вказівками.

Для своєчасного попередження населення введені сигнали попередження населення у мирний і воєнний час [15].

Сигнал "Увага всім!" повідомляє населення про надзвичайну обстановку в мирний час і на випадок загрози нападу противника у воєнний час. Сигнал подається органами цивільного захисту за допомогою сирени і виробничих гудків. Тривалі гудки означають попереджувальний сигнал.

\subsubsection{Сигнали оповіщення в мирний час}

"Аварія на атомній електростанції". Повідомляються місце, час, масштаби аварії, інформація про радіаційну обстановку та дії населення. Якщо є загроза забруднення радіоактивними речовинами, необхідно провести герметизацію житлових, виробничих і складських приміщень. Провести заходи захисту від радіоактивних речовин сільськогосподарських тварин, кормів, урожаю, продуктів харчування та води. Прийняти йодні препарати. Надалі діяти відповідно до вказівок штабу органів цивільного захисту.

"Аварія на хімічно небезпечному об'єкті". Повідомляються місце, час, масштаби аварії, інформація про можливе хімічне зараження території, напрямок та швидкість можливого руху зараженого повітря, райони, яким загрожує небезпека. Дається інформація про поведінку населення. Залежно від обставин: залишатися на місці, у закритих житлових приміщеннях, на робочих місцях чи залишати їх і, застосувавши засоби індивідуального захисту, вирушити на місця збору для евакуації або в захисні споруди. Надалі діяти відповідно до вказівок штабу органів управління цивільного захисту.

"Землетрус". Подається повідомлення про загрозу землетрусу або його початок. Населення попереджається про необхідність відключити газ, воду, електроенергію, погасити вогонь у печах; повідомити сусідів про одержану інформацію; взяти необхідний одяг, документи, продукти харчування, вийти на вулицю і розміститися на відкритій місцевості на безпечній відстані від будинків, споруд, ліній електропередачі.

"Затоплення". Повідомляється район, в якому очікується затоплення в результаті підйому рівня води в річці чи аварії дамби.

Населення, яке проживає в даному районі, повинне взяти необхідні речі, документи, продукти харчування, воду, виключити електроенергію, відключити газ і зібратись у вказаному місці для евакуації. Повідомити сусідів про стихійне лихо і надалі слухати інформацію штабу органів управління цивільного захисту.

"Штормове попередження". Подається інформація для населення про посилення вітру. Населенню необхідно зачинити вікна, двері. Закрити в приміщеннях сільськогосподарських тварин. Повідомити сусідів. Населенню, по можливості, перейти в підвали, погреби.

\subsubsection{Сигнали оповіщення в воєнний час}

Сигнал "Повітряна тривога" подається для всього населення. Попереджається про небезпеку ураження противником даного району. По радіо передається текст: "Увага! Увага! Повітряна тривога! Повітряна тривога!" Одночасно сигнал дублюється сиренами, гудками підприємств і транспорту. Тривалість сигналу 2—3 хв.

При цьому сигналі об'єкти припиняють роботу, транспорт зупиняється і все населення укривається в захисних спорудах. Робітники і службовці припиняють роботу відповідно до інструкції і вказівок адміністрації. Там, де неможливо через технологічний процес або через вимоги безпеки зупинити виробництво, залишаються чергові, для яких мають бути захисні споруди.

Сигнал може застати у будь-якому місці й будь-який час. В усіх випадках необхідно діяти швидко, але спокійно, впевнено, без паніки. Суворо дотримуватися правил поведінки, вказівок органів цивільного захисту.

Сигнал "Відбій повітряної тривоги". Органами цивільного захисту через радіотрансляційну мережу передається текст: "Увага! Увага! Громадяни! Відбій повітряної тривоги!". За цим сигналом населення залишає захисні споруди і повертається на свої робочі місця і в житла.

Сигнал "Радіаційна небезпека" подається в населених пунктах і в районах, в напрямку яких рухається радіоактивна хмара, що утворилася від вибуху ядерного боєприпасу.

Почувши цей сигнал, необхідно з індивідуальної аптечки ЛІ-2 прийняти 6 таблеток радіозахисного препарату № 1 із гнізда 4, надіти респіратор, протипилову пов'язку, ватно-марлеву маску або протигаз, взяти запас продуктів, документи, медикаменти, предмети першої потреби і направитися у сховище або ПРУ.

Сигнал "Хімічна тривога" подається у разі загрози або безпосереднього виявлення хімічного або бактеріологічного нападу (зараження). При цьому сигналі необхідно прийняти з індивідуальної аптечки АІ-2 одну таблетку препарату при отруєнні фосфорорганічними речовинами з пенала з гнізда 2 або 5 таблеток протибактеріального препарату № 1 із гнізда 5, швидко надіти протигаз, а за необхідності $-$ і засоби захисту шкіри, якщо можливо, та укритися в захисних спорудах. Якщо таких поблизу немає, то від ураження аерозолями отруйних речовин і бактеріальних засобів можна сховатися в житлових чи виробничих приміщеннях.

При застосуванні противником біологічної зброї населенню буде подана інформація про наступні дії.

Успіх захисту населення залежатиме від дисциплінованості, своєчасної і правильної поведінки, суворого дотримання рекомендацій і вимог органів цивільного захисту.

\subsubsection{Заходи протирадіаційного та протихімічного захисту}

Протирадіаційний та протихімічний захист (ПР та ПХЗ) $-$ це комплекс заходів ЦО, які направлені на запобігання чи послаблення дії іонізуючого опромінення, сильнодіючих та отруйних речовин

ПР та ПХЗ включають такі заходи:

\begin{enumerate}
	\item виявлення та оцінка радіаційної та хімічної обстановки;
	\item розробка та введення в дію режимів радіаційного захисту;
	\item організація та проведення дозиметричного та хімічного контролю;
	\item способи\hfill захисту\hfill населення\hfill при\hfill радіоактивному\hfill та\hfill хімічному\newline \hspace*{-20mm} забрудненні;
	\item забезпечення населення та формувань ЦО засобами ПР та ПХЗ;
	\item ліквідація\hfill наслідків\hfill зараження,\hfill спеціальна\hfill санітарна\hfill обробка,\newline \hspace*{-20mm} знезаражування\hfill місцевості\hfill та\hfill будівель\hfill тощо.\hfill ПР\hfill та\hfill ПХЗ\hfill організовують\newline \hspace*{-20mm} завчасно начальники ЦО об’єктів і командири формувань.
\end{enumerate}

\subsection{Висновки за розділом}

Таким чином, за допомогою правильно розроблених та впроваджених в життя заходів із оповіщення населення в умовах НС, в разі їх виникнення, можливо мінімізувати ризики для життя та здоров’я персоналу та відвідувачів цих підприємств і розміри заподіяної шкоди. 

%\newpage
%%\anonsection{Висновки}
\begin{center}
\textbf{ВИСНОВКИ}
\end{center}
\label{sec:Summary}
\addcontentsline{toc}{section}{Висновки}

%\hspace*{26pt} 
Виконання дипломної роботи складалося з наступних етапів:

\begin{enumerate}
	\item проведення аналізу літературних джерел;
	\item засвоєння\hfill алгоритмів\hfill Deep InfoMax\hfill та\hfill Momentum Contrast\hfill для\newline \hspace*{-20mm} вирішення задачі навчання без учителя;
	\item реалізація\hfill методів\hfill Deep InfoMax\hfill та\hfill Momentum Contrast\hfill з\newline \hspace*{-20mm} використанням бібліотек мови програмування Python.
\end{enumerate}

Результати прогнозування показали, що алгоритм Deep InfoMax дає кращі результати, в той час як Momentum Constras $-$ більш вигідний з точки зору часу та обчислювальних ресурсів.

%\newpage
%%\anonsection{Список джерел інформації}
\anonsection{СПИСОК ДЖЕРЕЛ ІНФОРМАЦІЇ}
\addcontentsline{toc}{section}{Список джерел інформації}

\hspace*{26pt} 1 

2 

3

4 

5 

6 

7 

8 

9 

10 

11 

12 Методичні вказівки для розробки розділу «Охорона праці та навколишнього середовища» у випускних дипломних роботах студентів інженерно $-$ фізичного факультету та факультету «Інформатика і управління» очної та заочної форм навчання / Уклад. В. В. Березуцький, О. О. Кузьменко, М. М. Латишева. $-$ Харків: НТУ «ХПІ», 2020.$-$ 60 с. $-$ Укр. мовою.

13 Кодекс цивільного захисту України – ВРУ №5403-VI, від 2.10.2012 р.

14 Концепція «Про захист населення і територій при загрозі і виникненні надзвичайної ситуації», схвалена Наказом Президента України від 26.03.1999 року № 234/99.

15 Кулаков М. А. Цивільна оборона : навч. посіб. / М. А. Кулаков, Т. В. Кукленко, В. О. Ляпун, В. О. Мягкий. – Х. : Факт, 2008. – 157 с.

16 Стеблюк М. І. Цивільна оборона  : підруч. 3-тє вид., перероб. і доп. / М. І. Стеблюк. – К.: Знання, 2004. – 332 с.

\end{document}
