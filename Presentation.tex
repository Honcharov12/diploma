\documentclass[c]{beamer}

\usetheme{Berlin}

%%% Работа с русским языком
\usepackage{cmap}					% поиск в PDF
\usepackage{mathtext} 				% русские буквы в формулах
\usepackage[T2A]{fontenc}			% кодировка
\usepackage[utf8]{inputenc}			% кодировка исходного текста
\usepackage[ukrainian,english,russian]{babel}	% локализация и переносы

%% Beamer по-русски
\newtheorem{rtheorem}{Теорема}
\newtheorem{rproof}{Доказательство}
\newtheorem{rexample}{Пример}

%%% Дополнительная работа с математикой
\usepackage{amsmath,amsfonts,amssymb,amsthm,mathtools} % AMS
\usepackage{icomma} % "Умная" запятая: $0,2$ --- число, $0, 2$ --- перечисление

%% Номера формул
%\mathtoolsset{showonlyrefs=true} % Показывать номера только у тех формул, на которые есть \eqref{} в тексте.
%\usepackage{leqno} % Нумерация формул слева

%% Свои команды
\DeclareMathOperator{\sgn}{\mathop{sgn}}

%% Перенос знаков в формулах (по Львовскому)
\newcommand*{\hm}[1]{#1\nobreak\discretionary{}
{\hbox{$\mathsurround=0pt #1$}}{}}

%%% Работа с картинками
\usepackage{graphicx}  % Для вставки рисунков
\graphicspath{{Mal/}}  % папки с картинками
\setlength\fboxsep{3pt} % Отступ рамки \fbox{} от рисунка
\setlength\fboxrule{1pt} % Толщина линий рамки \fbox{}
\usepackage{wrapfig} % Обтекание рисунков текстом

%%% Работа с таблицами
\usepackage{array,tabularx,tabulary,booktabs} % Дополнительная работа с таблицами
\usepackage{longtable}  % Длинные таблицы
\usepackage{multirow} % Слияние строк в таблице

%%% Программирование
\usepackage{etoolbox} % логические операторы

%%% Другие пакеты
\usepackage{lastpage} % Узнать, сколько всего страниц в документе.
\usepackage{soul} % Модификаторы начертания
\usepackage{csquotes} % Еще инструменты для ссылок
%\usepackage[style=authoryear,maxcitenames=2,backend=biber,sorting=nty]{biblatex}
\usepackage{multicol} % Несколько колонок

%%% Картинки
\usepackage{tikz} % Работа с графикой
\usepackage{pgfplots}
\usepackage{pgfplotstable}

\setbeamertemplate{footline}[frame number]

\title{Дипломна робота}
\author{\Large Розробка методів представлення візуальної інформації за допомогою методів самонавчання та contrastive learning}
\institute{\large Виконав: \\ студент групи КН-н119 Гончаров В. А. \\
\large Керівник дипломної роботи: \\ст. викладач каф. КМАД Колбасін В. О.}
\date{Харків 2021}
\subtitle[НТУ <<ХПІ>>]{
 Міністерство Освіти і Науки України \\
Національний технічний університет <<Харківський політехнічний інститут>> \\
Факультет: Комп’ютерних наук та програмної інженерії \\
Кафедра: Комп’ютерної математики та аналізу даних}

\begin{document}

\frame[plain]{\titlepage}	% Титульный слайд

\section{Вступ}

\begin{frame}
	\frametitle{\insertsection}
	Об'єкт дослідження $-$ аналіз роботи алгоритмів порівняльного навчання (contrastive learning) на датасеті CIFAR-10.
%	Объект исследованя - тестирующая система DOTS (от англ. Docker-oriented testing system). %$-$ это система автоматической проверки задач по программированию.
%	DOTS $-$ самая масштабная система в харьковской области и одна из самых распространённых на территории Украины. На её основе проводятся различные олимпиады, турниры, курсы по программированию.
	
	\pause

	Мета $-$ побудова моделей на основі машинного навчання та дослідження якості їхньої роботи в залежності від параметрів.
%	Для функционирования системы необходимы сервера. 
%	Цель исследования $-$ прогнозирование ряда загруженности серверов.

	\pause

	Методи дослідження $-$ алгоритми Deep InfoMax та Momentum Contrast.
%	Методы исследования $-$ математическая модель ARIMA и алгоритм SSA.
\end{frame}

\begin{frame}
	\frametitle{\insertsection}
	Задачі:\pause
	\begin{enumerate}
	\item вибір даних для аналізу роботи алгоритмів;\pause
	\item реалізація методів Deep InfoMax та Momentum Contrast;\pause
	\item дослідження роботи вищеназваних методів на обраних даних;\pause
	\item порівняння роботи алгоритмів.
	\end{enumerate}
\end{frame}

\section{Теоретична частина}

\subsection{Self-supervised learning}

\begin{frame}
	\frametitle{\insertsection}
	\frametitle{\insertsubsection}
	
  	\includegraphics[width=\textwidth, height=4cm, natwidth=850, natheight=450]{self_supervised.jpg}

	Самонавчання $-$ це технологія навчання комп'ютерів виконання різноманітних задач без надання людьми маркованих даних.
\end{frame}

\subsection{Contrastive learning}

\begin{frame}
	\frametitle{\insertsection}
	\frametitle{\insertsubsection}
	
  	\includegraphics[width=\textwidth, height=4cm, natwidth=497, natheight=349]{contrastive.jpg}

	Порівняльне навчання $-$ це техніка машинного навчання, яка використовується для вивчення загальних особливостей набору даних без міток, навчаючи моделі, які приклади даних подібні чи різні.
\end{frame}


\begin{frame}
	\frametitle{\insertsection}
	\frametitle{\insertsubsection}
	
	В якості оптимізуємої функції використовується взаємна інформація або пов'язані з нею функції, наприклад, її нижня оцінка InfoNCE :

\begin{equation*}\label{eq:infonce}
-E_{x}\left[ \log{\frac{exp(f(x)^{T}f(x^{+}))}{exp(f(x)^{T}f(x^{+}))+\sum_{j=1}^{N-1}exp(f(x)^{T}f(x_{j}))}} \right].
\end{equation*}

\end{frame}

\section{Практична частина}

\begin{frame}
	\frametitle{\insertsection}
	
	В процесі роботи була використана мова програмування Python зі спеціалізованими бібліотеками PyTorch, NumPy, Matplotliib.\pause
	
	
	Для демонстрації та аналізу роботи алгоритмів були використані дані з датасету CIFAR-10.

%	Набір даних CIFAR-10 складається з 60000 кольорових зображень розміром 32x32 у 10 класах, по 6000 зображень на клас. Існує 50000 навчальних зображень та 10000 тестових зображень [13].

  	\centering\includegraphics[width=0.5\textwidth, height=4cm, natwidth=471, natheight=370]{cifar10.jpg}
  %	\centering\includegraphics[width=0.80\textwidth, height=4cm, natwidth=854, natheight=476]{TimeSeries.jpg}

\end{frame}

\subsection{Deep InfoMax}

\begin{frame}
	\frametitle{\insertsection}
	\framesubtitle{\insertsubsection}

	Результати тестування алгоритму Deep InfoMax, $\alpha = 0,5$, $\beta = 0,9$, $\gamma = 0,1$, learning rate = 0,03. Помилка $-$ 32,24 \%, п'ять годин тренування.
		
    \includegraphics[width=\textwidth, height=5cm, natwidth=375, natheight=121]{deepinfodemo3.jpg}


\end{frame}


\subsection{Momentum Contrast}

\begin{frame}
	\frametitle{\insertsection}
	\framesubtitle{\insertsubsection}

	Результати тестування алгоритму Momentum Contrast, $\tau = 0,7$, learning rate = 0,2. Помилка $-$ 37,42 \%, чотири години тренування.
	
    \includegraphics[width=\textwidth, height=4cm, natwidth=621, natheight=456]{mocodemo3.jpg}

\end{frame}

\section{Висновки}

\begin{frame}
	\frametitle{\insertsection}
	Виконання дипломної роботи складалося з наступних етапів:

	\begin{enumerate}
		\item проведення аналізу літературних джерел;\pause
		\item засвоєння алгоритмів Deep InfoMax та Momentum Contrast для вирішення задачі навчання без учителя;\pause
		\item реалізація методів Deep InfoMax та Momentum Contrast з використанням бібліотек мови програмування Python.\pause
	\end{enumerate}

	Результати показали, що алгоритм Deep InfoMax дає кращі результати, в той час як Momentum Constras $-$ більш вигідний з точки зору часу та обчислювальних ресурсів.
\end{frame}

\begin{frame}
	\centering\LARGEДякую за увагу!
\end{frame}

\end{document}
