\anonsection{Реферат}
\label{sec:Refer}


\hspace*{26pt} У роботі проводиться розрахунок площі криволінійного паркану змінної висоти. Висота задається як функція декартових координат точки паркана.

Пошук алгоритмів розв’язання задачі виконаний на сайті \cite{GoogleWebPage}. Для обчислення площі використовуються криволінійні інтеграли 1-го роду \cite{Senchuk2006}. Чисельні приклади розв’язані у MATLAB \cite{Anufriev2005}. Використовувалася також сторінка \cite{IglinWebPage}.

\vspace{0.25cm}

Іл.\,\totfig. Табл.\,\tottab. Лист.\,\totalg. Бібліогр.:\,\totref\,назв.


