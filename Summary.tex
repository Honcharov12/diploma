%\anonsection{Висновки}
\anonsection{ВИСНОВКИ}
\label{sec:Summary}
\addcontentsline{toc}{section}{Висновки}

\hspace*{26pt} Виконання диплмоної роботи складалося з наступних етапів:

\begin{enumerate}
	\item проведення аналізу літературних джерел;
	\item засвоєння методів ARIMA та SSA для аналізу та прогнозування\newline \hspace*{-18mm}часових рядів;
	\item чисельне вирішення задач структурного аналізу та побудова\newline \hspace*{-18mm}прогнозу з використанням бібліотек мови програмування Python.
\end{enumerate}

Результати прогнозування показали, що для короткотривалих прогнозів краще пряцює метод ARIMA, в той час як для довготривалих $-$ SSA.