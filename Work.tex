%\section{Охорона праці і навколишнього середовища}
\section{ОХОРОНА ПРАЦІ І НАВКОЛИШНЬОГО СЕРЕДОВИЩА}

Темою дипломної бакалаврської роботи є «Структурний аналіз та прогнозування роботи тестирующей системи DOTS». В процесі роботи була розглянута необхідна теорія, а також був створений програмний продукт. Важиливим аспектом роботи, що забезпечується застосуванням дисципліни охорони праці, є збереження здоров’я розробника ПЗ, підтримання ефективності та надійності його праці на належному рівні.

\subsection{Загальні питання охорони праці}

Закон України «Про охорону праці» [15] визначає основні положення щодо реалізації конституційного права працівників на охорону їх життя і здоров’я у процесі трудової діяльності, на належні, безпечні і здорові умови праці, регулює за участю відповідних органів державної влади відносини між роботодавцем і працівником з питань безпеки, гігієни праці та виробничого середовища і встановлює єдиний порядок організації охорони праці в Україні.

Професійні захворювання мають різний характер відповідно області, де працює людина. Розробка автоматизованих систем управління, створення ЕОМ полегшують і прискорюють виконання роботи. Але слід враховувати, що для запобігання отримання професійних захворювань час роботи за ЕОМ необхідно обмежувати, а саму роботу проводити на правильно організованому робочому місці.

Крім того, важливим аспектом є виконання задач Закону України «»Про охорону навколишнього природного середовища» [16], адже будь-яка виробнича діяльність впливає на навколишнє середовище.

\subsection{Структура управління охорони праці на підприємстві}

На підприємстві, де виконувалась дипломна робота впроваджена наступна структура: директор, заступник директора, технічний відділ, відділ реалізації продукції, бухгалтерія. Сукупна кількість співробітників складає 60 чоловік на 4 відділа. Є інженер з охорони праці. До переліку найважливіших і першочергових обов`язків інженера з охорони праці належать:

\begin{enumerate}
	\item атестація робочих місць;
	\item атестація робочого місця;
	\item атестація робочих.
\end{enumerate}

Система управління охороною праці (СУОП) — це сукупність органів управління підприємством, які на підставі комплексу нормативної документації проводять цілеспрямовану, планомірну діяльність щодо здійснення завдань і функцій управління з метою забезпечення здорових, безпечних і високопродуктивних умов праці.% Головна мета управління охороною праці є створення здорових, безпечних і високопродуктивних умов праці, покращення виробничого побуту, запобігання травматизму і профзахворюванням.

\subsection{Загальна характеристика приміщення та робочого місця}

Розглянемо характеристики робочого місця, на якому виконувалась дипломна робота, що наведені у табл. \ref{tab:charumov} та \ref{tab:charrobit}. Необхідно звернути увагу на можливі ризики задля забезпечення безпечних умов праці.

\begin{table}[h]
	\captionstyle{ \raggedright}
	\caption{Загальна характеристика умов праці}\label{tab:charumov}
	%\centering
	\begin{tabular}{|m{0.3\textwidth}|m{0.3\textwidth}|m{0.3\textwidth}|}
		\hline
		Шкідливі та небезпечні фактори & Джерела виникнення небезпек & Технічна характеристика робочого місця \\
		\hlinewd{2pt}
		Електрична напруга вище 127 В; шум; випромінювання – електромагнітні, радіаційні, теплові; пил; статична електрика; іонізація повітря; пожежна небезпека у приміщенні; неякісне освітлення &
		Вентиляція; 4-ПЕОМ; папір; світильники (лампи) &
		Розміри приміщення, м: довжина – 10; ширина – 5; висота – 3. Кількість працюючих – 6 \\
		\hline
	\end{tabular}
\end{table}

\newpage

\begin{table}[h]
	\captionstyle{ \raggedright}
	\caption{Загальна характеристика робіт, що виконуються}\label{tab:charrobit}
	%\centering
	\begin{tabular}{|m{0.2\textwidth}|m{0.3\textwidth}|m{0.4\textwidth}|}
		\hline
		Категорія за важкістю робіт & Показники напруженості трудового процесу & Ступінь відповідальності за результат своєї діяльності \\
		\hlinewd{2pt}
		Легка Іа & ступінь ризику для власного життя – виключений; ступінь відповідальності за безпеку інших осіб – виключений & значущість помилки –  допустимий (напруженість праці середнього ступеня); вимагає додаткових зусиль з боку керівництва; спостереження за екраном (2 - 3 годин на зміну) \\
		\hline
	\end{tabular}
\end{table}

\subsection{Метеорологічні параметри робочої зони}

Робота на персональній ЕОМ та розробка програмного продукту супроводжуються незначним фізичним напруженням, енерговитрати організму не перевищують 139 Вт, тому така робота, відповідно до ДСН 3.3.6.042-99 [17],  відноситься до категорії Iа – легка фізична робота.

Враховуючи високе нервово-емоційне напруження користувачів ЕОМ, згідно з ДСН 3.3.6.042-99 [17] у приміщеннях необхідно встановити оптимальні параметри мікроклімату. Під оптимальними параметрами мікроклімату розуміють такі параметри, які при тривалому і систематичному впливі на людину забезпечують збереження нормального функціонального і теплового стану організму без напруги реакцій терморегуляції, створюють відчуття теплового комфорту та є передумовою збереження високого рівня працездатності. Оптимальні параметри мікроклімату наведені в табл. \ref{tab:optclim}.

\begin{table}[h]
	\captionstyle{ \raggedright}
	\caption{Оптимальні параметри мікроклімату в приміщенні з ЕОМ}\label{tab:optclim}
	%\centering
	\begin{tabular}{|m{0.13\textwidth}|m{0.3\textwidth}|m{0.16\textwidth}|m{0.15\textwidth}|m{0.16\textwidth}|}
		\hline
		Період року & Категорія виконуваних робіт за енерговитратами & Температура, $^{\degree}$С & Відносна вологість, \% & Швидкість руху повітря, м/с \\
		\hlinewd{2pt}
		Холодний & Легка Іа & 22-24 & 40-60 & 0,1 \\
		\hline
		Теплий & Легка Іа & 23-25 & 40-60 & 0,1 \\
		\hline
	\end{tabular}
\end{table}

Згідно з ДБН В.2.5-67-2013 [18], у теплий період року використовуються кондиціонери, що автоматично підтримують необхідні оптимальні параметри температури, незалежно від зовнішніх умов, а також проводиться природне провітрювання приміщення; в холодний період – проводиться опалення від центральної тепломережі і природна вентиляція.

\subsection{Освітлення приміщення}

Згідно з ДБН В.2.5-28-2018 [19] для об’єктів, пов’язаних з освітленням, відповідно до розміру об’єкта та характеристики зорової роботи визначені нормативні характеристики, які відображенні у табл. \ref{tab:normosv}.

\begin{table}[h]
	\captionstyle{ \raggedright}
	\caption{Нормативні параметри освітлення для роботи в ЕОМ}\label{tab:normosv}
	%\centering
	\begin{tabular}{|m{0.11\textwidth}|m{0.1\textwidth}|m{0.1\textwidth}|m{0.1\textwidth}|m{0.1\textwidth}|m{0.11\textwidth}|m{0.1\textwidth}|m{0.1\textwidth}|}
		\hline
		\parbox[t]{0.11\textwidth}{Харак-\\теристи-\\ка зорової роботи} & \parbox[t]{0.1\textwidth}{Наймен-\\ший розмір об’єкта розпізнавання, мм} & Розряд зорової роботи & \parbox[t]{0.1\textwidth}{Підроз-\\ряд зорової роботи} & Контраст об’єкта розпізнавання & \parbox[t]{0.11\textwidth}{Харак-\\теристи-\\ка фону} & \parbox[t]{0.1\textwidth}{Освітле-\\ність при штучному освітленні, лк} & КПО, $\text{е}_{\text{н}}$, при суміщеному освітленні, \% \\
		\hlinewd{2pt}
		Дуже високої точності & Від 0,15 до 0,3 & ІІ & в & Середній & Середній & 500 & 1,5 \\
		\hline
	\end{tabular}
\end{table}

Коефіцієнт природного освітлення (КПО) – процентне відношення природної освітленості у будь-якій точці в середині приміщення до одночасно виміряної на тому ж рівні освітленості зовнішньої горизонтальної площини рівномірно розсіяним (дифузійним) усього небосхилу:

\begin{equation}\label{eq:kpo}
\text{КПО} = \text{е}_{\text{н}} = \frac{E_{\text{в}}}{E_{\text{з}}} \cdot 100 \%.
\end{equation}

Коефіцієнт природного освітлення (КПО) показує, яку частину зовнішнього дифузійного світла небосхилу у процентах становить освітлення в певній точці на робочій поверхні всередині приміщення.

Для приміщень з одностороннім бічним освітленням нормується мінімальне значення КПО у точці, розташованій на відстані 1 м від стінки, найменш віддаленої від світлових прорізів, на перерізі вертикальної площини характерного розрізу приміщення та умовної робочої поверхні.

Для приміщень із двостороннім бічним освітленням нормується мінімальне значення КПО у точці посередині приміщення на перерізі вертикальної площини характерного розрізу приміщення та умовної робочої поверхні.

При верхньому або комбінованому освітленні нормується середнє значення КПО у точках, розташованих на перерізі вертикальної площини характерного розрізу приміщення та умовної робочої поверхні. При цьому перша та остання точки приближаються на відстані 1 м від поверхні стін або перегородок.

У разі комбінованого освітлення допускається розподіл приміщення на зони з бічним (прилеглі до зовнішніх стін з вікнами) та верхнім освітленням. Нормування та розрахунок природного освітлення у кожній зоні проводиться окремо.

Під час нормування природної освітленості визначається найменший розмір об’єкта розрізнення, відповідний йому розряд зорової роботи та нормований коефіцієнт природної освітленості.

В приміщенні, що розглядається, застосовують суміщене освітлення $–$ освітлення, при якому недостатнє за нормами природне освітлення доповнюється штучним. Мінімальна освітленість при цьому складає 500 лк. Штучне освітлення реалізується шляхом встановлення визначеної кількості ламп білого світла – ЛБ 80.

\newpage

\subsection{Шум та вібрація у робочому приміщенні}

Шум є одним з найбільш розповсюджених у виробництві шкідливих факторів. 

Джерелами шуму і вібрації є вентилятори системного блоку, накопичувач, розташовані в системному блоці комп’ютера, і принтер. Це може стати джерелом стресу і дискомфорту користувача, знижувати розумову працездатність, підвищувати втомлюваність, послаблювати увагу, сприяти появі головного болю тощо. Відповідно до ДСН 3.3.6. 037-99 [20] робочі місця у приміщеннях програмістів обчислювальних машин рівень шуму не повинен перевищувати 50 дБА. Відповідно до ДСН 3.3.6. 039-99 [21] рівень загальної вібрації для категорії 3, технологічного типу «в» не повинен перевищувати 75 дБ.

Як захист від шуму, який створюється вентиляторами системних блоків, використовується наступне:

\begin{enumerate}
	\item звукоізоляційний корпус;
	\item заміна вентилятора на більш якісний;
	\item використання звукопоглинаючих та звукоізолюючих засобів;
	\item мідні радіатори як альтернативу вентилятору;
	\item при монтажі кулерів замість гвинтів встановлювати гумові пробки,\newline \hspace*{-18mm}що дозволяють ізолювати вентилятор від корпуса.
\end{enumerate}

\subsection{Електробезпека у робочому приміщенні}

Сучасні приміщення та виробництво нерозривно пов’язане з використанням електроенергії. Тому ці приміщення є приміщеннями з підвищеною небезпекою ураження людей електричним струмом. Основними причинами ураження струмом є:

\begin{enumerate}
	\item випадкове доторкання до струмоведучих частин, що перебувають\newline \hspace*{-18mm}під напругою;
	\item поява напруги на металевих конструктивних частинах\newline \hspace*{-18mm}електрообладнання;
	\item поява напруги на відімкнених струмоведучих частинах в\newline \hspace*{-18mm}результаті;
	\item виникнення напруги кроку на ділянці землі, де перебуває людина,\newline \hspace*{-18mm} в результаті: замикання фази на землю; несправності у обладнанні\newline \hspace*{-18mm}захисного заземлення тощо.
\end{enumerate}

При виконанні роботи використовувався комп’ютер, який живиться з напругою 220 В від однієї фази трьохфазної чотирьохпровідної мережі з глухозаземленою нейтраллю.

Основними заходами захисту від ураження електричним струмом згідно з ПУЕ-2017 [22] є:

\begin{enumerate}
	\item забезпечення недоступності струмопровідних частин, що\newline \hspace*{-18mm}перебувають під напругою, для випадкового дотику;
	\item організація безпечної експлуатації електроустановок;
	\item компенсація ємнісної складової струму замикання на землю;
	\item застосування спеціальних засобів – переносних приладів і\newline \hspace*{-18mm}запобіжних пристроїв;
	\item відключення електроустаткування, що ремонтується, і вживання\newline \hspace*{-18mm}заходів проти помилкового його зворотного включення або самовключення;
	\item проведення низки організаційних заходів (спеціальне навчання,\newline \hspace*{-18mm}атестація і переатестація осіб електротехнічного персоналу, інструктажі \newline \hspace*{-18mm}ощо).
\end{enumerate}

Для захисту від ураження електричним струмом для ЕОМ застосовується занулення – це навмисне електричне з’єднання з нульовим захисним провідником металевих не струмоведучих частин електроустановки, які можуть опинитися під напругою. Головне призначення захисного заземлення – знизити потенціал на корпусі електроустаткування до безпечного значення.

\subsection{Ергономічні вимоги до робочого місця}

Електромагнітне випромінювання шкідливо впливає на здоров’я людини. Згідно НПАОП 0.00-1.28-2010 [23], потужність поглиненої дози в повітрі за рахунок супутнього не використаного рентгенівського випромінювання не повинна перевищувати 100  на відстані 5 см від поверхні пристрою, під час роботи якого воно виникає. Забезпечення захисту оператора та досягнення нормованих рівнів випромінювань ЕОМ рекомендовано застосування екранних фільтрів, локальних світлофорів та інших засобів захисту, які пройшли випробування в акредитованих лабораторіях та отримали позитивний висновок державної санітарно-епідеміологічної експертизи.

Основними принципами захисту від впливу ЕОВ є:

\begin{enumerate}
	\item тривалість роботи за ЕОМ не повинна перевищувати 4 години на\newline \hspace*{-18mm}день при цьому виконувати перерви через кожні 2 години роботи;
	\item на одну ЕОМ повинно бути виділено не менше 6 $\text{м}^{2}$, відстань між\newline \hspace*{-18mm}сусідніми ЕОМ – 1,5м.
	\item внутрішнє екранування, що дозволяє суттєво знизити\newline \hspace*{-18mm}інтенсивність шкідливого опромінювання;
	\item для попередження, своєчасної діагностики та лікування здоров’я\newline \hspace*{-18mm}людини, що пов’язано з негативним впливом ЕОМ, користувачі повинні проходити попередні (під час прийому на роботу) і періодичні\newline \hspace*{-18mm}медичні огляди.
\end{enumerate}

\subsection{Пожежна безпека}

Пожежна безпека – стан об’єкта, при якому з встановленою ймовірністю виключається ймовірність виникнення і розвитку пожежі.

По категорії вибухопожежної та пожежної небезпеки, згідно з ДСТУ Б.В.1.1-36:2016 [24] приміщення, у якому виконувалась дипломна бакалаврська робота, відноситься до категорії В – пожежонебезпечне через присутність твердих спаленних матеріалів, таких як: робочі столи, ізоляція, папір та інше. Виходячи з категорії пожежонебезпеки і поверховості будинку, ступінь вогнестійкості будинку ІІ згідно з ДБН В.1.1-7-2016 [25].

Причинами, які можуть викликати пожежу, в розглянутому приміщенні є: несправність електричної проводки і приладів, коротке замикання електричного кола, перегрів апаратури, блискавка тощо.

Згідно з вимогами ДБН В.2.5-56-2015 [26] пожежна небезпека забезпечується наступними мірами:

\begin{enumerate}
	\item організаційними заходами щодо пожежної безпеки;
	\item системою протипожежного захисту;
	\item системою запобігання пожеж, яка передбачає запобігання\newline \hspace*{-18mm}утворення пального середовища і запобігання утворення в пальному\newline \hspace*{-18mm}середовищі джерел запалювання.
\end{enumerate}

При виборі засобів гасіння пожежі для забезпечення безпеки людини від можливості поразки електричним струмом у приміщенні передбачено використання вуглекислотних вогнегасників. Вогнегасник знаходиться на видному і легко доступному місці. Відстань від можливого осередку пожежі до місця розташування вогнегасника має бути не більше, ніж 30 м. При виникненні пожежі передбачена можливість повідомлення в пожежну охорону по телефону. Також необхідним заходом безпеки є евакуаційні виходи (не менше двох).

Організаційними заходами протипожежної профілактики є вступний інструктаж при надходженні на роботу, навчання виробничого персоналу протипожежним правилам, видання необхідних інструкцій і плакатів, засобів наочної агітації, наявність плану евакуації.

\subsection{Охорона навколишнього природного середовища}

Проблема охорони й оптимізації навколишнього природного середовища виникла як неминучий наслідок сучасної промислової революції.

Збільшення використання енергії призводить до порушення екологічної рівноваги природного середовища, яке складалася століттями.

Поряд з цим, підвищення технічної оснащеності підприємств, застосування нових матеріалів, конструкцій і процесів, збільшення швидкостей і потужностей виробничих машин впливають на навколишнє середовище.

Основними задачами Закону України «Про охорону навколишнього природного середовища» [16], прийнятого 25 червня 1991 року, є регулювання відносин в області охорони природи, використання і відтворення природних ресурсів, забезпечення екологічної безпеки, попередження і ліквідація наслідків негативного впливу на навколишнє середовище господарської й іншої діяльності людини, збереження природних ресурсів, генетичного фонду, ландшафтів і інших природних об’єктів.

При масовому використанні моніторів та комп’ютерів не можна не враховувати їхній вплив на навколишнє середовище на всіх стадіях – при виготовленні, експлуатації та після закінчення терміну служби.

Міжнародні екологічні стандарти, що діють на сьогоднішній день в усьому світі, визначають набір обмежень до технологій виробництва та матеріалів, які можуть використовуватися в конструкціях пристроїв. Так, за стандартом ТСО-95, вони не повинні містити фреонів (турбота про озоновий шар), полівінілхлориді, бромідів (як засобів захисту від загоряння).

У стандарті ТСО-99 закладене обмеження за кадмієм у світлочутливому шарі екрана дисплея та ртуті в батарейках; э чіткі вказівки відносно пластмас, лаків та покриттів, що використовуються. Відмовитися від свинцю в ЕЛТ поки неможливо. Поверхня кнопок не повинна містити хром, нікель та інші матеріали, які визивають алергічну реакцію. ГДК пилу дорівнює 0,15 $\text{мг}/\text{м}^{3}$, рекомендовано 0,075 $\text{мг}/\text{м}^{3}$; ГДК озону під час роботи лазерного принтеру $-$ 0,02 $\text{мг}/\text{м}^{3}$. Особливо жорсткі вимоги до повторно використовуваних матеріалів. 

Апарати, тара і документація повинні допускати нетоксичну вторинну переробку після закінчення терміну експлуатації. В ЕПТ міститься багато біоактивних речовин, що треба ураховувати під час утилізації.

Міжнародні стандарти, починаючи з ТСО-92, включають вимоги зниженого енергоспоживання та обмеження припустимих рівнів потужності, що споживаються у неактивних режимах.

Дотримання приведених нормативних параметрів небезпечних і шкідливих виробничих факторів дозволить забезпечити більш здорові і безпечні умови роботи користувача ЕОМ.
\newpage
\subsection{Висновки за розділом}

У нашій країні питання охорони навколишнього середовища, зниження негативних наслідків втручання людини у всі сфери її життєдіяльності –  це одна з найгостріших проблем. Вона потребує негайного вирішення, особливо на сучасному етапі розвитку та вдосконалення комп’ютерної, обчислювальної техніки.

Дотримання приведених нормативних параметрів небезпечних і шкідливих виробничих факторів дозволить забезпечити більш здорові і безпечні умови роботи користувача ЕОМ.

