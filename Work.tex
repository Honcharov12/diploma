%\section{Охорона праці і навколишнього середовища}
\section{ОХОРОНА ПРАЦІ І НАВКОЛИШНЬОГО СЕРЕДОВИЩА}

\subsection{Аналіз умов праці на робочому місці}

Розділ виконано для етапу розробки на ЕОМ системи створення програмного комплексу для аналізу роботи методів представлення візуальної інформації за допомогою методів самонавчання та contrastive learning.

Робота виконувалась на кафедрі «Комп’ютерної математики та аналізу даних» НТУ «ХПІ», яка розташована на другому поверсі семи поверхової будівлі.

Обладнання, приміщення і режим праці користувача повинні відповідати вимогам наступних нормативно–технічних документів:

\begin{enumerate}
	\item НПАОП 0.00-7.15-18.\hfill Вимоги\hfill щодо\hfill безпеки\hfill та\hfill захисту\hfill здоров’я\newline \hspace*{-20mm} працівників під час роботи з екранними пристроями;
	\item ДСанПіН 3.3.2.007-98.\hfill Державні\hfill санітарні\hfill правила\hfill і\hfill норми\newline \hspace*{-20mm} роботи\hfill з\hfill візуальними\hfill дисплейними\hfill терміналами\hfill електронно-\newline \hspace*{-18mm}обчислювальних машин;
	\item ДСТУ Б В.1.1-36:2016.\hfill Визначення\hfill категорій\hfill приміщень,\hfill будинків\newline \hspace*{-18mm} та зовнішніх установок за вибухопожежною та пожежною небезпекою;
	\item ДБН В.1.1-7-2016.\hfill Пожежна\hfill безпека\hfill об’єктів\hfill будівництва.\newline \hspace*{-18mm} Загальні вимоги;
	\item ДСН 3.3.6.042-99.\hfill Санітарні\hfill норми\hfill мікроклімату\hfill виробничих\newline \hspace*{-18mm} приміщень;
	\item ДБН В.2.5-67:2013. Опалення, вентиляція та кондиціонування;
	\item ДБН В.2.5-28-2018. Природне та штучне освітлення;
	\item ДСН 3.3.6.037-99.\hfill Санітарні\hfill норми\hfill виробничого\hfill шуму,\hfill ультразвуку\newline \hspace*{-18mm} та інфразвуку;
	\item ДСН 3.3.6.039-99.\hfill Санітарні\hfill норми\hfill виробничої\hfill загальної\newline \hspace*{-18mm} та локальної вібрації;
	\item ПУЕ. Правила улаштування електроустановок;
	\item НПАОП 40.1-1.32-01.\hfill Правила\hfill будови\hfill електроустановок.\newline \hspace*{-18mm} Електрообладнання спеціальних установок;
	\item ДСТУ ГОСТ 7237:2011.\hfill Електробезпека.\hfill Загальні\hfill вимоги\newline \hspace*{-18mm} та номенклатура видів захисту;
	\item ГОСТ 14254-2015 (IEC 6029:2013).\hfill Степени\hfill защиты,\newline \hspace*{-18mm} обеспечиваемые оболочками;
	\item НАПБ А.01.001-2014. Правила пожежної безпеки в Україні;
	\item ДСТУ БВ.2.5-38:2008.\hfill Інженерне\hfill обладнання\hfill будинків\hfill і\hfill споруд.\newline \hspace*{-18mm} Улаштування блискавка захисту будівель і споруд (IEC 62305:2006, NEQ);
	\item НАПБ Б.06.004-2005.\hfill Перелік\hfill однотипних\hfill за\hfill призначенням\newline \hspace*{-18mm} об’єктів,\hfill які\hfill підлягають\hfill обладнанню\hfill автоматичними\hfill установками\hfill пожежа\newline \hspace*{-18mm} гасіння та пожежної сигналізації;
	\item ГН 3.3.5-8.6.6.1-2014. \hfill Гігієнічна\hfill класифікація\hfill праці\hfill за\newline \hspace*{-18mm} показниками\hfill шкідливості\hfill та\hfill небезпечності\hfill факторів\hfill виробничого\newline \hspace*{-18mm} середовища, важкості та напруженості трудового процесу.
\end{enumerate}

\subsubsection{Загальна характеристика виробничого приміщення}

Загальна характеристика виробничого приміщення, в якому виконувалась робота, приведена у таблицях \ref{tab:work1-1}-\ref{tab:work1-3}.

\begin{table}[h]
	\captionstyle{ \raggedright}
	\caption{Загальна характеристика виробничого приміщення}\label{tab:work1-1}
	%\centering
	\begin{tabular}{|m{0.18\textwidth}|m{0.2\textwidth}|m{0.19\textwidth}|m{0.16\textwidth}|m{0.175\textwidth}|}
		\hline
		Найменування показника& Характеристика показника & Обґрунтування вибору значення показника & Документ, що регламентує цей показник & Примітка \\
		\hlinewd{2pt}
		1 Розміри приміщення (м); & $3.5 \times 4 \times 4.2$ & \multirow{2}{*}{\parbox[t]{0.18\textwidth}{на одне р. м. з ЕОМ не менше 6.0 $\text{м}^{2}$ площі }} & \multirow{2}{*}{\parbox[t]{0.18\textwidth}{ДСанПіН \\3.3.2-007-98}} & \multirow{2}{*}{\parbox[t]{0.18\textwidth}{7 $\text{м}^{2}$ на одне р. м. з ЕОМ, що відповідає нормі}} \\
		\cline{1-2}
		2 Кількість робочих місць (р.м.) & 2 & & & \\ %[-0.3em]
		\hline
	\end{tabular}
\end{table}

%З характеристикою освітлення можна ознайомитись у таблиці \ref{tab:work1-2}.

\newpage

\vspace{-1.5em}

\begin{table}[h!]
	\captionstyle{ \raggedright}
	\caption{Характеристика освітлення}\label{tab:work1-2}
	%\centering
	\begin{tabular}{|m{0.18\textwidth}|m{0.2\textwidth}|m{0.19\textwidth}|m{0.16\textwidth}|m{0.175\textwidth}|}
		\hline
		Найменування показника& Характеристика показника & Обґрунтування вибору значення показника & Документ, що регламентує цей показник & Примітка \\
		\hlinewd{2pt}
		\multirow{2}{*}{\parbox[t]{0.18\textwidth}{1 Природне освітлення, вікна виходять на схід}} & \multirow{2}{*}{\parbox[t]{0.18\textwidth}{бокове, одностороннє; азимут $90^{\degree}$}} & див. таблицю \ref{tab:work2-2} & ДБН В.2.5-28-18 &  \\
		\cline{3-5}
		& & КПО не нижче 1.5 \% & ДСанПіН 3.3.2-007-98 & для р. м. з ЕОМ \\ [2em]
		\hline
		\multirow{2}{*}{\parbox[t]{0.18\textwidth}{2 Штучне освітлення, кількість світильників N; джерела світла}} & \multirow{2}{*}{\parbox[t]{0.18\textwidth}{загальне рівномірне; N=4; люмінесцентні лампи}} & див. таблицю \ref{tab:work2-2} & ДБН В.2.5.-28-18 &  \\
		\cline{3-5}
		& & не нижче 300-500 лк & ДСанПіН 3.3.2-007-98 & для р. м. з ЕОМ \\ [3.5em]
		\hline
	\end{tabular}
\end{table}

%У таблиці \ref{tab:work1-4} з факторами пожежної безпеки.

%\newpage

\begin{table}[h!]
	\captionstyle{ \raggedright}
	\caption{Характеристика пожежної безпеки}\label{tab:work1-4}
	%\centering
	\begin{tabular}{|m{0.22\textwidth}|m{0.22\textwidth}|m{0.22\textwidth}|m{0.22\textwidth}|}
		\hline
		Найменування показника& Характеристика показника & Обґрунтування вибору значення показника & Документ, що регламентує цей показник \\
		\hlinewd{2pt}
		1 Категорія приміщення з вибухо–пожеже небезпеки & В & є тверді спаленні матеріали: папір, деревина тощо & ДСТУ Б В.1.1-36:2016 \\
		\hline
		2 Ступінь вогнестійкості будівельних конструкцій & ІІ & 7-и поверхова будівля; категорія В & ДБН В.1.1.7-2016 \\
		\hline
	\end{tabular}
\end{table}

%Характеристика електромережі представлена у таблиці \ref{tab:work1-3}.

\newpage

\vspace{-1.5em}

\begin{table}[h]
	\captionstyle{ \raggedright}
	\caption{Характеристика електронної мережі}\label{tab:work1-3}
	%\centering
	\begin{tabular}{|m{0.18\textwidth}|m{0.2\textwidth}|m{0.19\textwidth}|m{0.16\textwidth}|m{0.175\textwidth}|}
		\hline
		Найменування показника& Характеристика показника & Обґрунтування вибору значення показника & Документ, що регламентує цей показник & Примітка \\
		\hlinewd{2pt}
		1 Характеристика трифазної електричної мережі & чотири провідна з глухо заземленою нейтраллю напругою 380/220 В, частотою 50 Гц & довгі кабельні мережі великої ємності & ПУЕ & \\
		\hline
		2 Клас приміщення за ступенем небезпеки ураження електрострумом & з підвищеною небезпекою & є можливість одночасного дотику до металоконструкцій будівлі, що мають з’єднання з землею, та до металевих корпусів ЕОМ & ПУЕ & необхідно передбачити заходи безпеки згідно вимог ПУЕ \\
		\hline
	\end{tabular}
\end{table}


\subsubsection{Небезпечні та шкідливі фактори}

Небезпечним називають виробничий фактор, вплив якого на організм працюючого у відповідних умовах праці може призвести до травм або іншого раптового, різкого погіршення стану здоров’я.

Шкідливим називають виробничий фактор, вплив якого на організм працюючого може призводити в певних умовах до захворювання або зниження рівня працездатності.

Згідно з державним стандартом шкідливі і небезпечні фактори по природі їх впливу поділяються на фізичні, хімічні, біологічні та психофізіологічні.

Однією із основних цілей охорони праці на підприємстві є оцінка обстановки та характеристик трудового процесу в частині його впливу на здоров’я і життя працівника.

У таблиці \ref{tab:work2-1} надано перелік потенційних небезпечних факторів на робочому місці користувача ЕОМ з монітором на рідинних кристалах.

\begin{table}[hbt]
	\captionstyle{ \raggedright}
	\caption{Перелік потенційних небезпечних факторів на робочому місці користувача ЕОМ з ЖК монітором}\label{tab:work2-1}
	%\centering
	\begin{tabular}{|m{0.18\textwidth}|m{0.2\textwidth}|m{0.185\textwidth}|m{0.17\textwidth}|m{0.165\textwidth}|}
		\hline
		Назва фактора& Джерела виникнення& Умови роботи& Нормативні параметри, їх значення & Документ, що регламентує показник \\
		\hlinewd{2pt}
		Висока електрична напруга & мережа живлення устаткування & нормальний режим роботи & струм $Іh = 0,3$ мА; Напруга $U_\text{дот} < 2$ В & ПУЕ \\ [8em]
		\hline
	\end{tabular}
\end{table}

Таблиця \ref{tab:work2-2} показує типовий перелік потенційних шкідливих факторів.

\newpage

\vspace{-1.5em}

\begin{table}[hbt!]
	\captionstyle{ \raggedright}
	\caption{Перелік потенційних шкідливих факторів на робочому місці користувача ЕОМ з ЖК монітором}\label{tab:work2-2}
	%\centering
	\begin{tabular}{|m{0.18\textwidth}|m{0.2\textwidth}|m{0.185\textwidth}|m{0.17\textwidth}|m{0.165\textwidth}|}
		\hline
		Назва фактора& Джерела виникнення& Умови роботи& Нормативні параметри, їх значення & Документ, що регламентує показник \\
		\hlinewd{2pt}
		Несприятливе освітлення & стан систем природного та штучного освітлення & МРОР 0,3–0,5 мм; підрозряд «в», фон та контраст середні & КПО $D^{н}_{min} = 1,2 \, \%$; Освітленість $E_{min} = 300$ лк & ДБН В.2.5.–28–18 \\
		\hline
		\parbox[t]{0.18\textwidth}{Несприятли-\\вий мікроклімат: $t$, $\phi$, $v$} & стан систем опалення та вентиляції & категорія важкості робіт Іа; холодний період & $t$ $-$ $22–24 \, ^{\degree}C$; $\phi$ $-$ $40-60 \, \%$; $v$ $-$ не більше $0,1$ м/с & ДСН 3.3.6.042–99 \\
		\hline
		Підвищений рівень шуму & кондиціонери, кулери, системи освітлювання & творча діяльність, програмування & рівень звуку $L_{A} = 50 \, \text{дБА}$ & ДСН 3.3.6.037–99 \\
		\hline
		Вібрація & те ж саме & загальна технологічна, категорія 3, тип «в», умови комфорту & рівень віброшвидкості $L_{V} = 75 \, \text{дБ}$ & ДСН 3.3.6.039–99 \\
		\hline
		\parbox[t]{0.18\textwidth}{Психо–фізіо-\\логічна перенапруга} & \parbox[t]{0.2\textwidth}{монотонність праці, стати-\\чність і незручність пози} & & 1 та 2 клас умов праці для напруженості & \parbox[t]{0.165\textwidth}{ГН 3.3.5–8.6.6.1-\\2014} \\
		\hline
	\end{tabular}
\end{table}

\newpage

\subsection{Захист від шкідливого впливу факторів виробничого середовища}

Підтримка оптимальних параметрів мікроклімату в робочій зоні здійснюється відповідно вимог ДБН В.2.5-67:2013 за допомогою кондиціонеру, який регулює температуру повітря. Передбачена можливість природнього провітрювання приміщення. У холодний період року проводиться опалення від центральної тепломережі.

Згідно ДСН 3.3.6.042-99, у приміщеннях із значними площами засклених поверхонь передбачаються заходи щодо захисту:

\begin{enumerate}
	\item від\hfill перегрівання\hfill при\hfill попаданні\hfill прямих\hfill сонячних\hfill променів\hfill в\newline \hspace*{-18mm} теплий\hfill період\hfill року\hfill (орієнтація\hfill віконних\hfill прорізів\hfill схід\hfill – захід,\newline \hspace*{-18mm} улаштування лоджій, жалюзі, сонцезахисних плівок та інше);
	\item від\hfill радіаційного\hfill охолодження $-$ в\hfill зимовий\hfill (використання\hfill стін\newline \hspace*{-18mm} певної товщини, подвійних стекол).
\end{enumerate}

Робочі місця повинні бути віддалені від стін на відстань не менше $1 \, \text{м}$. 

Визначений в таблиці \ref{tab:work2-2} коефіцієнт природного освітлення реалізується через вікна визначеної площини, яка розраховується при проектуванні будівлі, а нормований показник штучного освітлення ($E_{min}$) реалізується шляхом встановлення визначеної кількості світильників і вибором потужності ламп в них.

Згідно вимог ДСанПіН 3.3.2.007-98, в разі штучного освітлення як джерела світла мають застосовуватись переважно люмінісцентні лампи типу ЛБ і світильники серії ЛПО3б із дзеркальними гратами, укомплектовані високочастотними пускорегулювальними апаратами (ВЧ ПРА).

Система загального освітлення має становити суцільні або преривчасті лінії світильників, розташовані збоку від робочих місць (переважно ліворуч), паралельно лінії зору працюючих. Слід передбачити обмеження прямої блискості від джерел природного та штучного освітлення та обмежувати відбиту блискість на робочих поверхнях. Необхідно чистити вікна і світильники не менше двох разів на рік та вчасно заміняти перегорілі лампи.

Заходи захисту від шуму та вібрації повинні відповідати вимогам ГОСТ 12.1.029-80 і ДСТУ ГОСТ 2656885:2009. Устаткування, що є джерелом шуму, слід розташовувати поза приміщенням для роботи з ЕОМ. Для забезпечення допустимих рівнів шуму у виробничих приміщеннях слід застосовувати засоби звукопоглинання, наприклад, перфоровані плити, панелі, підвісні стелі.

Як захист від шуму, який створюється вентиляторами системних блоків, використовується звукоізоляційний корпус. Вентилятор можна замінити на більш якісний або на мідні радіатори з водяним охолодженням. Крім того встановлюють перехідник з регулятором напруги і швидкості обертання процесорного кулеру, а при монтажі кулерів металеві гвинти заміняють гумовими пробками, що дозволяють ізолювати вентилятор від корпусу. Якщо принтер розташований на твердій поверхні, то для зменшення вібрації потрібно підстелити під нього щільний прогумований килимок.

\vspace{1.5em}

\subsection{Електробезпека}

Персональна ЕОМ є однофазним споживачем електроенергії, який живиться від трифазної чотирьох провідної мережі перемінного струму напругою 380/220 В частотою 50 Гц з глухо заземленою нейтраллю.

У разі випадкового дотику до струмопровідних частин, що знаходяться під напругою, або появі напруги дотику на металевих кожухах електроустаткування, наприклад, при пошкодженні ізоляції можливі нещасні випадки в результаті дії електричного струму.

Клас пожежа небезпечної зони приміщення, згідно ПУЕ-2017 та НПАОП 40.1-1.32-01 $-$ бо у приміщенні знаходяться тверді спаленні матеріали. 

Для приміщень з підвищеною небезпекою поразки людини електричним струмом ПУЕ передбачені конструктивні, схемно-конструктивні й експлуатаційні міри електробезпеки (ДСТУ ГОСТ 7237:2011).

	По-перше, експлуатаційні міри. Необхідно дотримуватися правил безпеки при роботі з високою напругою і використовувати наступні запобіжні заходи, що передбачені НПАОП 0.00-7.15-18: не підключати і не відключати кабелю, якщо обладнання знаходиться під напругою; технічне обслуговування і ремонтні роботи виконувати тільки при вимкнутому живленні в мережі; встановлювати у приміщенні загальний вимикач для відключення електроустаткування. Забороняється залишати працюючу апаратуру без нагляду.

	Також важливими є конструктивні заходи. ЕОМ відноситься до електроустановок до 1000 В закритого виконання, усі струмоведучі частини знаходяться в кожухах. Вибираємо ступінь захисту оболонки від зіткнення персоналу із струмоведучими частинами усередині захисного корпусу і від потрапляння води усередину корпусу ІP-44, де перша «4» $-$ захист від твердих тіл, розміром більш $1,0 \, \text{мм}$, друга «4» $-$ захист від бризків води (ГОСТ 14254-96).

	Як схемно-конструктивна міра безпеки застосовується подвійна ізоляція (для монітору), малі напруги до $42 \, \text{В}$, занулення (так як мережа живлення до $1000 \, \text{В}$ з глухо заземленою нейтраллю). Відповідно ДСТУ ГОСТ 7237:2011, занулення $-$ це навмисне електричне з’єднання металевих неструмоведучих частин комп’ютера, які у випадку аварії можуть виявитися під напругою, з нульовим захисним провідником.

Занулення використовується в чотири провідних трифазових мережах із заземленою нейтраллю напругою до $1000 \, \text{В}$.

Розрахунок занулення виконаний відповідно вимог методичних вказівок [12].

Мета розрахунку $-$ визначення такого перерізу нульового захисного провідника, при якому струм короткого замикання ($\text{І}_{\text{К}}$) у задане число разів ($\text{К}$) перевищить номінальний струм апарату захисту ($\text{І}^{\text{АЗ}}_{\text{НОМ}}$), що забезпечить селективне відключення споживача, тобто повинна виконуватися умова:

\begin{equation}\label{eq:work1}
	\text{І}_{\text{К}} \ge \text{К} \cdot \text{І}^{\text{АЗ}}_{\text{НОМ}}.
\end{equation}

\vspace{1.5em}

Вихідні дані для розрахунку:

\begin{enumerate}
	\item $P_{1}$ $-$ потужність \hfill однофазового\hfill споживача\hfill електроенергії,\newline \hspace*{-18mm} наприклад, електронно–обчислювальної машини ( ЕОМ), $325 \, \text{Вт}$;
	\item $P_{2}$ $-$ потужність\hfill усіх\hfill споживачів,\hfill які\hfill живляться\hfill від\hfill цього\newline \hspace*{-18mm} фазового\hfill провідника\hfill (кондиціонери,\hfill вентилятори,\hfill освітлювальні\hfill прилади,\newline \hspace*{-18mm} інші ЕОМ, принтери, тощо), $1 \, \text{кВт}$;
	\item $l_{1}$ $-$ довжина ділянки 1, $15 \, \text{м}$;
	\item $l_{2}$ $-$ довжина ділянки 2, $151 \, \text{м}$;
	\item $U_{\text{Л}}$ $-$ лінійна напруга; $U_{\text{Л}} = 380 \, \text{В}$; 
	\item $U_{\text{Ф}}$ $-$ фазова напруга; $U_{\text{Ф}} = 220 \, \text{В}$; 
	\item матеріал проводів $-$ мідь.
\end{enumerate}

Спосіб прокладки проводів на ділянці 1–2. На ділянці 2 кабель пролягає у землі, на першій $-$ в повітрі трубах.

\subsubsection{Вибір запобіжника}

Визначення струму $І_{1}$ в А, що живить електроустановку (ЕУ) потужністю $P_{1}$, Вт:

\begin{equation}
	I_{1} = \frac{P_{1}}{U_{\text{Ф}}},
\end{equation}
\[
	I_{1} = \frac{325}{220} = 1,47.
\]

\vspace{1.5em}

Визначення пускового $І_{\text{ПУСК}}$ ЕУ потужністю $P_{\text{1}}$, Вт:

\begin{equation}\label{eq:work3}
	І_{\text{ПУСК}} = \frac{\text{К}_{\text{П}}}{\text{К}_{\text{Т}}} I_{1},
\end{equation}

\noindent де $\text{К}_{\text{П}}$ $-$ коефіцієнт кратності пускового струму; \\
\hspace*{15pt} $\text{К}_{\text{Т}}$ $-$ коефіцієнт важкості пуску, залежить від часу пуску; \\
\hspace*{15pt} $\text{К}_{\text{Т}}$ $-$ 1,6; якщо час пуску понад 10 с $-$ тяжкий пуск; \\
\hspace*{15pt} $\text{К}_{\text{Т}}$ $-$ 2; якщо час пуску дорівнює 10 с $-$ середній пуск; \\
\hspace*{15pt} $\text{К}_{\text{Т}}$ $-$ 2,5; якщо час пуску дорівнює 5 с $-$ легкий пуск. \\

\vspace{1.5em}

Для ЕОМ: $\text{К}_{\text{П}} = 3; \, \text{К}_{\text{Т}} = 2,5$.

\[
	І_{\text{ПУСК}} = \frac{3}{2,5} \cdot 1,47 = 1,76.
\]

\vspace{1.5em}

\subsubsection{Вибір апарата захисту}

Номінальний струм, при якому спрацьовує апарат захисту, повинен перевищувати $\text{І}_{\text{ПУСК}}$, інакше апарат захисту буде спрацьовувати при кожному вмиканні електроустановки.

В нашому випадку $\text{І}^{\text{АЗ}}_{\text{НОМ}}$ дорівнює 4 А, тому обираємо запобіжник ВПШ 6–12.

\subsubsection{Визначення струму короткого замикання фази на корпус ЕУ}

Струм короткого замикання $I_{\text{К}}$ визначаємо за формулою \ref{eq:work4}:

\begin{equation}\label{eq:work4}
	I_{\text{К}} = \frac{U_{\text{Ф}}}{\frac{Z_{\text{ТР}}}{3} + Z_{\text{ПФН}}},
\end{equation}

\noindent де $Z_{\text{ТР}}$ $-$ повний опір трансформатора, Ом; \\
\hspace*{15pt} $Z_{\text{ПФН}}$ $-$ повний опір петлі фаза–нуль, Ом.

\vspace{1.5em}

\subsubsection{Визначення повного опору трансформатора}

Величина $Z_{\text{ТР}}$ залежить від потужності трансформатора, конструктивного виконання, напруги і схеми з'єднання його обмоток (зіркою або трикутником).

Потужність трансформатора визначається за умовою:

\begin{equation}
	N_{\text{ТР}} = 4 \cdot P_{2},
\end{equation}
\[
	N_{\text{ТР}} = 4 \cdot 1 = 4.
\]

\vspace{1.5em}

Отже,  $Z_{\text{ТР}} = 3,110 \, \text{Ом}$,  так як  ми обираємо  трансформатор  потужністю $N_{\text{ТР}} = 25 \, \text{кВт}$.

\subsubsection{Визначення повного опору петлі фаза–нуль}

Повний опір петлі фаза–нуль визначається по формулі:

\begin{equation}\label{eq:work6}
	Z_{\text{ПФН}} = \sqrt{(R_{\text{Ф}} + R_{\text{НЗ}})^{2} + X^{2}},
\end{equation}

\noindent де $R_{\text{Ф}}$ $-$ активний опір фазового захисного провідника, Ом; \\
\hspace*{15pt} $R_{\text{НЗ}}$ $-$ активний опір нульового захисного провідника, Ом; \\
\hspace*{15pt} $X$ $-$  індуктивний опір петлі фаза–нуль, Ом. 

\vspace{1.5em}

Індуктивний опір визначається за формулою:

\begin{equation}\label{eq:work7}
	X = X_{\text{Ф}} + X_{\text{НЗ}} + X_{\text{ВЗ}},
\end{equation}

\noindent де $X_{\text{Ф}}$ $-$ внутрішній індуктивний опір фазового провідника,  Ом; \\
\hspace*{15pt} $X_{\text{НЗ}}$ $-$ внутрішній індуктивний опір нульового провідника,  Ом; \\
\hspace*{15pt} $X_{\text{ВЗ}}$ $-$ зовнішній індуктивний опір, який зумовлено взаємоіндукцією \newline
\hspace*{15pt}петлі фаза–нуль, Ом.

\vspace{1.5em}

Для мідних та алюмінієвих провідників $Х_{\text{Ф}}$ та $Х_{\text{НЗ}}$ порівняно малі (близько 0,0156 Ом/км), тому ними можна знехтувати.

Зовнішній індуктивний опір $X_{\text{ВЗ}}$ залежить від відстані між проводами Д та їхнього діаметру $d$. Якщо нульові захисні проводи прокладають спільно з фазовими, значення Д  мале й порівняльне з діаметром $d$, тому опір $Х_{\text{ВЗ}}$ незначний (не більш 0,1 Ом/км) і ним можна знехтувати. Тоді:

\begin{equation}\label{eq:work8}
	Z_{\text{ПФН}} = R_{\text{Ф}} + R_{\text{НЗ}}.
\end{equation}

\vspace{1.5em}

Таким чином формула \ref{eq:work4} має наступний вигляд:

\begin{equation}\label{eq:work9}
	I_{\text{К}} = \frac{U_{\text{Ф}}}{\frac{Z_{\text{ТР}}}{3} + R_{\text{Ф}} + R_{\text{НЗ}}}.
\end{equation}

\vspace{1.5em}

Визначення активного опору фазового провідника:

\begin{equation}\label{eq:work10}
	R_{\text{Ф}} = R_{\text{Ф}_{1}} + R_{\text{Ф}_{2}},
\end{equation}

\noindent де $R_{\text{Ф}_{1}}$ $-$ опір фазового провідника на ділянці 1 та, Ом.\newline
\hspace*{15pt} $R_{\text{Ф}_{2}}$ $-$ опір фазового провідника на ділянкці 2 , Ом.

Для провідників з кольорових металів:

\begin{equation}\label{eq:work11}
	R_{\text{Ф}_{1}} = \rho \cdot \frac{l_{1}}{S_{\text{Ф}_{1}}},
\end{equation}

\begin{equation}\label{eq:work12}
	R_{\text{Ф}_{2}} = \rho \cdot \frac{l_{2}}{S_{\text{Ф}_{2}}},
\end{equation}

\noindent де $\rho$ $-$ питомий опір, $\frac{\text{Ом} \cdot \text{мм}^{2}}{\text{м}}$ , який дорівнює для міді 0,018; \\
\hspace*{15pt} $S_{\text{Ф}_{1}}$ $-$ переріз фазового провідника для ділянки 1, $\text{мм}^{2}$;\newline
\hspace*{15pt} $S_{\text{Ф}_{1}}$ $-$ переріз фазового провідника для ділянки 2, $\text{мм}^{2}$.

\vspace{1.5em}

Перерізи фазових проводів визначають при проектуванні електричної мережі струму, умов прокладання кабелю, матеріалу провідників тощо.

Для ділянки 1 вибираємо переріз, який відповідає струму $I_{1}$, для ділянки 2 $-$ струму $I_{2}$. В нашому випадку переріз для ділянки 1 дорівнює 1 $\text{мм}^{2}$, а для ділянки 2  дорівнює 1,5 $\text{мм}^{2}$. Тому:

\[
	R_{\text{Ф}_{1}} = \frac{0,018 \cdot 15}{1} = 0,27,
\]

\[
	R_{\text{Ф}_{2}} = \frac{0,018 \cdot 151}{1,5} = 1,81,
\]

\[
	R_{\text{Ф}} = 2,08.
\]

\vspace{1.5em}

Струм $І_{2}$ в А визначаємо за формулою:

\begin{equation}
	I_{2} = \frac{P_{2}}{U_{\text{Ф}}}
\end{equation}
\[
	I_{2} = \frac{1000}{220} = 4,54.
\]

\vspace{1.5em}

Визначення опору нульового захисного провідника:

\begin{equation}\label{eq:work14}
	R_{\text{НЗ}} = R_{\text{НЗ}_{1}} + R_{\text{НЗ}_{2}},
\end{equation}

\noindent де $R_{\text{НЗ}_{1}}$ $-$ опір нульового захисного провідника на ділянці 1, Ом;\newline
\hspace*{15pt} $R_{\text{НЗ}_{2}}$ $-$ опір нульового захисного провідника на ділянці 2, Ом.

Площа перерізу нульового робочого та нульового захисного провідників в груповій три провідній мережі повинна бути не менш площі фазового провідника, тобто:

\[
	S_{\text{НЗ}_{1}} = S_{\text{Ф}_{1}}, \, S_{\text{НЗ}_{2}} = S_{\text{Ф}_{2}}.
\]

\vspace{1.5em}

Відповідно: $R_{\text{НЗ}} = R_{\text{Ф}}$.

Отже, згідно з формулою \ref{eq:work9}:

\[
	I_{\text{К}} = \frac{U_{\text{Ф}}}{\frac{Z_{\text{ТР}}}{3} + R_{\text{Ф}} + R_{\text{НЗ}}} = \frac{220}{\frac{3,110}{3} + 1,35 + 1,35} = 58,87.
\]

\vspace{1.5em}

\subsection{Перевірка виконання умов надійності та ефективності роботи занулення}

Повинно виконуватися співвідношення \ref{eq:work1}:

\[
	\text{І}_{\text{К}} \ge \text{К} \cdot \text{І}^{\text{АЗ}}_{\text{НОМ}},
\]

\[
	58,87 \ge 3 \cdot 4,
\]

\noindent де $\text{К}$ $-$ запас надійності. Для запобіжників $\text{К} = 3$.

\vspace{1.5em}

Утрати напруги на ділянках1 та 2 не повинні перебільшувати 22 В:

\begin{equation}\label{eq:work15}
	U_{\text{П}_{1}} + U_{\text{П}_{2}} \le 22,
\end{equation}

\begin{equation}\label{eq:work16}
	U_{\text{П}_{1}} = I_{1} \cdot R_{\text{Ф}_{1}},
\end{equation}

\begin{equation}\label{eq:work17}
	U_{\text{П}_{2}} = I_{2} \cdot R_{\text{Ф}_{2}},
\end{equation}

\vspace{1.5em}

В нашому випадку $U_{\text{П}_{1}} = 0,39, \, U_{\text{П}_{2}} = 8,21.$ Обидві умови виконуються.

В результаті розрахунку у якості запобіжника було обрано запобіжник типу ВПШ 6–12, а перерізи фазового і нульового захисного провідників $-$ $1 \, \text{мм}^{2}$ та $1,5 \, \text{мм}^{2}$ відповідно на ділянках 1 та 2.

\vspace{1.5em}

\subsection{Пожежна безпека}

У зв’язку з поширенням комп’ютерної техніки, що може привести до загоряння, треба передбачати можливі наслідки і розробляти заходи щодо їх попередження. Причинами загоряння стають: несправність електричного обладнання, пошкодження ізоляції, коротке замикання кола струму, перегрів проводів, поганий контакт в місцях з’єднання; розряди статичної електрики, які особливо небезпечні в вибухонебезпечних приміщеннях, блискавка.

Пожежна безпека забезпечується наступними мірами:

\begin{enumerate}
	\item системою запобігання пожеж;
	\item системою пожежного захисту;
	\item організаційними заходами щодо пожежної безпеки.
\end{enumerate}

Система запобігання пожеж передбачає запобігання утворенню пального середовища і запобігання утворенню в пальному середовищі джерел запалювання.

Для зменшення небезпеки утворення в пальному середовищі джерел запалювання передбачено:

\begin{enumerate}
	\item використання\hfill електроустаткування,\hfill що\hfill відповідає\hfill класу\hfill пожеже\newline \hspace*{-18mm} небезпечної\hfill зони\hfill приміщення\hfill П-ІІа\hfill за\hfill ПУЕ\hfill та\hfill НПАОП 40.1-1.32-01:\newline \hspace*{-18mm} ступінь\hfill захисту\hfill електроапаратури\hfill не\hfill менш\hfill ІP-44,\hfill ступінь\hfill захисту\newline \hspace*{-18mm} світильників ІР-2Х;
	\item забезпечення\hfill захисту\hfill від\hfill короткого\hfill замикання\hfill (контроль і\newline \hspace*{-18mm} профілактика ізоляції, використання запобіжників);
	\item вибір\hfill перетину\hfill провідників\hfill по\hfill максимально\hfill допустимому\newline \hspace*{-18mm} нагріванню;
	\item будівлі,\hfill в\hfill яких\hfill встановлено\hfill обладнання\hfill інформаційних\hfill технологій\newline \hspace*{-18mm} чи\hfill будь-яке\hfill інше\hfill електронне\hfill обладнання,\hfill чутливе\hfill до\hfill атмосферних\newline \hspace*{-18mm} перешкод,\hfill незалежно\hfill від\hfill кількості\hfill уражень\hfill об’єктів\hfill за\hfill рік\hfill потребує\hfill І\hfill або\newline \hspace*{-18mm} ІІ рівня блискавка захисту (ДСТУ БВ.2.5-38:2008).
\end{enumerate}

Система протипожежного захисту призначена для локалізації та гасіння пожежі. При виборі засобів гасіння пожежі для забезпечення безпеки людини від можливості поразки електричним струмом у приміщенні відповідно вимог НАПБ А.01.001-2014 передбачено використання вуглекислотних вогнегасникiв ВВК-5. Вогнегасник знаходиться на видному і легко доступному місці. При виникненні пожежі передбачені можливості аварійного відключення апаратури і комунікацій та повідомлення в пожежну охорону по телефону. У якості сповіщувачів використовуються система автоматичної пожежної сигналізації відповідно вимог НАПБ Б.06.004-2005. Ступінь вогнестійкості будинку ІІ, що відповідає вимогам НПАОП 0.00-1.28-2010, згідно яких комп’ютери повинно розташовувати в будівлях не нижче ІІ ступеню (ДБН В.1.1-7-2016). У приміщенні є два незалежних виходи для евакуації людей під час пожежі.

Організаційними заходами протипожежної профілактики є вступний інструктаж при надходженні на роботу, навчання виробничого персоналу протипожежним правилам, видання необхідних інструкцій і плакатів, засобів наочної агітації, наявність плану евакуації.

\vspace{1.5em}

\subsection{Охорона навколишнього середовища}

У зв’язку з прискореним темпом промислової революції виникли проблеми пов’язані з охороною та оптимізацією оточуючого природнього середовища.

Як зазначено у Законі України «Про охорону навколишнього природного середовища», прийнятого 25 червня 1991 року, основними задачами є регулювання відносин в області охорони природи, використання і відтворення природних ресурсів, забезпечення екологічної безпеки, попередження і ліквідація наслідків негативного впливу на навколишнє середовище господарської й іншої діяльності людини, збереження природних ресурсів, генетичного фонду, ландшафтів і інших природних об'єктів. 

При масовому використанні моніторів та комп’ютерів не можна не враховувати їхній вплив на навколишнє середовище на всіх стадіях $-$ при виготовленні, експлуатації та після закінчення терміну служби.

Міжнародні екологічні стандарти, що діють на сьогоднішній день в усьому світі, визначають набір обмежень до технологій виробництва та матеріалів, які можуть використовуватися в конструкціях пристроїв. Так, за стандартом ТСО-95, вони не повинні містити фреонів (турбота про озоновий шар), полівінілхлориді, бромідів (як засобів захисту від загоряння).

У стандарті ТСО-99 закладене обмеження за кадмієм у світлочутливому шарі екрана дисплея та ртуті в батарейках; э чіткі вказівки відносно пластмас, лаків та покриттів, що використовуються. Поверхня кнопок не повинна містити хром, нікель та інші матеріали, які визивають алергічну реакцію. ГДК пилу дорівнює $0,15 \, \frac{\text{мг}}{\text{м}^{3}}$, рекомендовано $0,075 \, \frac{\text{мг}}{\text{м}^{3}}$; ГДК озону під час роботи   лазерного принтеру  $-$ $0,02 \, \frac{\text{мг}}{\text{м}^{3}}$. Особливо жорсткі вимоги до повторно використовуваних матеріалів. Міжнародні стандарти, починаючи з ТСО-92, включають вимоги зниженого енергоспоживання та обмеження припустимих рівнів потужності, що споживаються у неактивних режимах.

Дотримання приведених нормативних параметрів небезпечних і шкідливих виробничих факторів дозволить забезпечити більш здорові і безпечні умови роботи користувача ЕОМ.

\vspace{1.5em}

\subsection{Висновки за розділом}

Дотримання розглянутих нормативних положень, зокрема наведених у таблицях \ref{tab:work1-1}$-$\ref{tab:work2-2}, забезпечує працездатність людей, які виконують роботи в розглянутому приміщенні протягом робочого дня.
