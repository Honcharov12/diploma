%\section{Цивільний захист}
\section{ЦИВІЛЬНИЙ ЗАХИСТ}

Цивільний захист $-$ це функція держави, спрямована на захист населення, території, навколишнього природного середовища та майна від надзвичайних ситуацій шляхом запобігання таких ситуацій, ліквідації їх наслідків та надання допомоги постраждалим в мирний час та в особливий період [13].

У даному розділі дипломної роботи розглянуті питання щодо концепції оповіщення населення в умовах надзвичайної ситуації (НС).

Актуальність проблеми оповіщення населення зумовлена тенденціями зростання втрат людей і шкоди територіям у результаті небезпечних природних явищ і катастроф. Ризик надзвичайних ситуацій техногенного і природного характеру постійно зростає

Рівень національної безпеки не може бути достатнім, якщо в загальнодержавному масштабі не буде вирішено завдання захисту населення, об'єктів економіки і національного надбання від надзвичайних ситуацій техногенного і природного характеру [14].

Основними завданнями захисту населення і території від надзвичайних ситуацій техногенного і природного характеру є:

\begin{enumerate}
	\item здійснення \hfill комплексу \hfill заходів\hfill щодо\hfill запобігання\hfill і\hfill реагування \newline \hspace*{-20mm} на надзвичайні ситуації техногенного і природного характеру;
	\item забезпечення\hfill готовності\hfill і\hfill контролю\hfill за\hfill станом\hfill готовності\hfill до\newline \hspace*{-20mm} дій\hfill і\hfill взаємодії\hfill органів\hfill управління\hfill в\hfill цій\hfill сфері,\hfill сил\hfill і\hfill засобів,\hfill призначених\newline \hspace*{-20mm} для\hfill запобігання\hfill надзвичайних\hfill ситуацій\hfill техногенного\hfill і\hfill природного\newline \hspace*{-20mm} характеру і реагування на них
\end{enumerate}

\vspace{1.5em}

\subsection{Оповіщення населення}

Серед комплексу заходів з захисту населення за надзвичайних умов важливе місце посідає організація своєчасного інформування та оповіщення, які покладаються на органи цивільної оборони і є невід’ємним елементом усієї системи заходів [15].

Центральні та місцеві органи влади зобов’язані надавати населенню через засоби масової інформації оперативну і достовірну інформацію про стан захисту населення від НС, методи та способи їх захисту, вжиття заходів щодо забезпечення безпеки [16].

Оповіщення про загрозу виникнення НС і постійне інформування населення про них забезпечуються шляхом:

\begin{enumerate}
	\item завчасного\hfill створення\hfill і\hfill підтримки\hfill у\hfill постійній\hfill готовності\newline \hspace*{-20mm} загальнодержавної\hfill і\hfill територіальних\hfill автоматизованих\hfill систем\hfill центрального\newline \hspace*{-20mm} оповіщення населення;
	\item організаційно-технічного\hfill з’єднання\hfill територіальних\newline \hspace*{-20mm} систем\hfill центрального\hfill оповіщення\hfill і\hfill систем\hfill оповіщення\hfill на\hfill об’єктах\newline \hspace*{-20mm} господарювання;
	\item завчасного\hfill створення\hfill та\hfill організації\hfill технічного\hfill з’єднання\hfill з\newline \hspace*{-20mm} системами\hfill спостереження\hfill і\hfill контролю\hfill постійно\hfill діючих\hfill локальних\newline \hspace*{-20mm} систем\hfill оповіщення\hfill та\hfill інформування\hfill населення\hfill в\hfill зонах\hfill катастрофічного\newline \hspace*{-20mm} затоплення,\hfill районах\hfill розміщення\hfill радіаційних,\hfill хімічних\hfill підприємств,\hfill інших\newline \hspace*{-20mm} об’єктів підвищеної небезпеки;
	\item центрального\hfill використання\hfill загальнодержавних\hfill і\hfill галузевих\newline \hspace*{-20mm} систем\hfill зв’язку:\hfill радіо,\hfill провідного,\hfill телевізійного\hfill оповіщення,\newline \hspace*{-20mm} радіотрансляційних мереж та інших технічних засобів передачі інформації.
\end{enumerate}

Оповіщення організовують засобами радіо та телебачення. Для того, щоб населення своєчасно увімкнуло засоби оповіщення, використовують сигнали транспортних засобів, а також переривисті гудки підприємств.

Завивання сирен, переривисті гудки підприємств та сигнали транспортних засобів означають попереджувальний сигнал «Увага всім!». Той, хто почув цей сигнал, повинен негайно увімкнути теле- чи радіоприймачі та прослухати екстрене повідомлення місцевих органів влади чи управління з НС та цивільного захисту населення. Усі подальші дії визначаються їхніми вказівками.

Для\hfill своєчасного\hfill попередження\hfill населення\hfill введені\hfill сигнали\newline попередження населення у мирний і воєнний час [15].

Сигнал «Увага всім!» повідомляє населення про надзвичайну обстановку в мирний час і на випадок загрози нападу противника у воєнний час. Сигнал подається органами цивільного захисту за допомогою сирени і виробничих гудків. Тривалі гудки означають попереджувальний сигнал.

\subsubsection{Сигнали оповіщення в мирний час}

«Аварія на атомній електростанції». Повідомляються місце, час, масштаби аварії, інформація про радіаційну обстановку та дії населення. Якщо є загроза забруднення радіоактивними речовинами, необхідно провести герметизацію житлових, виробничих і складських приміщень. Провести заходи захисту від радіоактивних речовин сільськогосподарських тварин, кормів, урожаю, продуктів харчування та води. Прийняти йодні препарати. Надалі діяти відповідно до вказівок штабу органів цивільного захисту.

«Аварія на хімічно небезпечному об'єкті». Повідомляються місце, час, масштаби аварії, інформація про можливе хімічне зараження території, напрямок та швидкість можливого руху зараженого повітря, райони, яким загрожує небезпека. Дається інформація про поведінку населення. Залежно від обставин: залишатися на місці, у закритих житлових приміщеннях, на робочих місцях чи залишати їх і, застосувавши засоби індивідуального захисту, вирушити на місця збору для евакуації або в захисні споруди. Надалі діяти відповідно до вказівок штабу органів управління цивільного захисту.

«Землетрус». Подається повідомлення про загрозу землетрусу або його початок. Населення попереджається про необхідність відключити газ, воду, електроенергію, погасити вогонь у печах; повідомити сусідів про одержану інформацію; взяти необхідний одяг, документи, продукти харчування, вийти на вулицю і розміститися на відкритій місцевості на безпечній відстані від будинків, споруд, ліній електропередачі.

«Затоплення». Повідомляється район, в якому очікується затоплення в результаті підйому рівня води в річці чи аварії дамби.

Населення, яке проживає в даному районі, повинне взяти необхідні речі, документи, продукти харчування, воду, виключити електроенергію, відключити газ і зібратись у вказаному місці для евакуації. Повідомити сусідів про стихійне лихо і надалі слухати інформацію штабу органів управління цивільного захисту.

«Штормове попередження». Подається інформація для населення про посилення вітру. Населенню необхідно зачинити вікна, двері. Закрити в приміщеннях сільськогосподарських тварин. Повідомити сусідів. Населенню, по можливості, перейти в підвали, погреби.

\subsubsection{Сигнали оповіщення в воєнний час}

Сигнал «Повітряна тривога» подається для всього населення. Попереджається про небезпеку ураження противником даного району. По радіо передається текст: «Увага! Увага! Повітряна тривога! Повітряна тривога!» Одночасно сигнал дублюється сиренами, гудками підприємств і транспорту. Тривалість сигналу 2—3 хв.

При цьому сигналі об'єкти припиняють роботу, транспорт зупиняється і все населення укривається в захисних спорудах. Робітники і службовці припиняють роботу відповідно до інструкції і вказівок адміністрації. Там, де неможливо через технологічний процес або через вимоги безпеки зупинити виробництво, залишаються чергові, для яких мають бути захисні споруди.

Сигнал може застати у будь-якому місці й будь-який час. В усіх випадках необхідно діяти швидко, але спокійно, впевнено, без паніки. Суворо дотримуватися правил поведінки, вказівок органів цивільного захисту.

Сигнал «Відбій повітряної тривоги». Органами цивільного захисту через радіотрансляційну мережу передається текст: «Увага! Увага! Громадяни! Відбій повітряної тривоги!». За цим сигналом населення залишає захисні споруди і повертається на свої робочі місця і в житла.

Сигнал «Радіаційна небезпека» подається в населених пунктах і в районах, в напрямку яких рухається радіоактивна хмара, що утворилася від вибуху ядерного боєприпасу.

Почувши цей сигнал, необхідно з індивідуальної аптечки ЛІ-2 прийняти шість таблеток радіозахисного препарату № 1 із гнізда 4, надіти респіратор, протипилову пов'язку, ватно-марлеву маску або протигаз, взяти запас продуктів, документи, медикаменти, предмети першої потреби і направитися у сховище або ПРУ.

Сигнал\hfill «Хімічна тривога»\hfill подається\hfill у\hfill разі\hfill загрози\hfill або\newline безпосереднього виявлення хімічного або бактеріологічного нападу (зараження). При цьому сигналі необхідно прийняти з індивідуальної аптечки АІ-2 одну таблетку препарату при отруєнні фосфорорганічними речовинами з пенала з гнізда 2 або п'ять таблеток протибактеріального препарату № 1 із гнізда 5, швидко надіти протигаз, а за необхідності $-$ і засоби захисту шкіри, якщо можливо, та укритися в захисних спорудах. Якщо таких поблизу немає, то від ураження аерозолями отруйних речовин і бактеріальних засобів можна сховатися в житлових чи виробничих приміщеннях.

При застосуванні противником біологічної зброї населенню буде подана інформація про наступні дії.

Успіх захисту населення залежатиме від дисциплінованості, своєчасної і правильної поведінки, суворого дотримання рекомендацій і вимог органів цивільного захисту.

\subsubsection{Заходи протирадіаційного та протихімічного захисту}

Протирадіаційний та протихімічний захист (ПР та ПХЗ) $-$ це комплекс заходів ЦЗ, які направлені на запобігання чи послаблення дії іонізуючого опромінення, сильнодіючих та отруйних речовин

ПР та ПХЗ включають такі заходи:

\begin{enumerate}
	\item виявлення та оцінка радіаційної та хімічної обстановки;
	\item розробка та введення в дію режимів радіаційного захисту;
	\item організація та проведення дозиметричного та хімічного контролю;
	\item способи\hfill захисту\hfill населення\hfill при\hfill радіоактивному\hfill та\hfill хімічному\newline \hspace*{-20mm} забрудненні;
	\item забезпечення населення та формувань ЦЗ засобами ПР та ПХЗ;
	\item ліквідація\hfill наслідків\hfill зараження,\hfill спеціальна\hfill санітарна\hfill обробка,\newline \hspace*{-20mm} знезаражування\hfill місцевості\hfill та\hfill будівель\hfill тощо.\hfill ПР\hfill та\hfill ПХЗ\hfill організовують\newline \hspace*{-20mm} завчасно начальники ЦЗ об’єктів і командири формувань.
\end{enumerate}

\newpage

\subsection{Висновки за розділом}

Таким чином, за допомогою правильно розроблених та впроваджених в життя заходів із оповіщення населення в умовах НС, в разі їх виникнення, можливо мінімізувати ризики для життя та здоров’я персоналу та відвідувачів цих підприємств і розміри заподіяної шкоди. 
