%\section{Економічне обґрунтування}
\section{ЕКОНОМІЧНЕ ОБҐРУНТУВАННЯ}

Темою дипломної бакалаврської роботи є «Розробка методів представлення візуальної інформації за допомогою методів самонавчання та contrastive learning». В процесі роботи була розглянута необхідна теорія, а також був створений програмний продукт. Важливою частиною дипломної роботи є економічне обгрунтування.

\subsection{Розрахунок кошторису витрат на проведення й впровадження результатів науково-дослідної роботи}
Виконання наукових дослiджень, а також впровадження результатiв НДР вимагає певних витрат, якi необхiдно розглядати як додатковi капiталовкладення. Витрати на проведення й впровадження результатiв НДР вiдносяться до виробничих витрат. 

Як правило, всi витрати документально оформляються у виглядi кошторису. Основними статтями кошторису витрат є заробiтна плата, нарахування на заробiтну плату, вартiсть електроенергiї (технологiчна й освiтлювальної), вартiсть оренди примiщення, амортизацiйнi вiдрахування на обчислювальну технiку, вартiсть впровадження й освоєння результатiв НДР i плановi накопичення.

\subsubsection{Розрахунок фонду заробітної плати виконавців}
Розрахунок фонду заробiтної плати виконавцiв проводиться виходячи зi штатного розкладу й зайнятостi виконавцiв у данiй НДР.
Виконавцями даної НДР є керівник дипломної роботи, консультанти частини економічного обґрунтування й частини охорони праці дипломної роботи, а також студент. Штатний розклад приведено у табл. \ref{tab:staff}.

\newpage

\begin{table}
	\captionstyle{ \raggedright}
	\caption{Штатний розклад}\label{tab:staff}
	\begin{tabular}{| p{0.20\textwidth} | p{0.12\textwidth} | p{0.12\textwidth} | p{0.14\textwidth} | p{0.14\textwidth} | p{0.14\textwidth} |}
		\hline
		Посада & Кількість виконавців & Час зайнятості, міс & Коефіціент трудової участі & Оклад на місяць, грн & Заробітна плата, $\text{Зп}_{\text{оклад}}$, грн \\
		\hlinewd{2pt}
		1 Керівник роботи, старший викладач & 1 & 4 & 0,075 & 7000 & 2100\\
		\hline
		2 Консультант частини економічного обгрунтування, професор & 1 & 4 & 0,005 & 10000 & 200\\
		\hline
		3 Консультант частини охорони праці, доцент & 1 & 4 & 0,005 & 8000 & 160\\
		\hline
		4 Консультант частини цивільного захисту, старший викладач & 1 & 4 & 0,005 & 7000 & 140\\
		\hline
		5 Виконавець, студент & 1 & 4 & 1 & 1600 & 6400\\
		\hline
	\end{tabular}
\end{table}

Заробітна плата виконавців НДР складається з основної заробітної плати й різних доплат до неї:

\begin{equation}\label{eq:zp}
\text{Зп} = \text{Зп}_{\text{осн}} + \text{Зп}_{\text{д}},
\end{equation}

\noindent де $\text{Зп}_{\text{осн}}$ $-$ основна заробітна плата; \newline 
\hspace*{15pt} $\text{Зп}_{\text{д}}$ $-$ доплати до заробітної плати.


\begin{equation}
\text{Зп}_{\text{осн}} = \text{Зп}_{\text{оклад}} + \text{Зп}_{\text{прем}},
\end{equation}

\noindent де $\text{Зп}_{\text{оклад}}$ $-$ розмір заробітної плати за штатним розкладом; \newline
\hspace*{15pt} $\text{Зп}_{\text{прем}}$ $-$ розмір премій. 

\begin{equation}
\text{Зп}_{\text{прем}} = \text{К}_{\text{прем}} \cdot \text{Зп}_{\text{оклад}},
\end{equation}

\noindent де $\text{К}_{\text{прем}}$ $-$ коефіціент преміювання, $\text{К}_{\text{прем}} = 0,1$;

\begin{equation}
\text{Зп}_{\text{д}} = \text{К}_{\text{д}} \cdot \text{Зп}_{\text{осн}},
\end{equation}

\noindent де $\text{К}_{\text{д}}$ $-$ коефіцієнт доплат заробітної плати, $\text{К}_{\text{д}} = 0,1$. 

\vspace{1.5em}

Розрахуємо заробітню плату виконавців НДР.

Керівник дипломної роботи:

\[
\text{Зп} = 2100 + 2100 \cdot 0,1 + 0,1 \cdot (2100 + 2100 \cdot 0,1) = 1220,80 \, \text{грн}.
\]

\vspace{1.5em}

Консультант частини економічного обгрунтування:

\[
\text{Зп} = 250 + 250 \cdot 0,12 + 0,09 \cdot (250 + 250 \cdot 0,12) = 305,20 \, \text{грн}.
\]

\vspace{1.5em}

Консультант частини охорони праці:

\[
\text{Зп} = 200 + 200 \cdot 0,12 + 0,09 \cdot (200 + 200 \cdot 0,12) = 244,16 \, \text{грн}.
\]

\vspace{1.5em}

Консультант частини цивільного захисту:

\[
\text{Зп} = 200 + 200 \cdot 0,12 + 0,09 \cdot (200 + 200 \cdot 0,12) = 244,16 \, \text{грн}.
\]

\vspace{1.5em}

Фонд заробiтної плати виконавцiв складе:

\[
\text{Зп}_{\text{заг}} = 1220,80 + 305,20 + 244,16 + 7000,00 = 8770,16 \, \text{грн}.
\]

\vspace{1.5em}

Всього витрати на заробiтну плату склали 8770,16 грн.

\subsubsection{Відрахування на соціальне страхування}

Відрахування на соціальне страхування й інші відрахування розраховуються на підставі отриманого значення фонду заробітної плати:

\begin{equation}\label{eq:soc}
\text{Від} = \text{К}_{\text{від}} \cdot \text{Зп},
\end{equation}


\noindent де $\text{К}_{\text{від}}$ $-$ коефіцієнт нарахувань на фонд заробітної плати приймається в розмірі $0,362$.

\[
\text{Від} = 0,362 \cdot 1770,16 = 640,79 \, \text{грн}.
\]

\vspace{1.5em}

\subsubsection{Розрахунок технологічної електроенергії}

Розрахунок технологічної електроенергії проводиться виходячи із завантаження устаткування, що використовується під час проведення НДР (ЕОМ, принтер, сканер і ін.), по формулі (\ref{eq:texenergy}):

\begin{equation}\label{eq:texenergy}
\text{Е}_{\text{тех}} = P \sum_{i=1}^{N}\text{П}_{i}T_{i}
\end{equation}

\noindent де $P$ $-$ тариф на електроенергію, $P = 1,7808 \, \text{грн}/\text{кВт}$; \newline
\hspace*{15pt} $\text{П}_{i}$ $-$ споживана потужність $i$-ої одиниці встаткування, для комп'ютера $\text{П}_{i} = 0,3 \, \text{кВт}/\text{год}$;\newline
\hspace*{15pt} $T_{i}$ $-$ час роботи $i$-ої одиничі встаткування, $T_{1} = 336 \, \text{год}$.

\[
\text{Е}_{\text{тех}} = 1,7808 \cdot 0,3 \cdot 336 = 179,50 \, \text{грн}.
\]

\vspace{1.5em}

\subsubsection{Розрахунок електроенергії, що витрачається на освітлення}

Розрахунок електроенергії, що витрачається на освітлення, виконується виходячи з норм охорони праці по освітленню робочих місць та розраховується наступним чином:

\begin{equation}\label{eq:osv}
\text{Е}_{\text{осв}} = P \cdot N_{\text{л}} \cdot \text{П}_{\text{л}} \cdot T,
\end{equation}

\noindent де $P$ $-$ тариф на електроенергію, $P = 1,7808 \, \text{грн}/\text{кВт}$; \newline
\hspace*{15pt}$N_{\text{л}}$ $-$ кількість ламп, $N_{\text{л}} = 2$;\newline
\hspace*{15pt}$\text{П}_{\text{л}}$ $-$ споживана потужність однієї лампи, $\text{П}_{\text{л}} = 0,1 \, \text{кВт}/\text{год}$;\newline
\hspace*{15pt}$T$ $-$ час роботи ламп для освітлення, $T = 169 \, \text{год}$.

\[
\text{Е}_{\text{осв}} = 1,7808 \cdot 2 \cdot 0,1 \cdot 169 = 60,19 \, \text{грн}.
\]

\vspace{1.5em}

\subsubsection{Амортизаційні відрахування на устаткування}

Амортизацiйнi витрати розраховуються виходячи з формули (\ref{eq:amort}):

\begin{equation}\label{eq:amort}
A = \frac{a_{\text{ЕОМ}}}{12}\sum_{i=1}^{N}\text{ЗВ}_{i} \cdot T_{i},
\end{equation}

\noindent де $a_{\text{ЕОМ}}$ $-$ річна норма амортизації, прийнята в розмірі $ 26 \, \%$ залишкової вартості устаткування;\newline
\hspace*{23pt}$\text{ЗВ}_{i}$ $-$ залишкова вартість $i$-ої одиниці устаткування, $\text{ЗВ}_{1} = 4739,78 \, \text{грн}$;\newline
\hspace*{23pt}$T_{i}$ $-$ час використання $i$-ої одиниці устаткування $T_{1} = 3,5 \, \text{міс}$.

\[
A = \frac{0,26}{12} \cdot 4739,78 \cdot 3,5 = 359,43 \, \text{грн}.
\]

\vspace{1.5em}

\subsubsection{Вартість оренди приміщення}

Витрати на оренду примiщення розраховуються виходячи з формули (\ref{eq:orenda}):

\begin{equation}\label{eq:orenda}
\text{Д} = \text{К}_{\text{а}} \cdot S \cdot P \cdot T_{OP},
\end{equation}

\noindent де $\text{К}_{\text{а}}$ $-$ коефіцієнт, що враховує податок на майно, $\text{К}_{\text{а}} = 1,25$;\newline
\hspace*{19pt}$S$ $-$ площа приміщення, де проводилася НДР, $S = 10 \, \text{м}^{2}$;\newline
\hspace*{19pt}$P$ $-$ вартість оренди одного квадратного метра приміщення, $P = 195 \, \text{грн}/\text{міс}$;\newline
\hspace*{19pt}$T_{OP}$ $-$ строк оренди, $T_{OP} = 5 \, \text{міс}$.

\[
\text{Д} = 1,25 \cdot 10 \cdot 195 \cdot 5 = 12187,50 \, \text{грн}.
\]

\vspace{1.5em}

\subsubsection{Інші витрати}

Інші витрати (опалення, робота кондиціонера й ін.), згідно з формулою (\ref{eq:inshi}), приймаються в розмірі $7,5\%$ від вартості оренди приміщення. 

\begin{equation}\label{eq:inshi}
\text{З}_{\text{ін}} = \text{Д} \cdot \text{К}_{\text{ін}} = 12187,50 \cdot 0,075 = 914,06 \, \text{грн}. 
\end{equation}

\vspace{1.5em}

\subsubsection{Вартість впровадження й освоєння результатів НДР}

Результатом НДР є система, яка дослілджує та аналізує динаміку відправлень рішень до тестувальної системи. Для використання цієї системи необхідно впровадити та освоїти програми у аналітичних відділах.

При впровадженнi та освоєннi результатiв НДР необхiдно залучити хоча б одного асистента кафедри в науково дослiдному iнститутi з вiдповiдної спецiальностi. Заробiтна плата становить 5000 грн на мiсяць в середньому. На впровадження необхiдно не менше мiсяця. У пiдсумку вартiсть впровадження та освоєння результатiв НДР складе $5000$ грн.

\subsubsection{Витрати на проведення НДР}

Витрати на проведення НДР, згідно з формулою (\ref{eq:vitraty}), являють собою суму витрат по окремих статтях:

\begin{equation}\label{eq:vitraty}
\text{З} = \text{Зп} + \text{Від} + \text{Е}_{\text{тех}} + \text{Е}_{\text{осв}} + \text{А} + \text{Д} + \text{З}_{\text{ін}} + \text{В}_{\text{вп}},
\end{equation}

\noindent де $\text{З}_{\text{ін}}$ $-$ інші витрати; \newline
\hspace*{15pt} $ \text{В}_{\text{вп}}$ $-$ вартість впровадження й освоєння результатів НДР.

Таким чином, сукупнi витрати складають:

\begin{eqnarray*}
\text{З} = 8770,16 + 640,79 + 179,50 + 60,19 + 359,43 + \\ + 12187,50 + 914,06 + 5000  = 28111,63 \, \text{грн}.
\end{eqnarray*}

\vspace{1.5em}

\subsubsection{Планові накопичення}

Планові накопичення обираються в розмірі $30\%$ від витрат на проведення НДР та наведені у формулі (\ref{eq:plan}).

\begin{equation}\label{eq:plan}
\text{ПН} = 0,3 \cdot 28111,63 = 8433,48 \, \text{грн}.
\end{equation}

\vspace{1.5em}

\subsubsection{Кошторис витрат на проведення НДР}

Кошторис витрат на проведення НДР є сумою витрат на проведення НДР і планових накопичень. Результати розрахунку кошторису витрат представлені у табл. \ref{tab:sumNDR}.

\begin{table}[hbt]
	\captionstyle{ \raggedright}
	\caption{Кошторис витрат на проведення НДР}\label{tab:sumNDR}
	%\centering
	\begin{tabular}{|p{0.65\textwidth}|p{0.15\textwidth}|}
		\hline
		Стаття витрат & Сума, грн \\
		\hlinewd{2pt}
		1 Заробiтна плата & 8770,16 \\
		\hline
		2 Вiдрахування на соцiальне страхування & 640,79 \\
		\hline
		3 Технологiчна електроенергiя & 179,50 \\
		\hline
		4 Електроенергiя на освiтлення & 60,19 \\
		\hline
		5 Амортизацiйнi вiдрахування на устаткування & 359,43 \\
		\hline
		6 Вартiсть оренди примiщення & 12187,50 \\
		\hline
		7 Iншi витрати & 914,06 \\
		\hline
		8 Вартiсть впровадження та освоєння НДР & 5000 \\
		\hline
		9 Разом витрат & 28111,63 \\
		\hline
		10 Плановi накопичення & 8433,48 \\
		\hline
		11 Усього кошторис витрат на НДР & 36545,11 \\
		\hline
	\end{tabular}
\end{table}

\subsection{Висновки за розділом}

В процесі роботи над економічною частиною НДР було проведено ознайомлення з методикою складання кошторису витрат на НДР та подальший його розрахунок за основними статтями: витрати на заробiтну платню виконавцiв дипломної роботи, витрати на електроенергiю, амортизацiйнi вiдрахування, вiдрахування на соцiальне страхування, оренду примiщення i плановi накопичення. За результатами проведених розрахункiв кошторис склав $36545,11$ грн.
