%\section{Економічне обґрунтування}
\section{ЕКОНОМІЧНЕ ОБҐРУНТУВАННЯ}

Темою дипломної роботи є «Розробка методів представлення візуальної інформації за допомогою методів самонавчання та contrastive learning». В процесі роботи була розглянута необхідна теорія, а також був створений програмний продукт. Важливою частиною дипломної роботи є економічне обгрунтування.

\subsection{Розрахунок кошторису витрат на проведення й впровадження результатів науково-дослідної роботи}
Виконання наукових дослiджень, а також впровадження результатiв НДР вимагає певних витрат, якi необхiдно розглядати як додатковi капiталовкладення. Витрати на проведення й впровадження результатiв НДР вiдносяться до виробничих витрат. 

Як правило, всi витрати документально оформляються у виглядi кошторису. Основними статтями кошторису витрат є заробiтна плата, нарахування на заробiтну плату, вартiсть електроенергiї (технологiчна й освiтлювальної), вартiсть оренди примiщення, амортизацiйнi вiдрахування на обчислювальну технiку, вартiсть впровадження й освоєння результатiв НДР i плановi накопичення.

\subsubsection{Розрахунок фонду заробітної плати виконавців}
Розрахунок фонду заробiтної плати виконавцiв проводиться виходячи зi штатного розкладу й зайнятостi виконавцiв у данiй НДР.
Виконавцями даної НДР є керівник дипломної роботи, консультанти частини економічного обґрунтування, частини охорони праці дипломної роботи й частини цивільного захисту, а також інженер-математик. Штатний розклад приведено у табл. \ref{tab:staff}.

\begin{table}
	\captionstyle{ \raggedright}
	\caption{Штатний розклад}\label{tab:staff}
	\begin{tabular}{| p{0.20\textwidth} | p{0.12\textwidth} | p{0.12\textwidth} | p{0.14\textwidth} | p{0.14\textwidth} | p{0.14\textwidth} |}
		\hline
		Посада & Кількість виконавців & Час зайнятості, міс & Коефіціент трудової участі & Оклад на місяць, грн & Заробітна плата, $\text{Зп}_{\text{оклад}}$, грн \\
		\hlinewd{2pt}
		1 Керівник роботи, старший викладач & 1 & 4 & 0,075 & 7000 & 2100\\
		\hline
		2 Консультант частини економічного обгрунтування, професор & 1 & 4 & 0,005 & 10000 & 200\\
		\hline
		3 Консультант частини охорони праці, доцент & 1 & 4 & 0,005 & 8000 & 160\\
		\hline
		4 Консультант частини цивільного захисту, старший викладач & 1 & 4 & 0,005 & 7000 & 140\\
		\hline
		5 Виконавець, інженер-математик & 1 & 4 & 1 & 6000 & 24000\\
		\hline
	\end{tabular}
\end{table}

\newpage

Заробітна плата виконавців НДР складається з основної заробітної плати й різних доплат до неї:

\begin{equation}\label{eq:zp}
\text{Зп} = \text{Зп}_{\text{осн}} + \text{Зп}_{\text{д}},
\end{equation}

\noindent де $\text{Зп}_{\text{осн}}$ $-$ основна заробітна плата; \newline 
\hspace*{15pt} $\text{Зп}_{\text{д}}$ $-$ доплати до заробітної плати.


\begin{equation}
\text{Зп}_{\text{осн}} = \text{Зп}_{\text{оклад}} + \text{Зп}_{\text{прем}},
\end{equation}

\noindent де $\text{Зп}_{\text{оклад}}$ $-$ розмір заробітної плати за штатним розкладом; \newline
\hspace*{15pt} $\text{Зп}_{\text{прем}}$ $-$ розмір премій. 

\begin{equation}
\text{Зп}_{\text{прем}} = \text{К}_{\text{прем}} \cdot \text{Зп}_{\text{оклад}},
\end{equation}

\noindent де $\text{К}_{\text{прем}}$ $-$ коефіціент преміювання, $\text{К}_{\text{прем}} = 0,1$;

\begin{equation}
\text{Зп}_{\text{д}} = \text{К}_{\text{д}} \cdot \text{Зп}_{\text{осн}},
\end{equation}

\noindent де $\text{К}_{\text{д}}$ $-$ коефіцієнт доплат заробітної плати, $\text{К}_{\text{д}} = 0,1$. 

\vspace{1.5em}

Розрахуємо заробітню плату виконавців НДР.

Керівник дипломної роботи:

\[
\text{Зп} = 2100 + 2100 \cdot 0,1 + 0,1 \cdot (2100 + 2100 \cdot 0,1) = 2541,00 \, \text{грн}.
\]

\vspace{1.5em}

Консультант частини економічного обгрунтування:

\[
\text{Зп} = 200 + 200 \cdot 0,1 + 0,1 \cdot (200 + 200 \cdot 0,1) = 242,00 \, \text{грн}.
\]

\vspace{1.5em}

Консультант частини охорони праці:

\[
\text{Зп} = 160 + 160 \cdot 0,1 + 0,1 \cdot (160 + 160 \cdot 0,1) = 193,60 \, \text{грн}.
\]

\vspace{1.5em}

Консультант частини цивільного захисту:

\[
\text{Зп} = 140 + 140 \cdot 0,1 + 0,1 \cdot (140 + 140 \cdot 0,1) = 169,40 \, \text{грн}.
\]

\vspace{1.5em}

Консультант дипломної роботи:

\[
\text{Зп} = 24000 + 24000 \cdot 0,1 + 0,1 \cdot (24000 + 24000 \cdot 0,1) = 29040,00 \, \text{грн}.
\]

\vspace{1.5em}

Фонд заробiтної плати виконавцiв складе:

\[
\text{Зп}_{\text{заг}} = 2541,00 + 242,00 + 193,60 + 169,40 + 6400,00 = 32186,00 \, \text{грн}.
\]

\vspace{1.5em}

Всього витрати на заробiтну плату склали 32186,00 грн.

\subsubsection{Відрахування на соціальне страхування}

Відрахування на соціальне страхування й інші відрахування розраховуються на підставі отриманого значення фонду заробітної плати:

\begin{equation}\label{eq:soc}
\text{Від} = \text{К}_{\text{від}} \cdot \text{Зп},
\end{equation}


\noindent де $\text{К}_{\text{від}}$ $-$ коефіцієнт нарахувань на фонд заробітної плати приймається в розмірі $0,362$.

\[
\text{Від} = 0,22 \cdot 32186,00 = 7080,92 \, \text{грн}.
\]

\vspace{1.5em}

\subsubsection{Розрахунок технологічної електроенергії}

Розрахунок технологічної електроенергії проводиться виходячи із завантаження устаткування, що використовується під час проведення НДР (ЕОМ, принтер, сканер і ін.), по формулі (\ref{eq:texenergy}):

\begin{equation}\label{eq:texenergy}
\text{Е}_{\text{тех}} = P \sum_{i=1}^{N}\text{П}_{i}T_{i}
\end{equation}

\noindent де $P$ $-$ тариф на електроенергію, $P = 1,7808 \, \text{грн}/\text{кВт}$; \newline
\hspace*{15pt} $\text{П}_{i}$ $-$ споживана потужність $i$-ої одиниці встаткування, для комп'ютера $\text{П}_{i} = 0,3 \, \text{кВт}/\text{год}$;\newline
\hspace*{15pt} $T_{i}$ $-$ час роботи $i$-ої одиничі встаткування, $T_{1} = 288 \, \text{год}$.

\[
\text{Е}_{\text{тех}} = 1,7808 \cdot 0,3 \cdot 288 = 153,86 \, \text{грн}.
\]

\vspace{1.5em}

\subsubsection{Розрахунок електроенергії, що витрачається на освітлення}

Розрахунок електроенергії, що витрачається на освітлення, виконується виходячи з норм охорони праці по освітленню робочих місць та розраховується наступним чином:

\begin{equation}\label{eq:osv}
\text{Е}_{\text{осв}} = P \cdot N_{\text{л}} \cdot \text{П}_{\text{л}} \cdot T,
\end{equation}

\noindent де $P$ $-$ тариф на електроенергію, $P = 1,7808 \, \text{грн}/\text{кВт}$; \newline
\hspace*{15pt}$N_{\text{л}}$ $-$ кількість ламп, $N_{\text{л}} = 1$;\newline
\hspace*{15pt}$\text{П}_{\text{л}}$ $-$ споживана потужність однієї лампи, $\text{П}_{\text{л}} = 0,1 \, \text{кВт}/\text{год}$;\newline
\hspace*{15pt}$T$ $-$ час роботи ламп для освітлення, $T = 122 \, \text{год}$.

\[
\text{Е}_{\text{осв}} = 1,7808 \cdot 1 \cdot 0,1 \cdot 122 = 22,73 \, \text{грн}.
\]

\vspace{1.5em}

\subsubsection{Амортизаційні відрахування на устаткування}

Амортизацiйнi витрати розраховуються виходячи з формули (\ref{eq:amort}):

\begin{equation}\label{eq:amort}
A = \frac{a_{\text{ЕОМ}}}{12}\sum_{i=1}^{N}\text{ЗВ}_{i} \cdot T_{i},
\end{equation}

\noindent де $a_{\text{ЕОМ}}$ $-$ річна норма амортизації, прийнята в розмірі $ 25 \, \%$ залишкової вартості устаткування;\newline
\hspace*{23pt}$\text{ЗВ}_{i}$ $-$ залишкова вартість $i$-ої одиниці устаткування, $\text{ЗВ}_{1} = 4634,55 \, \text{грн}$;\newline
\hspace*{23pt}$T_{i}$ $-$ час використання $i$-ої одиниці устаткування $T_{1} = 3 \, \text{міс}$.

\[
A = \frac{0,25}{12} \cdot 4634,55 \cdot 3 = 289,66 \, \text{грн}.
\]

\vspace{1.5em}

\subsubsection{Вартість оренди приміщення}

Витрати на оренду примiщення розраховуються виходячи з формули (\ref{eq:orenda}):

\begin{equation}\label{eq:orenda}
\text{Д} = \text{К}_{\text{а}} \cdot S \cdot P \cdot T_{OP},
\end{equation}

\noindent де $\text{К}_{\text{а}}$ $-$ коефіцієнт, що враховує податок на майно, $\text{К}_{\text{а}} = 1,2$;\newline
\hspace*{19pt}$S$ $-$ площа приміщення, де проводилася НДР, $S = 6 \, \text{м}^{2}$;\newline
\hspace*{19pt}$P$ $-$ вартість оренди одного квадратного метра приміщення, $P = 200 \, \text{грн}/\text{міс}$;\newline
\hspace*{19pt}$T_{OP}$ $-$ строк оренди, $T_{OP} = 4 \, \text{міс}$.

\[
\text{Д} = 1,2 \cdot 6 \cdot 200 \cdot 4 = 5760,00 \, \text{грн}.
\]

\vspace{1.5em}

\subsubsection{Інші витрати}

Інші витрати (опалення, робота кондиціонера й ін.), згідно з формулою (\ref{eq:inshi}), приймаються в розмірі $7\%$ від вартості оренди приміщення. 

\begin{equation}\label{eq:inshi}
\text{З}_{\text{ін}} = \text{Д} \cdot \text{К}_{\text{ін}} = 5760,00 \cdot 0,07 = 403,20 \, \text{грн}. 
\end{equation}

\vspace{1.5em}

\subsubsection{Вартість впровадження й освоєння результатів НДР}

Результатом НДР є система яка дозволяє більш якісно використовувати алгоритми contrastive learning. Для використання цієї системи необхідно впровадити та освоїти програми у аналітичних відділах.

При впровадженнi та освоєннi результатiв НДР необхiдно залучити хоча б одного асистента кафедри в науково дослiдному iнститутi з вiдповiдної спецiальностi. Заробiтна плата становить 5000 грн на мiсяць в середньому. На впровадження необхiдно не менше мiсяця. У пiдсумку вартiсть впровадження та освоєння результатiв НДР складе $5000$ грн.

\subsubsection{Витрати на проведення НДР}

Витрати на проведення НДР, згідно з формулою (\ref{eq:vitraty}), являють собою суму витрат по окремих статтях:

\begin{equation}\label{eq:vitraty}
\text{З} = \text{Зп} + \text{Від} + \text{Е}_{\text{тех}} + \text{Е}_{\text{осв}} + \text{А} + \text{Д} + \text{З}_{\text{ін}} + \text{В}_{\text{вп}},
\end{equation}

\noindent де $\text{З}_{\text{ін}}$ $-$ інші витрати; \newline
\hspace*{15pt} $ \text{В}_{\text{вп}}$ $-$ вартість впровадження й освоєння результатів НДР.

Таким чином, сукупнi витрати складають:

\begin{eqnarray*}
\text{З} = 32186,00 + 7080,92 + 153,86 + 22,73 + 289,66 + \\ + 5760,00 + 403,20 + 5000  = 50896,37 \, \text{грн}.
\end{eqnarray*}

\vspace{1.5em}

\subsubsection{Планові накопичення}

Планові накопичення обираються в розмірі $30\%$ від витрат на проведення НДР та наведені у формулі (\ref{eq:plan}).

\begin{equation}\label{eq:plan}
\text{ПН} = 0,3 \cdot 50896,37 = 15268,91 \, \text{грн}.
\end{equation}

\vspace{1.5em}

\subsubsection{Кошторис витрат на проведення НДР}

Кошторис витрат на проведення НДР є сумою витрат на проведення НДР і планових накопичень. Результати розрахунку кошторису витрат представлені у табл. \ref{tab:sumNDR}.

\newpage

\begin{table}[hbt]
	\captionstyle{ \raggedright}
	\caption{Кошторис витрат на проведення НДР}\label{tab:sumNDR}
	%\centering
	\begin{tabular}{|p{0.65\textwidth}|p{0.15\textwidth}|}
		\hline
		Стаття витрат & Сума, грн \\
		\hlinewd{2pt}
		1 Заробiтна плата & 32186  \\
		\hline
		2 Вiдрахування на соцiальне страхування & 7080,92 \\
		\hline
		3 Технологiчна електроенергiя & 153,86 \\
		\hline
		4 Електроенергiя на освiтлення & 22,73 \\
		\hline
		5 Амортизацiйнi вiдрахування на устаткування & 289,66 \\
		\hline
		6 Вартiсть оренди примiщення & 5760 \\
		\hline
		7 Iншi витрати & 403,20 \\
		\hline
		8 Вартiсть впровадження та освоєння НДР & 5000 \\
		\hline
		9 Разом витрат & 50896,37 \\
		\hline
		10 Плановi накопичення & 15268,91 \\
		\hline
		11 Усього кошторис витрат на НДР & 66165,28  \\
		\hline
	\end{tabular}
\end{table}

\subsection{Класифікація й кодування запропонованої інновації}

Основними критеріями класифікації інновацій повинні бути:

\begin{enumerate}
	\item комплексність набору класифікаційних ознак, що враховують, для аналізу й кодування;
	\item можливість кількісного (якісного) визначення критерію;
	\item наукова новизна й практична цінність пропонованої ознаки класифікації
\end{enumerate}

З урахуванням наявного досвіду й наведених критеріїв пропонується наступна класифікація нововведень і інновацій.

За рівнем новизни інновації:

\begin{enumerate}
	\item Радикальні (відкриття, винаходи).
	\item Ординарні (ноу-хау, раціоналізаторські пропозиції).
\end{enumerate}

За стадією життєвого циклу товару, на якій впроваджується інновація:

\begin{enumerate}
	\item Інновації, впроваджувані на стадії стратегічного маркетингу.
	\item Інновації, впроваджувані на стадії НДОКР.
	\item Інновації, впроваджувані на стадії ОТПВ.
	\item Інновації, впроваджувані на стадії виробництва.
	\item Інновації, впроваджувані на стадії сервісного обслуговування.
\end{enumerate}

За масштабом новизни інновації:

\begin{enumerate}
	\item Інновації у світовому масштабі.
	\item Інновації в країні.
	\item Інновації в галузі.
	\item Інновації для підприємства.
\end{enumerate}

За галузю народного господарства, де впроваджується інновація:

\begin{enumerate}
	\item Наука.
	\item Освіта.
	\item Соціальна сфера.
	\item Матеріальне виробництво.
	\item Роботи й послуги.
\end{enumerate}

За сферою застосування інформації:

\begin{enumerate}
	\item Інновації для внутрішнього застосування.
	\item Інновації для нагромадження в організації.
	\item Нововведення для продажу.
\end{enumerate}

За частотою застосування інновації:

\begin{enumerate}
	\item Разові.
	\item Повторювані.
\end{enumerate}

За формою нововведення:

\begin{enumerate}
	\item Відкриття, винаходи, патенти.
	\item Раціоналізаторські пропозиції.
	\item Ноу-хау.
	\item Товарні знаки, торговельні марки, емблеми.
	\item Нові документи, що описують технологічні, виробничі, управлінські процеси, конструкції, структури, методи.
\end{enumerate}

За видом ефекту, отриманого в результаті впровадження інновації:

\begin{enumerate}
	\item Науково-технічний.
	\item Соціальний.
	\item Екологічний.
	\item Економічний (комерційний).
	\item Інтегральний.
\end{enumerate}

За підсистемою системи керування, у якій впроваджується інновація:

\begin{enumerate}
	\item Підсистема наукового супроводу.
	\item Цільова підсистема.
	\item Підсистема, що забезпечує.
	\item Керована підсистема.
	\item Керуюча підсистема.
\end{enumerate}

Наведена класифікація охоплює всі аспекти інноваційної діяльності. Для спрощення управління інноваційною діяльністю на основі цієї класифікації інновації можна кодувати. Кодування може бути спрощене (з одним знаком для ознаки) і детальне (із двома й більше знаками для ознаки). У цьому випадку використовується спрощене кодування, при якому код інновації буде мати 9 цифр. Номер цифри відповідає ознаці класифікації в запропонованому вище порядку, а значення цифри відповідає виду інновації.

Відповідно до наведеної класифікації, код інновації даної НДР 2.4.3.1.1.2.5.1.1.

\subsection{Розрахунок економічного ефекту від впровадження результатів НДР}

Щоб показати доцільність застосування досліджень, проведених у дипломній роботі, необхідно виконати розрахунок економічного ефекту.

Економічний ефект розраховується виходячи із суми, отриманих від впроваджених результатів НДР доходів: 

\begin{equation}\label{eq:economy11}
	\text{Д}_{\text{Е}} = \sum_{i=1}^{n}\text{Д}_{i},
\end{equation}

\noindent де $\text{Д}_{i}$ $-$ величина додаткових доходів або економії коштів, отриманих у результаті впровадження НДР, по $i$-му фактору. 

\vspace{1.5em}

Факторами виступають:

\begin{enumerate}
	\item Економія робочого часу особи, яка приймає рішення. Програмне забезпечення, що допомагає у вирішенні складних задач експертної оцінки об’єктів, зменшує час на прийняття рішення та кількість працівників. Середня заробітна плата фахівця у цих галузях складає 7000 грн на місяць. Зменшення загальної заробітної плати у два рази за рахунок скорочення кількості працівників призведе до економії коштів, що на рік складатиме: $\text{Д}_{1} = \frac{7000,00 \cdot 12}{2} = 42000,00 \, \text{грн}$.
	\item 2	Зниження виробничої площі за рахунок зменшення одиниць устаткування (персональних комп’ютерів). Оскільки програмне забезпечення дозволяє скоротити кількість працівників, відповідно, зменшиться кількість обладнання. Чисельність комп’ютерів в одному офісі – 10 штук. В середньому вартість одного стаціонарного комп’ютера складає 12 500 грн. Оскільки кількість працівників було зменшено вдвічі, то й у два рази зменшиться кількість одиниць устаткування, отже даний проект дозволяє зменшити витрати на 50 \%: $\text{Д}_{2} = 10 \cdot 12500,00 \cdot 0,5 = 62500,00 \, \text{грн}$.
\end{enumerate}

Для розрахунку коштів, отриманих за рахунок цих факторів, необхідно проводити детальний аналіз з залученням фахівців різних областей, що виходить за рамки розрахунку економічного ефекту від даної НДР.

Отже економічний ефект даної НДР складає:

\[
	\text{Д}_{\text{Е}} = 42000,00 + 62500,00 = 104500,00 \, \text{грн}.
\]

\vspace{1.5em}

\subsection{Укрупнена оцінка прибутковості запропонованого інноваційного проекту}

Укрупнена оцінка прибутковості інноваційного проекту дипломної роботи припускає визначення наступних показників:

\begin{enumerate}
	\item Чистий дисконтований доход по роках реалізації проекту.
	\item Чиста поточна вартість проекту по роках реалізації проекту.
	\item Індекс прибутковості проекту.
	\item Внутрішня норма прибутковості.
	\item Строк окупності проекту.
\end{enumerate}

Розрахунок цих показників проводиться виходячи з наступних даних:

\begin{enumerate}
	\item Одноразові витрати в розрахунковому році (кошторис витрат на проведення й впровадження результатів НДР).
	\item Щорічні очікувані доходи від проекту.
	\item Процентна ставка в розрахунковому році.
	\item Інфляція на розглянутому ринку.
	\item Рівень ризику проекту.
\end{enumerate}

Для початку визначимо ставку дисконту проекту по формулі:

\begin{equation}\label{eq:economy12}
	d = k + i + r,
\end{equation}

\noindent де $k$ $-$ ціна капіталу (процентна ставка), частки одиниці, $k = 0,12$;
\hspace*{19pt}$i$ $-$ інфляція на ринку, частки одиниці, $i = 0.15$;
\hspace*{19pt}$r$ $-$ рівень ризику проекту, частки одиниці, $r = 0,06$.

Отже, згідно з формулою \ref{eq:economy12} $d = 0,33$.

Чистий дисконтований доход розраховуємо по формулі:

\begin{equation}\label{eq:economy13}
	\text{ЧДД}_{t} = \frac{\text{Д}_{t} - \text{К}_{t}}{(1 + d)^{t}},
\end{equation}

\noindent де $\text{Д}_{t}$ $-$ доходи $t$-го року, грн;
\hspace*{19pt}$\text{К}_{t}$ $-$ капіталовкладення (витрати) $t$-го року, грн (у цьому випадку  кошторис витрат на НДР).

\vspace{1.5em}

Чисту поточну вартість для $t$-го року реалізації проекту визначаємо по формулі:

\begin{equation}\label{eq:economy14}
	\text{ЧПВ}_{t} = -\frac{\text{К}_{t-1}}{(1 + d)^{t-1}} + \frac{\text{Д}_{t}}{(1 + d)^{t}}.
\end{equation}	

\vspace{1.5em}

Розрахунок даного показника варто здійснювати до першого позитивного значення ЧПВ. Цей рік і завершить розрахунковий період для даного інноваційного проекту.

Розрахунок даного показника варто здійснювати до першого позитивного значення ЧПВ. Цей рік і завершить розрахунковий період для даного інноваційного проекту. Приклад розрахунку чистого дисконтного доходу і чистої поточної вартості приведено в табл. \ref{tab:chdiscdokh}

\newpage

\begin{table}
	\captionstyle{ \raggedright}
	\caption{Розрахунок чистого дисконтного доходу і чистої поточної вартості }\label{tab:chdiscdokh}
	\begin{tabular}{| p{0.02\textwidth} | p{0.12\textwidth} | p{0.10\textwidth} | p{0.08\textwidth} | p{0.11\textwidth} | p{0.11\textwidth} | p{0.12\textwidth} | p{0.12\textwidth} |}
		\hline
		$t$ & \text{Д} & \text{К} & $\frac{1}{(1 + d)^{t}}$ & $\frac{\text{Д}}{(1 + d)^{t}}$ & $\frac{\text{К}}{(1 + d)^{t}}$ & $\text{ЧДД}$ & $\text{ЧПВ}$ \\
		\hlinewd{2pt}
		0 & $-$ & 66165,28 & 1 & $-$ & 66165,28 & -66165,28 & -66165,28 \\
		\hline
		1 & 104500,00 & $-$ & 0,75 & 78375,00 & $-$ & 78375,00 & 12209,72 \\ 
		\hline
		$\Sigma$ & 104500,00 & 66165,28 & $-$ &  78375,00 & 66165,28 & 12209,72 & $-$ \\
		\hline
	\end{tabular}
\end{table}

Індекс прибутковості визначимо за формулою:

\begin{equation}\label{eq:economy15}
	\text{ІП} = \frac{\sum_{t=0}^{T}\frac{\text{Д}}{(1 + d)^{t}}}{\sum_{t=0}^{T}\frac{\text{К}}{(1 + d)^{t}}} = \frac{78375,00}{66165,28} = 1,18,
\end{equation}

\noindent де $T$ $-$ кількість років у розрахунковому періоді.

\vspace{1.5em}

Внутрішня норма прибутковості являє собою ставку дисконту, при якій величина дисконтованих доходів усього розрахункового періоду дорівнює дисконтованим капіталовкладенням. Цей показник допомагає ухвалювати рішення щодо доцільності розробки й впровадження інноваційного проекту в умовах мінливих процентних ставок, ризиках, інфляції.

Внутрішню норму прибутковості можна визначити з табл. \ref{tab:vnp}

\begin{table}[h!]
	\captionstyle{ \raggedright}
	\caption{Розрахунок ВНП}\label{tab:vnp}
	\begin{tabular}{| p{0.18\textwidth} | p{0.18\textwidth} | p{0.18\textwidth} | p{0.18\textwidth} | p{0.18\textwidth} |}
		\hline
		$d$ & 0,33 & 0.5 & 0.57 & 0.58 \\
		\hlinewd{2pt}
		$\text{ЧДД} \, \text{грн}$ & 12406,14 & 3501,38 & 395,22 & -26,03 \\
		\hline
	\end{tabular}
\end{table}

Виходячи з табл. \ref{tab:vnp} ВНП складає приблизно 0,578.

Строк окупності розраховується починаючи з місяця запуску проекту до місяця в якому досягається наступна рівність:

\begin{equation}\label{eq:economy16}
	\sum_{t=1}^{T}\frac{\text{Д}}{(1 + d)^{t}} = \sum_{t=1}^{T}\frac{\text{К}}{(1 + d)^{t}}.
\end{equation}

\vspace{1.5em}

З табл. \ref{tab:chdiscdokh} можна зробити висновок, що термін окупності складає 11 місяців.

\subsection{Висновки за розділом}

У даному розділі дипломної роботи було проведено економічне обґрунтування дослідження параметрів порівняльного навчання для вирішення задач навчання без вчителя.

Проведено ознайомлення з методикою складання кошторису витрат на НДР. Були розраховані витрати на заробітну платню виконавців дипломної роботи, витрати на електроенергію, амортизаційні відрахування, відрахування на соціальне страхування, оренду приміщення, витратні матеріали і планові накопичення. Було встановлено код інновації. Також був зроблений розрахунок економічного ефекту від впровадження результатів НДР, розрахована укрупнена оцінка прибутковості запропонованого проекту та визначено його строк окупності. Даний проект є прибутковим, оскільки індекс прибутковості більше одиниці. Цей показник допомагає ухвалювати рішення щодо доцільності розробки й впровадження інноваційного проекту в умовах мінливих процентних ставок, ризиках, інфляції.

У таблиці \ref{tab:summary} можна ознайомитись з техніко-економічними показниками.

\begin{table}[h!]
	\captionstyle{ \raggedright}
	\caption{Техніко-економічні показники}\label{tab:summary}
	\begin{tabular}{| p{0.4\textwidth} | p{0.3\textwidth} |} 
		\hline
		Найменування показника & Величина \\
		\hlinewd{2pt}
		Кошторис витрат на НДР, грн & 66165б28 \\
		\hline
		Код інновації & 2.4.3.1.1.2.5.1.1 \\
		\hline
		Економічний ефект, грн & 104500,00 \\
		\hline
		Індекс прибутковості проекту & 1,18 \\
		\hline
		Внутрішня норма прибутковості & 0,578 \\
		\hline
		Строк окупності проекту, міс & 11 \\
		\hline
	\end{tabular}
\end{table}

