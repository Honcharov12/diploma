%\anonsection{Список джерел інформації}
\anonsection{СПИСОК ДЖЕРЕЛ ІНФОРМАЦІЇ}
\addcontentsline{toc}{section}{Список джерел інформації}

\hspace*{26pt} 1 Тимофеев В. С. Эконометрика: учебник для бакалавров / В. С. Тимофеев, А. В. Фадеенков, В. Ю. Щеколдин. $-$ 2-е изд., перераб. и доп. $-$ М.: Издательство Юрайт, 2013. $-$ 328 с. $-$ Серия: Бакалавр. Базовый курс.

2 Кизбикенов, К. О. Прогнозирование и временные ряды [Електроний ресурс] : учебное пособие / К. О. Кизбикенов. $–$ Барнаул : АлтГПУ, 2017.

3 Орлов А. И. Прикладная статистика $-$ М.: Издательство «Экзамен», 2004.

4 Stud.com.ua $-$ [Електронний ресурс] // https://m.stud.com.ua : Моделі і методи авторегресії URL: https://m.stud.com.ua/52035/ekonomika/modeli\_metodi\_avtoregresiyi

5 Голяндина Н. Э. Метод «Гусеница»$-$SSA: анализ временных рядов: Учеб. пособие. $-$ СПб., 2003. $-$ 87с.

6 Erricos J. 2005. «Handbook of Parallel Computing and Statistics»

7 Python $-$ [Електронний ресурс] // https://www.python.org :  The Python Tutoria URL: https://docs.python.org/3/tutorial/index.html

8 Muller A and Guido S (2016). «Introduction to Machine Learning with Python: A Guide for Data Scientists». O'Reilly Media

9 Рашка С. Python и машинное обучение / пер. с англ. А. В. Логунова. $-$ М.: ДМК Пресс, 2017. $-$ 418 с.: ил.

10 Уэс Маккинли Python и анализ данных/ Пер. с англ. Слинкин А. А. $-$ М.: ДМК Пресс, 2015. $-$ 482 с.: ил. 

11 NumPy $-$ [Електронний ресурс] // https://www.numpy.org

12 Кодекс цивільного захисту України – ВРУ №5403-VI, від 2.10.2012 р.

13 Концепція «Про захист населення і територій при загрозі і виникненні надзвичайної ситуації», схвалена Наказом Президента України від 26.03.1999 року № 234/99.

14 Кулаков М. А. Цивільна оборона : навч. посіб. / М. А. Кулаков, Т. В. Кукленко, В. О. Ляпун, В. О. Мягкий. – Х. : Факт, 2008. – 157 с.

15 Стеблюк М. І. Цивільна оборона  : підруч. 3-тє вид., перероб. і доп. / М. І. Стеблюк. – К.: Знання, 2004. – 332 с.
