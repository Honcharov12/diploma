%\anonsection{Список джерел інформації}
\anonsection{СПИСОК ДЖЕРЕЛ ІНФОРМАЦІЇ}
\addcontentsline{toc}{section}{Список джерел інформації}

\hspace*{26pt} 1 Тимофеев В. С. Эконометрика: учебник для бакалавров / В. С. Тимофеев, А. В. Фадеенков, В. Ю. Щеколдин. $-$ 2-е изд., перераб. и доп. $-$ М.: Издательство Юрайт, 2013. $-$ 328 с. $-$ Серия: Бакалавр. Базовый курс.

2 Кизбикенов, К. О. Прогнозирование и временные ряды [Електроний ресурс] : учебное пособие / К. О. Кизбикенов. $–$ Барнаул : АлтГПУ, 2017.

3 Орлов А. И. Прикладная статистика $-$ М.: Издательство «Экзамен», 2004.

4 Stud.com.ua $-$ [Електронний ресурс] // https://m.stud.com.ua : Моделі і методи авторегресії URL: https://m.stud.com.ua/52035/ekonomika/modeli\_metodi\_avtoregresiyi

5 Studbooks.net $-$ [Електронний ресурс] // https://studbooks.net : Идентификация модели ARMA URL: https://studbooks.net/1875074/ekonomika/identifikatsiya\_modeli\_arma

6 Дзендзелюк О. Побудова ARIMA моделей часових рядів для прогнозування метеоданих на мові програмування R / О. Дзендзелюк, Л. Костів, В. Рабик // Електроніка та інформаційні технології. $-$ 2013. $-$ Вип. 3. - С. 211-219.

7 Machine Learning $-$ Notes on [Електронний ресурс] // http://strijov.com : Сингулярное разложение URL: \newline http://strijov.com/files/eksamen/l\_svd.pdf

8 Голяндина Н. Э. Метод «Гусеница»$-$SSA: анализ временных рядов: Учеб. пособие. $-$ СПб., 2003. $-$ 87с.

9 Erricos J. 2005. «Handbook of Parallel Computing and Statistics»

10 Python $-$ [Електронний ресурс] // https://www.python.org :  The Python Tutoria URL: https://docs.python.org/3/tutorial/index.html

11 Muller A and Guido S (2016). «Introduction to Machine Learning with Python: A Guide for Data Scientists». O'Reilly Media

12 Рашка С. Python и машинное обучение / пер. с англ. А. В. Логунова. $-$ М.: ДМК Пресс, 2017. $-$ 418 с.: ил.

13 Уэс Маккинли Python и анализ данных/ Пер. с англ. Слинкин А. А. $-$ М.: ДМК Пресс, 2015. $-$ 482 с.: ил. 

14 NumPy $-$ [Електронний ресурс] // https://www.numpy.org

15 Закон України про охорону праці від 21.11.2002 року.

16 Закон України про охорону навколишнього середовища від 25.06.1991 року.

17 ДСН 3.3.6.042-99. Санітарні норми мікроклімату виробничих приміщень // Затверджено постановою Головного санітарного лікаря України від 01 грудня 1999 року, №42.

18 ДБН В.2.5-67-2013. Державні будівельні норми України. Опалення, вентиляція та кондиціонування. $-$ Чинний від 01.01. 2014 // Затверджений наказами Міністерства регіонального розвитку, будівництва та житлово-комунального господарства України від 25.01.2013 року, №24 та від 28.08.2013 року, № 410.

19 ДБН В.2.5-28-2018 Державні будівельні норми. Інженерне обладнання будинків і споруд. Природне і штучне освітлення. $-$ К. : Мінбуд України, 2018. 

20 ДСН 3.3.6.037-99. Санітарні норми виробничого шуму, ультразвуку та інфразвуку // Затверджено постановою Головного санітарного лікаря України від 01 грудня 1999 року, № 37.

21 ДСН 3.3.6. 039-99 Державні санітарні норми виробничої загальної та локальної вібрації // Затв. постановою Головного санітарного лікаря України від 01 грудня 1999 року №39.

22 ПУЕ-2017, Правила улаштування електроустановок. $-$ Чинний від 21.07.2017.

23 НПАОП 0.00-1.28-2010. Про затвердження правил охорони праці під час експлуатації електронно-обчислювальних машин. $-$ К., 2010.

24 ДСТУ Б.В.1.1:36-2016. Норми визначення категорій приміщень, будинків та зовнішніх установок за вибухопожежною та пожежною небезпекою. $-$ України, 2016. 

25 ДБН В.1.1-7:2016. Державні будівельні норми України. Пожежна безпека об’єктів будівництва. Загальні положення.

26 ДБН В.2.5-56-2015. Системи протипожежного захисту. $-$ Чинний від 01.07.2015.
