%\anonsection{Список джерел інформації}
\anonsection{СПИСОК ДЖЕРЕЛ ІНФОРМАЦІЇ}
\addcontentsline{toc}{section}{Список джерел інформації}

\hspace*{26pt} 1 

2 

3

4 

5 

6 

7 

8 

9 

10 

11 

12 Методичні вказівки для розробки розділу «Охорона праці та навколишнього середовища» у випускних дипломних роботах студентів інженерно $-$ фізичного факультету та факультету «Інформатика і управління» очної та заочної форм навчання / Уклад. В. В. Березуцький, О. О. Кузьменко, М. М. Латишева. $-$ Харків: НТУ «ХПІ», 2020.$-$ 60 с. $-$ Укр. мовою.

13 Кодекс цивільного захисту України – ВРУ №5403-VI, від 2.10.2012 р.

14 Концепція «Про захист населення і територій при загрозі і виникненні надзвичайної ситуації», схвалена Наказом Президента України від 26.03.1999 року № 234/99.

15 Кулаков М. А. Цивільна оборона : навч. посіб. / М. А. Кулаков, Т. В. Кукленко, В. О. Ляпун, В. О. Мягкий. – Х. : Факт, 2008. – 157 с.

16 Стеблюк М. І. Цивільна оборона  : підруч. 3-тє вид., перероб. і доп. / М. І. Стеблюк. – К.: Знання, 2004. – 332 с.
