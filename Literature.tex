%\anonsection{Список джерел інформації}
\anonsection{СПИСОК ДЖЕРЕЛ ІНФОРМАЦІЇ}
\addcontentsline{toc}{section}{Список джерел інформації}

\hspace*{26pt} 
1 Stack Exchange $-$ [Електронний ресурс] // https://stackexchange.com/ What is self-supervised learning in machine learning?: https://ai.stackexchange.com/questions/10623/what-is-self-supervised-learning-in-machine-learning

2 Анализ малых данных $-$ [Електронний ресурс] // dyakonov.org : Самообучение\hfill (Self-Supervision):\hfill https://dyakonov.org/2020/06/03/\newline самообучение-self-supervision/

3 Xingyi Yang (2020). «Transfer Learning or Self-supervised Learning? A Tale of Two Pretraining Paradigms». ICLR.

4 R Devon Hjelm, et al. (2019). «LEARNING DEEP REPRESENTATIONS BY MUTUAL INFORMATION ESTIMATION AND MAXIMIZATION». ICLR.

5 Kaiming He, Haoqi Fan, Yuxin Wu, Saining Xie and Ross Girshick (2020). «Momentum Contrast for Unsupervised Visual Representation Learning». Facebook AI Research.

6 towards data science $-$ [Електронний ресурс] // https://towardsdatascience.com/ : Understanding Contrastive Learning:\newline https://towardsdatascience.com/understanding-contrastive-learning-d5b19fd96607

7 Python $-$ [Електронний ресурс] // https://www.python.org :  The Python Tutoria URL: https://docs.python.org/3/tutorial/index.html

8 Muller A and Guido S (2016). «Introduction to Machine Learning with Python: A Guide for Data Scientists». O'Reilly Media

9 Рашка С. Python и машинное обучение / пер. с англ. А. В. Логунова. $-$ М.: ДМК Пресс, 2017. $-$ 418 с.: ил.

10 towards data science $-$ [Електронний ресурс]  // https://towardsdatascience.com/\newline : What is PyTorch?: https://towardsdatascience.com/what-is-pytorch-a84e4559f0e3

11 NumPy $-$ [Електронний ресурс] // https://www.numpy.org

12 Matplotlib $-$ [Електронний ресурс] // https://matplotlib.org

13 Computer Science UNIVERSITY OF TORONTO $-$ [Електронний ресурс] // https://web.cs.toronto.edu/ : The CIFAR-10 dataset: https://www.cs.toronto.edu/~kriz/cifar.html 

14 Методичні вказівки для розробки розділу «Охорона праці та навколишнього середовища» у випускних дипломних роботах студентів інженерно $-$ фізичного факультету та факультету «Інформатика і управління» очної та заочної форм навчання / Уклад. В. В. Березуцький, О. О. Кузьменко, М. М. Латишева. $-$ Харків: НТУ «ХПІ», 2020.$-$ 60 с. $-$ Укр. мовою.

15 Кодекс цивільного захисту України – ВРУ №5403-VI, від 2.10.2012 р.

16 Концепція «Про захист населення і територій при загрозі і виникненні надзвичайної ситуації», схвалена Наказом Президента України від 26.03.1999 року № 234/99.

17 Кулаков М. А. Цивільна оборона : навч. посіб. / М. А. Кулаков, Т. В. Кукленко, В. О. Ляпун, В. О. Мягкий. – Х. : Факт, 2008. – 157 с.

18 Стеблюк М. І. Цивільна оборона  : підруч. 3-тє вид., перероб. і доп. / М. І. Стеблюк. – К.: Знання, 2004. – 332 с.

19 Окнянське відділення державного професійно-технічного навчального закладу«Захарівський професійний ліцей» $-$ [Електронний ресурс] // http://krasniokny-fpl.edukit.od.ua/ : Як працює оповіщення про надзвичайні ситуації: http://krasniokny-fpl.edukit.od.ua/ohorona_praci_i_bzhd/yak_pracyuye_opovischennya_pro_nadzvichajni_situacii/

20 В.В. Ляхов, В.О. Васійчук, С.М. Мирошниченко Цивільний захист: Конспект лекцій для проведення занять з цивільного захисту зі співробітниками НУ «Львівська політехніка».- Львів: Видавництво Національного університету «Львівська політехніка». 2020 – 80 с.
